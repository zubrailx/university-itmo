%----document---------------------------------------------------------------------------------
\documentclass[11pt, a4paper]{article}
\usepackage{geometry}
% total - determines printable width, height
\geometry{ 
	a4paper, total={160mm,267mm}
}
%----utility---------------------------------------------------------------------------------
\usepackage{filecontents}
\usepackage{xcolor}
\usepackage{import} % relative importing
%----text,fonts------------------------------------------------------------------------------------
\usepackage{mmap}
\usepackage[T2A]{fontenc}
\usepackage[utf8]{inputenc}
\usepackage[russian,english]{babel}
\usepackage{setspace}
\setstretch{0,9}
\usepackage{fancyvrb}
\usepackage{hyperref}
\hypersetup{
    colorlinks=true,
    linkcolor=blue,
    filecolor=magenta,      
    urlcolor=blue,
    pdftitle={Title},
    pdfpagemode=FullScreen,
}
%----math,graphics---------------------------------------------------------------------------------
\usepackage{amsmath,amsfonts,amssymb}
\usepackage{amsthm}
\usepackage{tikz}
\usetikzlibrary{calc}
\usepackage{pgfplots}
\pgfplotsset{
	compat=1.17
}

\usepackage{graphicx}
\graphicspath{{fig/}}
  
\usepackage{wrapfig}
\usepackage{tabularx}
%----programming---------------------------------------------------------------------------------
\usepackage{listings}
%------------------------------------------------------------------------------------------------

\definecolor{mGreen}{rgb}{0,0.6,0}
\definecolor{mGray}{rgb}{0.5,0.5,0.5}
\definecolor{mPurple}{rgb}{0.58,0,0.82}
\definecolor{backgroundColour}{rgb}{0.95,0.95,0.92}

\lstdefinestyle{User-C}
{
  backgroundcolor=\color{backgroundColour},   
  commentstyle=\color{mGreen},
  keywordstyle=\color{magenta},
  numberstyle=\tiny\color{mGray},
  stringstyle=\color{mPurple},
  basicstyle=\footnotesize\fontfamily{cmtt}\selectfont,
  breakatwhitespace=false,         
  breaklines=true,                 
  captionpos=b,                    
  keepspaces=true,                 
  numbers=left,                    
  numbersep=5pt,                  
  showspaces=false,                
  showstringspaces=false,
  showtabs=false,                  
  tabsize=2,
  language=C
}

\lstdefinestyle{User-scripts}
{
  backgroundcolor=\color{backgroundColour},   
  commentstyle=\color{mGreen},
  keywordstyle=\color{magenta},
  numberstyle=\tiny\color{mGray},
  stringstyle=\color{mPurple},
  basicstyle=\footnotesize\fontfamily{cmtt}\selectfont,
  breakatwhitespace=false,         
  breaklines=true,                 
  captionpos=b,                    
  keepspaces=true,                 
  numbers=left,                    
  numbersep=5pt,                  
  showspaces=false,                
  showstringspaces=false,
  showtabs=false,                  
  tabsize=2,
  language=bash
}

\lstdefinestyle{User-output}
{
  backgroundcolor=\color{backgroundColour},   
  commentstyle=\color{mGreen},
  keywordstyle=\color{magenta},
  numberstyle=\tiny\color{mGray},
  stringstyle=\color{mPurple},
  basicstyle=\footnotesize\fontfamily{cmtt}\selectfont,
  breakatwhitespace=false,         
  breaklines=true,                 
  captionpos=b,                    
  keepspaces=true,                 
  numbers=none,                    
  showspaces=false,                
  showstringspaces=false,
  showtabs=false,                  
  tabsize=2,
}


\begin{document}

\import{.}{titular.tex}
\newpage

\section{Описание задания}

\begin{enumerate}
    \item Датасет с данными про оценки студентов инженерного и педагогического факультетов (для данного датасета нужно ввести метрику: студент успешный/неуспешный на основании грейда)
    \item Отобрать случайным образом sqrt(n) признаков.
    \item Реализовать без использования сторонних библиотек построение дерева решений (numpy и pandas использовать можно).
    \item Провести оценку реализованного алгоритма с использованием Accuracy, precision и recall
    \item Построить AUC-ROC и AUC-PR.
\end{enumerate}

\section{Реализация.}

\subsection{Подсчет количества признаков}

Общее количество признаков в датасете - 32. Тогда требуемое количество: 

$$
  k = \lceil \sqrt{32} \rceil = 6
$$

\subsection{Реализация и вывод формул}

Обратиться к пункту 3.

\subsection{Оценка реализованного алгоритма с использованием accuracy, precision и recall}

\begin{Verbatim}
Accuracy:  0.6551724137931034
Precision:  0.6153846153846154
Recall:  0.6153846153846154
\end{Verbatim}

\subsection{AUC-ROC и AUC-PR}

\begin{figure}[ht]
  \centering
  \includegraphics[width=0.7\textwidth]{auc-roc.png}
  % \label{fig:result-png}
\end{figure}

\newpage

\begin{figure}[ht]
  \centering
  \includegraphics[width=0.7\textwidth]{auc-pr.png}
  % \label{fig:result-png}
\end{figure}

\section{Исходный код}

 \noindent Результаты, полученные в ходе выполнения лабораторной работы: \href{https://github.com/zubrailx/University-ITMO/blob/main/Year-3/Artificial-intelligence/lab-3}{sources}.
 
\section{Вывод.}

\noindent В ходе выполнения лабораторной работы я познакомился с алгоритмом C4.5 и реализовал его на питоне. Также
вспомнил, что такое accuracy, precision, recall, а также разобрался, как строить графики AUC-ROC, AUC-PR.

\medskip
\noindent Графики AUC-PR, AUC-ROC довольно неинформативны, поскольку размер входных данных составляет 150 строк, тестовое разбиение 3:7. Именно поэтому количество данных, по которым, учится алгоритм, а также количество тестовых данных, мало, что свидетельствует о небольшом кол-ве точек на графике.

\end{document}
