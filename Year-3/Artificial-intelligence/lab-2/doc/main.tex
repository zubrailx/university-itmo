%----document---------------------------------------------------------------------------------
\documentclass[11pt, a4paper]{article}
\usepackage{geometry}
% total - determines printable width, height
\geometry{ 
	a4paper, total={160mm,267mm}
}
%----utility---------------------------------------------------------------------------------
\usepackage{filecontents}
\usepackage{xcolor}
\usepackage{import} % relative importing
%----text,fonts------------------------------------------------------------------------------------
\usepackage{mmap}
\usepackage[T2A]{fontenc}
\usepackage[utf8]{inputenc}
\usepackage[russian,english]{babel}
\usepackage{setspace}
\setstretch{0,9}
\usepackage{fancyvrb}
\usepackage{hyperref}
\hypersetup{
    colorlinks=true,
    linkcolor=blue,
    filecolor=magenta,      
    urlcolor=blue,
    pdftitle={Overleaf Example},
    pdfpagemode=FullScreen,
    }
%----math,graphics---------------------------------------------------------------------------------
\usepackage{amsmath,amsfonts,amssymb}
\usepackage{amsthm}
\usepackage{tikz}
\usetikzlibrary{calc}
\usepackage{pgfplots}
\pgfplotsset{
	compat=1.17
}

\usepackage{graphicx}
\graphicspath{{image/}}
  
\usepackage{wrapfig}
\usepackage{tabularx}
%----programming---------------------------------------------------------------------------------
\usepackage{listings}
%------------------------------------------------------------------------------------------------

\definecolor{mGreen}{rgb}{0,0.6,0}
\definecolor{mGray}{rgb}{0.5,0.5,0.5}
\definecolor{mPurple}{rgb}{0.58,0,0.82}
\definecolor{backgroundColour}{rgb}{0.95,0.95,0.92}

\lstdefinestyle{User-C}
{
  backgroundcolor=\color{backgroundColour},   
  commentstyle=\color{mGreen},
  keywordstyle=\color{magenta},
  numberstyle=\tiny\color{mGray},
  stringstyle=\color{mPurple},
  basicstyle=\footnotesize\fontfamily{cmtt}\selectfont,
  breakatwhitespace=false,         
  breaklines=true,                 
  captionpos=b,                    
  keepspaces=true,                 
  numbers=left,                    
  numbersep=5pt,                  
  showspaces=false,                
  showstringspaces=false,
  showtabs=false,                  
  tabsize=2,
  language=C
}

\lstdefinestyle{User-scripts}
{
  backgroundcolor=\color{backgroundColour},   
  commentstyle=\color{mGreen},
  keywordstyle=\color{magenta},
  numberstyle=\tiny\color{mGray},
  stringstyle=\color{mPurple},
  basicstyle=\footnotesize\fontfamily{cmtt}\selectfont,
  breakatwhitespace=false,         
  breaklines=true,                 
  captionpos=b,                    
  keepspaces=true,                 
  numbers=left,                    
  numbersep=5pt,                  
  showspaces=false,                
  showstringspaces=false,
  showtabs=false,                  
  tabsize=2,
  language=bash
}

\lstdefinestyle{User-output}
{
  backgroundcolor=\color{backgroundColour},   
  commentstyle=\color{mGreen},
  keywordstyle=\color{magenta},
  numberstyle=\tiny\color{mGray},
  stringstyle=\color{mPurple},
  basicstyle=\footnotesize\fontfamily{cmtt}\selectfont,
  breakatwhitespace=false,         
  breaklines=true,                 
  captionpos=b,                    
  keepspaces=true,                 
  numbers=none,                    
  showspaces=false,                
  showstringspaces=false,
  showtabs=false,                  
  tabsize=2,
}


\begin{document}

\import{.}{titular.tex}
\newpage

\section{Описание задания}

\noindent
Описание предметной области. Имеется транспортная сеть, связывающая
города СНГ. Сеть представлена в виде таблицы связей между городами. Связи
являются двусторонними, т.е. допускают движение в обоих направлениях.
Необходимо проложить маршрут из одной заданной точки в другую.

\medskip\noindent
Этап 1. Неинформированный поиск. На этом этапе известна только
топология связей между городами. Выполнить:
\begin{enumerate}
  \item поиск в ширину;
  \item поиск глубину;
  \item поиск с ограничением глубины;
  \item поиск с итеративным углублением;
  \item двунаправленный поиск.
\end{enumerate}

\noindent
Отобразить движение по дереву на его графе с указанием сложности
каждого вида поиска. Сделать выводы.

\medskip\noindent
Этап 2. Информированный поиск. Воспользовавшись информацией о
протяженности связей от текущего узла, выполнить:

\begin{enumerate}
  \item жадный поиск по первому наилучшему соответствию;
  \item затем, использую информацию о расстоянии до цели по прямой от
  каждого узла, выполнить поиск методом минимизации суммарной оценки А*.
\end{enumerate}

\noindent
Отобразить на графе выбранный маршрут и сравнить его сложность с
неинформированным поиском. Сделать выводы.

\section{Листинг и результаты.}

\noindent Код алгоритмов: \href{https://github.com/zubrailx/University-ITMO/blob/main/Year-3/Artificial-intelligence/lab-2/src/algorithms.cpp}{src/algorithms.cpp}.

\noindent Весь код с утилитами: \href{https://github.com/zubrailx/University-ITMO/tree/main/Year-3/Artificial-intelligence/lab-2/src}{src}.

\noindent Вывод программы: \href{https://github.com/zubrailx/University-ITMO/blob/main/Year-3/Artificial-intelligence/lab-2/test/out.1}{test/out.1}.

\section{Вывод.}

\noindent В ходе выполнения данной лабораторной работы познакомился с алгоритмами поиска, которые реализовал на языке c++. Кроме того, 
немного разобрался с латехом.


\end{document}
