%форматирование размера документа
\documentclass[11pt, a4paper]{article}

\usepackage{geometry}
% total - determines printable width, height
\geometry{ 
	a4paper, total={160mm,267mm}
}

%----text,fonts------------------------------------------------------------------------------------
\usepackage{mmap}
\usepackage[T2A]{fontenc}
\usepackage[utf8]{inputenc}
\usepackage[english, russian]{babel}
\usepackage{setspace}
\setstretch{0,9}
\usepackage{fancyvrb}

%----math,graphics---------------------------------------------------------------------------------
\usepackage{amsmath,amsfonts,amssymb}
\usepackage{amsthm}
\usepackage{listings}
\usepackage{xcolor}
\usepackage{filecontents}

\usepackage{tikz}
\usetikzlibrary{calc}
\usepackage{pgfplots}
\pgfplotsset{
	compat=1.17
}

\usepackage{graphicx}
\graphicspath{{image/}}
  
\usepackage{wrapfig}
\usepackage{tabularx}

% relative importing
\usepackage{import}

% \definecolor{mGreen}{rgb}{0,0.6,0}
\definecolor{mGray}{rgb}{0.5,0.5,0.5}
\definecolor{mPurple}{rgb}{0.58,0,0.82}
\definecolor{backgroundColour}{rgb}{0.95,0.95,0.92}

\lstdefinestyle{User-C}
{
  backgroundcolor=\color{backgroundColour},   
  commentstyle=\color{mGreen},
  keywordstyle=\color{magenta},
  numberstyle=\tiny\color{mGray},
  stringstyle=\color{mPurple},
  basicstyle=\footnotesize\fontfamily{cmtt}\selectfont,
  breakatwhitespace=false,         
  breaklines=true,                 
  captionpos=b,                    
  keepspaces=true,                 
  numbers=left,                    
  numbersep=5pt,                  
  showspaces=false,                
  showstringspaces=false,
  showtabs=false,                  
  tabsize=2,
  language=C
}

\lstdefinestyle{User-scripts}
{
  backgroundcolor=\color{backgroundColour},   
  commentstyle=\color{mGreen},
  keywordstyle=\color{magenta},
  numberstyle=\tiny\color{mGray},
  stringstyle=\color{mPurple},
  basicstyle=\footnotesize\fontfamily{cmtt}\selectfont,
  breakatwhitespace=false,         
  breaklines=true,                 
  captionpos=b,                    
  keepspaces=true,                 
  numbers=left,                    
  numbersep=5pt,                  
  showspaces=false,                
  showstringspaces=false,
  showtabs=false,                  
  tabsize=2,
  language=bash
}

\lstdefinestyle{User-output}
{
  backgroundcolor=\color{backgroundColour},   
  commentstyle=\color{mGreen},
  keywordstyle=\color{magenta},
  numberstyle=\tiny\color{mGray},
  stringstyle=\color{mPurple},
  basicstyle=\footnotesize\fontfamily{cmtt}\selectfont,
  breakatwhitespace=false,         
  breaklines=true,                 
  captionpos=b,                    
  keepspaces=true,                 
  numbers=none,                    
  showspaces=false,                
  showstringspaces=false,
  showtabs=false,                  
  tabsize=2,
}


\begin{document}

\import{.}{titular.tex}

\newpage

\section{Текст задания}
\noindentОсновная цель лабораторной работы - знакомство с системными инструментами анализа производительности и поведения программ. Для этого предлагается для выданной по варианту программы выяснить следующую информацию:

\begin{enumerate}
  \item Количество потоков создаваемое программой;
  \item Список файлов и сетевых соединений с которыми работает программа
  \item Карту памяти процесса;
  \item Содержимое передаваемых по сети данных;
  \item Построить графики:
  \begin{itemize}
    \item Потребления программой cpu;
    \item Нагрузки генерируемой программой на подсистему ввода-вывода;
    \item Нагрузки генерируемой программой на сетевую подсистему.
    \item Смены состояния исполнения потоков;
  \end{itemize}
\end{enumerate}

\smallskip

\noindentСодержание отчета:
\begin{enumerate}
  \item Описание шагов выполненных для сбора информации (включая исходные тексты всех использованных скриптов и вспомогательных программ);
  \item  Полученные графики;
  \item  Выводы по работе.
\end{enumerate}

\noindentТемы для подготовки к защите лабораторной работы:
\begin{enumerate}
  \item Структура процесса;
  \item Виртуальная память;
  \item Системные утилиты сбора статистики ядра;
  \item Основы ввода-вывода (блочный и последовательный ввод-вывод);
  \item Файловая система procfs;
  \item Использование утилиты strace, ltrace, bpftrace;
  \item Профилирование и построение flamegraph'а и stap;
\end{enumerate}

\section{Выполнение}

\section{Вывод}
\end{document}
