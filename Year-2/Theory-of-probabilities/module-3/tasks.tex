%! TEX root = main.tex

\section{Условные вероятности. Независимость событий.}

%---------------------------------------------------------------------------------------------------
\begin{task}{1}
Брошено две игральные кости. Какова вероятность того, что выпало две <<3>>, если известно, что сумма
выпавших очков делится на три?
\end{task}

\begin{solution}
Пусть $P(A)$ - вероятность того, что сумма выпавших очков делится на три, $P(B)$ - вероятность 
выпадения двух <<3>>. Тогда $P(A) = \frac{12}{36}$, так как $\Omega$ состоит из 36 элементарных 
событий, а кол-во событий выпадения суммы, делящейся на 3: $A =  \{\left(1, 2\right) , \left(
1, 5\right) , \left(2, 1\right) , \left(2, 4\right) , \left(3, 3\right) \left(3, 6\right) ,
\left(4, 2\right) , \left(4, 5\right) , \left(5, 1\right) , \left(5, 4\right) , \left(6, 3\right) ,
\left(6, 6\right) \}$, а мощность данного множества равна 12. $P(B) = P(AB) = \frac{1}{36}$, а значит
$P(B | A) = \frac{P(AB)}{P(A)} = \frac{1}{12}$.
\end{solution}

\begin{result}
  $\frac{1}{12}$.
\end{result}

%---------------------------------------------------------------------------------------------------
\medskip
\begin{task}{2}
  Известно, что при бросании 10 игральных костей появилась по крайне мере одна <<1>>. Какова 
  вероятность того, что появилось две <<1>> или более?
\end{task}

\begin{solution}
По формуле условной вероятности $P(B | A) = \frac{P(AB)}{P(A)}$, где $P(A)$ - вероятность 
выпадения по крайней мере одной <<1>>, $P(AB)$ - вероятность выпадения двух <<1>> или более. 

\begin{flalign*}
  P(A) = \sum_{i = 1}^{10} C_{10}^i (\frac{5}{6})^{10 - i} (\frac{1}{6})^i, \;\;\;
  P(AB) = \sum_{i = 2}^{10} C_{10}^i (\frac{5}{6})^{10 - i} (\frac{1}{6})^i, \;\;\; 
  \frac{P(AB)}{P(A)} \approx 0.6147724 &&
\end{flalign*}

\begin{verbatim}
#!/usr/bin/python

p_a  = sum([c_n_k(10, i) * (5/6)**(10 - i) * (1/6) ** i for i in range(1, 11)])
p_ab = sum([c_n_k(10, i) * (5/6)**(10 - i) * (1/6) ** i for i in range(2, 11)])
print(p_ab/p_a)
\end{verbatim}
\end{solution}

\begin{result}
  $\approx 0.6147724$
\end{result}

%---------------------------------------------------------------------------------------------------
\medskip
\begin{task}{3}
  Из множества $\left\{1, 2, \ldots, N\right\} $ по схеме случайного выбора без возвращения
  выбираются три числа. Найти условную вероятность того, что третье число попадет в интервал,
  образованный первыми двумя, если известно, что первое число меньше второго.
\end{task}

\begin{solution}
Есть несколько вариантов решения: напрямую посчитать вероятность по формуле условной вероятности 
или воспользоваться здравым смыслом и сразу же назвать ответ.

\medskip
\noindentЕсли воспользоваться вторым вариантом, то так как по условию первое 
число должно быть меньше второго, то возможны следующие конфигурации: $\left\{\left(
i_1, i_2, i_3\right) , \left(i_1, i_3, i_2\right) , \left(i_3, i_1, i_2\right) \right\} $, где
$i_k, \; k \in {1, 2, 3}$ - число, которое мы выбрали первым, вторым, третьим (между числами 
могут лежать или не лежать другие числа). Нам подходит только
вторая конфигурация из 3, а значит вероятность $\frac{1}{3}$.
  
\medskip
\noindentЕсли считать напрямую,то тогда вероятность выбрать 2 числа так, чтобы первое было 
меньше второго: $P(A) = \frac{C_{n}^2}{n\cdot (n - 1)}$, а вероятность, что в таком случае 
третье число окажется между вторым: 

\[
  P(AB) = \frac{\sum_{i = 1}^n \sum_{j = i + 1}^n (j - i - 1)}{n\cdot (n - 1) \cdot (n - 2)}
.\] 

\noindentЕсли посчитать, то скорее всего окажется то, что нам требуется.
\end{solution}

\begin{result}
  $\frac{1}{3}$
\end{result}

%---------------------------------------------------------------------------------------------------
\medskip
\begin{task}{4}
 Из урны, содержащей 3 белых и 5 черных шаров, последовательно без возвращения извлекают 8 шаров. 
 Пусть $A_0^{(i)} (A_1^{(i)}$ - событие, состоящее в том, что i-й шар был черный (белый). Найти 
 условные вероятности:
\end{task}

\begin{subtask}{1}
  $P(A_1^{(5)} | A_1^{(1)} A_0^{(2)} A_0^{(3)} A_1^{(4)})$ 
\end{subtask}

\begin{solution}
\begin{equation*}
  P(A_1^{(5)} | A_1^{(1)} A_0^{(2)} A_0^{(3)} A_1^{(4)}) = \frac{P\left(A_1^{(5)} A_1^{(1)} A_0^{(2)} A_0^{(3)} A_1^{(4)}\right)}
  {P\left(A_1^{(1)} A_0^{(2)} A_0^{(3)} A_1^{(4)}\right)} = \frac{\frac{3}{8} \cdot \frac{5}{7} \cdot \frac{4}{6}\cdot \frac{2}{5}\cdot \frac{1}{4}}
  {\frac{3}{8} \cdot \frac{5}{7} \cdot \frac{4}{6}\cdot \frac{2}{5}\cdot} = \frac{1}{4}
\end{equation*}
\end{solution}


\begin{subtask}{2}
  $P(A_0^{(4)} | A_{\alpha_1}^{(1)} A_{\alpha_2}^{(2)} A_{\alpha_3}^{(3)}), \left(\alpha_1, \alpha_2, \alpha_3\right) =
  \left(0, 0, 1\right), \left(0, 1, 0\right), \left(1, 0, 0\right) $
\end{subtask}

\begin{solution}
  Поскольку на момент появления события $A_0^{(4)}$ в урне остается лежать всегда  3 черных и 2 белых
  шара, то вероятность выпадения черного шара равна $\frac{3}{5}$.

\medskip
\noindentЕсли считать напрямую: $P_1(\alpha_1, \alpha_2, \alpha_3) = \frac{5}{8} \cdot  \frac{4}{7} \cdot 
\frac{3}{6}$, $P_2(\alpha_1, \alpha_2, \alpha_3) = \frac{5}{8} \cdot \frac{3}{7} \cdot \frac{4}{6}$, 
$P_3(\alpha_1, \alpha_2, \alpha_3) = \frac{3}{8} \cdot  \frac{5}{7} \cdot \frac{4}{6}$. 
$P(A) = \sum_{i = 1}^3 = \frac{60 + 60 + 60}{336}$.  $P(AB) = \sum_{i = 1}^3 P_i(A) * P(A_0^{(4)}) = 
\frac{3}{5} \cdot \frac{60 + 60 + 60}{336}$. $P(B | A) = \frac{3}{5}$
\end{solution}

\begin{result}
1) $\frac{1}{4}$, 2) $\frac{3}{5}$.
\end{result}


%---------------------------------------------------------------------------------------------------
\medskip
\begin{task}{5}
  Доказать, что события $A$, $\overline{B}$ независимы, если независимы события $A$ и $B$.
\end{task}

\begin{solution}

\noindent\textbullet~Пусть $A$, $B$ независимы, тогда $P(AB) = P(A) \cdot  P(B)$.

\smallskip
\noindent\textbullet~$P(\overline{B}) = 1 - P(B)$.

\smallskip
\noindent\textbullet~ $P(AB) = P(A) \cdot P(B) = P(A) \cdot (1 - P(\overline{B})) = P(A) - P(A) \cdot 
P(\overline{B}) = P(A) - P(A) \setminus P(B) = P(A) - P(A\overline{B})$. Последнее равенство получено
из упражнения 1.1.1.

\smallskip
\noindent\textbullet~Тогда $P(A) - P(A) \cdot P(\overline{B}) = P(A) - P(A\overline{B}) \Leftrightarrow
P(A) \cdot P(\overline{B}) = P(A\overline{B})$.
\end{solution}

%---------------------------------------------------------------------------------------------------
\medskip
\begin{task}{6}
  Случайная точка $\left(\xi_1, \xi_2\right)$ имеет равномерное распределение в квадрате $\left\{
  \left(x_1, x_2\right) : 0 \le x_1 \le 1, \; 0 \le x_2 \le 1\right\}$. При каких значениях 
  $r$ независимы события 
  \[
    A_r = \left\{\abs{\xi_1 - \xi_2} \ge r\right\}, \;\;\; B_r = \left\{\xi_1 + \xi_2 \le  3r\right\} 
  .\] 
\end{task}

\begin{solution} 
  Найдем <<объем>> вероятности, для этого построим функции вероятности на основании имеющихся данных:
\[
P_r(A) = F_A(x) =
\begin{cases}
  1, & r \le 0 \\
  (1 - r)^2, & 0 < r \le 1 \\
  0, & r > 1
\end{cases}, 
\;\;\;P_r(B) = F_B(x) = 
\begin{cases}
  0, & r \le 0 \\
  \frac{1}{2}(3r)^2, & 0 < r \le  \frac{1}{3} \\
  1 - \frac{1}{2}(2 - 3r)^2, & \frac{1}{3} < r \le \frac{2}{3} \\
  1, & r > \frac{2}{3}
\end{cases}
.\] 

\noindentТак как события называются независимыми тогда, когда выполняются какие-либо из равенств 
$P(A | B) = P(A)$, $P(B | A) = P(B)$, $P(AB) = P(A) \cdot P(B)$, то воспользовавшись ими увидим, 
что события независимы, когда $r \ge  \frac{2}{3}$, так как тогда $P(AB) = 1 \cdot P(A) = P(A | B)$,
когда $r \le 0$, так как $P(AB) = P(B) \cdot 1 = P(B | A)$.

\medskip
\noindentТеперь найдем при каких $r$ графически $P(A) * P(B) = P(AB)$:

\smallskip
\noindentПри $0 < r \le \frac{1}{3}$: $P_r(A) \cdot  P_r(B) = (1 - r)^2 \cdot \frac{1}{2} \cdot (3r)^2$.

\begin{center}
\scalebox{1.5}{
\begin{tikzpicture}
\begin{axis}[
    xmin=-1,xmax=1,
    ymin=-1,ymax=1,
    grid=both,
    grid style={line width=.1pt, draw=gray!10},
    major grid style={line width=.2pt,draw=gray!50},
    axis lines=middle,
    minor tick num=4,
    ticklabel style={font=\tiny,fill=white},
    xlabel style={at={(ticklabel* cs:1)},anchor=north west},
    ylabel style={at={(ticklabel* cs:1)},anchor=south west},
    axis equal,
    xlabel={$x$},
    ylabel={$y$}
]

\coordinate (O) at (0,0);
\node[fill=white,circle,inner sep=0pt] (O-label) at ($(O)+(-135:10pt)$) {$O$};


\draw (axis cs: -2, -5/3) -- (axis cs: 2/3, 1);
\draw (axis cs: -2/3, -1) -- (axis cs: 2, 5/3);
\draw (axis cs: 0, 1) -- (axis cs: 3/2, -1/2);

\node[fill=white,circle,inner sep=0pt] (r-label) at (3/4, -3/4) {$r = \frac{1}{3}$};

\end{axis}
\end{tikzpicture}
}
\end{center}

Графически объем равен: $2\cdot r^2$. Приравняем то, с тем, что написано выше: $(1 - r)^2 \cdot \frac{1}{2} \cdot (3r)^2 = 2\cdot r^2$.
Решая уравнение, найдем, что $r = \frac{1}{3}$.

\smallskip
\noindentДля остальных интервалов равенства не возникает. 
\end{solution}

\begin{result}
  $r \le 0, r \ge \frac{2}{3}, r = \frac{1}{3}$
\end{result}

%---------------------------------------------------------------------------------------------------
\medskip
\begin{task}{8}
  События $A_1, A_2,A_3,A_4$ взаимно независимы. Доказать взаимную независимость событий $\overline{A_1}A_2$ 
  и $A_3 A_4$.
\end{task}

\begin{solution}
Так как все события взаимно независимы, то независимы и события $A_3, A_4$, также $A_1,A_2$ тоже независимы,
а тогда и $\overline{A_1}A_2$, поскольку это было доказано ранее в задании 5.

\medskip
\noindent\textbullet~$P(\overline{A_1}A_2) = P(\overline{A_1}) \cdot P(A_2)$, $P(A_3A_4) = P(A_3) \cdot P(A_4)$

\medskip
\noindent\textbullet~$P(\overline{A_1}A_2)\cdot P(A_3A_4) = P(\overline{A_1}) \cdot P(A_2) \cdot  P(A_3) \cdot P(A_4)$

\medskip
\noindent\textbullet~ Так как независимы $A_1, A_2, A_3, A_4$, то также будут независимы и $\overline{A_1},
A_2, A_3, A_4$ (можно доказать по индукции), а значит $P(\overline{A_1}) \cdot P(A_2) \cdot  P(A_3) \cdot P(A_4) = P(\overline{A_1}A_2A_3A_4)$, 
то есть $P(\overline{A_1}A_2)\cdot P(A_3A_4) = P(\overline{A_1}A_2A_3A_4)$, ч.т.д.
\end{solution}

%---------------------------------------------------------------------------------------------------
\medskip
\begin{task}{9}
  События $A_1,A_2,A_3,A_4$ взаимно независимы: $P(A_k) = p_k$, $k = 1, 2, 3, 4$. Найти вероятность
  событий:
\end{task}

\begin{subtask}{1}
  $A_1\overline{A_3}A_4$. $P(A_1\overline{A_3}A_4) = P(A_1)\cdot P(A_3)P(A_4) = p_1 (1 - p_3)p_4$.
\end{subtask}

\begin{subtask}{2}
  $A_1 + A_2$.  $P(A_1 + A_2) =  P(A_1) + P(A_2) - P(A_1)\cdot P(A_2) = p_1 + p_2 - p_1 \cdot p_2$.
\end{subtask}

\begin{subtask}{3}
  $(A_1 + A_2) (A_3 + A_4)$. $P((A_1 + A_2) (A_3 + A_4)) = 
  P(A_1 + A_2) \cdot P(A_3 + A_4) = (P(A_1) + P(A_2) - P(A_1)\cdot P(A_2)) (P(A_3) + P(A_4) - P(A_3) \cdot P(A_4)) =
  (p_1 + p_2 - p_1 \cdot p_2) (p_3 + p_4 - p_3 \cdot p_4) $, так как события независимы.
\end{subtask}

\begin{result}  
  1) $p_1 (1 - p_3)p_4$, 2) $p_1 + p_2 - p_1 \cdot p_2$, 3) $(p_1 + p_2 - p_1 \cdot p_2) (p_3 + p_4 - p_3 \cdot p_4)$
\end{result}

%---------------------------------------------------------------------------------------------------
\medskip
\begin{task}{11}
  Из урны, содержащей 3 белых и 5 черных шаров, два игрока по очереди вытащили по одному шару. Положим
  $A_k = \{$ $k$-ый игрок вытащил белый шар $\}$. Найти вероятность событий:
\end{task}

\begin{subtask}{1}
  Вероятность $A_1$. $P(A_1) = a_1 = \frac{3}{8}$.
\end{subtask}

\begin{subtask}{2}
Вероятность $A_2$. $P(A_2) = P(A_2 | A_1) + P(A_2 | \overline{A_1}) = a_{11} \cdot  a_1  + a_{12} \cdot (1 - a_1) =
\frac{2}{7} \cdot \frac{3}{8} + \frac{3}{7} \cdot (1 - \frac{3}{8}) = \frac{6}{56} + \frac{15}{56} = \frac{21}{56} = \frac{3}{8}$.
\end{subtask}

\begin{subtask}{3}
Вероятность $ A_1 A_2$. $P(A_1A_2) = a_1 + a_{11} = \frac{3}{8} \cdot \frac{2}{7} = \frac{3}{28}$.
\end{subtask}

\begin{result}
  1) $\frac{3}{8}$, 2) $\frac{3}{8}$, 3) $\frac{3}{28}$.
\end{result}

%---------------------------------------------------------------------------------------------------
\medskip
\begin{task}{15}
  Предположим, что 5\% всех мужчин и 0.25\% всех женщин -- дальтоники. Наугад выбранное лицо оказалось
  дальтоником. Какова вероятность того, что это мужчина? (Считать, что количество мужчин и женщин одинаково.)
\end{task}

\begin{solution}
  Расчитаем по формуле Байеса: $P(B_m | A) = \frac{P(AB_m)}{P(A)} = \frac{P(AB_m)}{P(B_w) \cdot P(A | B_w) + P(B_m) \cdot P(A | B_m)} =
  \frac{0.05 \cdot \frac{1}{2}}{0.05 \cdot \frac{1}{2} + 0.0025 \cdot \frac{1}{2}} = \frac{0.025}{0.02625} = \frac{20}{21}$.
\end{solution}

\begin{result}
  $\frac{20}{21}$.
\end{result}

%---------------------------------------------------------------------------------------------------


