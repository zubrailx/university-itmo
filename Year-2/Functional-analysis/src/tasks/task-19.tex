\subsection*{Компактные множества.}

\noindent\textbullet~Пусть $X$ - НП, $\norm{x}$ - норма точки, $V_r(a) = \{ \norm{x - a} < r \}$ - открытый шар, $\tau = \{ G = \bigcup\limits_\alpha V_\alpha\}$
- топология, где $\forall G \in \tau$ - открытое множество. Нормированное пространство рассматриваем как частный случай топологического пространства, $V_r(a) \in \tau$.
Семейство замкнутых множеств $F = \overline{G}$, где $G \in \tau$, это класс двойственен открытому, определены операции, $E \subset X$, $Cl E$, $Int E$, $Fr E$.

\smallskip
\noindent\textbullet~В этом пункте выделим один из самых важный классов множеств в НП, которые называются \textit{компактные множества}. На базе этих множеств строится
теория вполне непрерывных или компактных операторов.

\medskip
\noindent\textasteriskcentered~Пусть $K \subset X$, $\{ G_\alpha\}$ - семейство открытых множеств, $K \subset \bigcup_\alpha G_\alpha$. Тогда это семейство 
открытых множеств называется \textit{открытым покрытием} множества $K$.

\smallskip 
\noindent\textasteriskcentered~Множество $K$ называется \textit{компактом} или \textit{компактным множеством}, если из любого его открытого покрытия можно выделить 
конечное подпрокрытие. То есть $K \subset \bigcup_\alpha G_\alpha \Rightarrow \exists \alpha_1, \alpha_2, \dots, \alpha_p : K \subset \bigcup_{j = 1}^p G_{\alpha_j}$.

\medskip 
\noindent\textasteriskcentered~Важным свойством НП является то, что с точки зрения топологии такое пространство \textit{хаудсорфово}. Под этим понимается то, что для $\forall x
\neq y; x, y \in X \; \exists O(x) \cap O(y) = \emptyset$ - для любых неравных точек существуют непересекающиеся окрестности. Если пространство обладает таким свойством,
то оно называется \textit{хаудсорфовым}.

\smallskip 
\noindent\textbullet~НП всегда хаудсорфово, так как если $x \neq y \Rightarrow d = \norm{x - y} \neq 0 \Rightarrow V_{\frac{d}{3}}(x) \cap V_{\frac{d}{3}}(y) = \emptyset$.
Если бы $\exists z \in V_{\frac{d}{3}}(x) \cap V_{\frac{d}{3}}(y) \Rightarrow d = \norm{x - y} = \norm{(x - z) + (z - y)} \le \norm{x - z} + \norm{z - y}$. Так как 
точка $z$ входит в оба шара, то получилось бы $\norm{x - z} + \norm{z - y} \le \frac{d}{3} + \frac{d}{3}$, а тогда бы было $d < \frac{2d}{3}$, что невозможно.

\medskip
\noindent\textbullet~$E \subset X, E$ - \textit{ограничено}, если $\forall e \in E \Rightarrow \norm{e} \le M$ - const.

\bigskip 
\noindent\textbf{Утверждение.}\textit{ $K$ - компакт в $X \Rightarrow K$ ограничено и замкнуто.}
\par\begin{proof}
\smallskip
\par\noindent\textbullet~Сначала докажем ограниченность. $\forall x \in X$ рассмотрим открытые шары $V_1(x), r = 1$. Ясно, что $K \subset \bigcup_{x \in X} V_1(x)$. Тогда по 
компактности $K \subset \bigcup_{j = 1}^p V_1(x_j)$. Возьмем $\forall y \in K$, для него $\exists j : y \in V_1(x_j) \Rightarrow \norm{y - x_j} < 1$. Тогда если $\norm{y} =
\norm{(y - x_j) + x_j} \le \norm{y - x_j} + \norm{x_j} < 1 + \norm{x_j}$. Точек $x_j$ конечное число: $x_1, \dots, x_p$, а значит $\norm{x_j} \le d = \max \{ \norm{x_1}, 
\dots, \norm{x_p}\}$. Тогда из неравенства получится $\norm{y} \le 1 + d, \; \forall y \in K \Rightarrow$ $K$ - ограниченное множество.

\medskip
\noindent\textbullet~Теперь проверим замкнутость. Для этого достаточно доказать, что $\overline{K}$ - открыто. Открытость в НП означает, что вместе с любой своей точкой 
содержится и некоторый открый шар с центром в этой точке. Берем $x \in \overline{K}$, $x \notin K$, поэтому $\forall y \in K \Rightarrow x \neq y$, а тогда по 
хаусдорфовости $X$ подбираем пару открытых множеств $G_y(x), G_y : x \in G_y(x), y \in G_y, G_y(x) \cap G_y = \emptyset$. Так как $y$ любое, то $K \subset 
\bigcap_{y \in K}G_y$, а это открытое покрытие, тогда по компактности $K$ $\exists y_1, \dots, y_p : K \in \bigcup_{j = 1}^p G_{y_j}$.  

\smallskip
\noindent\textbullet~Положим $G(x) = \bigcap_{j = 1}^p G_{y_j}(x)$, по определению оно открытое множество. Допустим $\exists z : z \in \bigcup_{j = 1}^p G_{y_j}, \;z = \bigcap_{j = 1}^p G_{y_j}(x)$. Значит для некоторого $j_0$ $z \in G_{y_{j_0}}$ и автоматически $z \in 
G_{y_{j_0}}(x)$, потому что она взята из пересечения. А тогда получится $G_{y_{j_0}} \cap G_{y_{j_0}}(x) \neq \emptyset$, что противоречит тому, что любая такая пара
множеств не пересекается, напр. $(G_y(x) \cap G_y = \emptyset)$. Таким образом множество $\bigcup_{j = 1}^p G_{y_j} \cap G(x) = \emptyset$, а значит автоматически $K \cap G(x) 
= \emptyset$, а тогда $G(x) \in \overline{K}$. 

\smallskip
\noindent\textbullet~Итак, мы взяли любую точку $x$ из дополнительного множества, построили открытое множество $G(x) \ni x$ и при этом $G(x) \subset \overline{K}$. 
Значит $\overline{K}$ - открыто, а значит $K$ - замкнуто.
\end{proof}

\bigskip 
\noindent\textbullet~В НП любой компакт замкнут и ограничен, в общем случае обратное неверно. Например, возьмем пространство $l_2$, $K = \{ \overline{e_1}, \overline{e_2},
\dots\}$, $\overline{e_n} = (0, \dots, 0, 1, 0, \dots)$. Норма каждой точки $\norm{\overline{e_n}}_{l_2} = 1 \Rightarrow K$ - ограничено; $K$ - замкнуто, потому что 
это дискретная последовательность. С другом стороны, если замерить расстояние между разными точками этого множества, то $\norm{\overline{e_n} - \overline{e_m}} = \sqrt{2}$
. Если теперь рассмотреть шары $\bigcup_{n = 1}^p V_{\frac{\sqrt{2}}{10}}(\overline{e_n}) \supset K$, однако никакого конечного подпокрытия здесь не выбрать, потому что 
ни одна из точек множества, кроме самого центра этого шара в эти шары не входит (бесконечно много непересекающихся шаров), поэтому это множество хоть и ограничено и 
замкнуто, но оно не является компактом.

\smallskip
\noindent\textbullet~Если рассматривать конечномерные НП, то в них компакность равносильна ограниченности и замкнутости.


\subsection*{Относительно компактные множества. Секвенциально компактные множества.}

\noindent\textasteriskcentered~Помимо компактных множеств удобно говорить о так называемых \textit{относительно компактных} множествах. $E$ - относительный компакт, если 
$Cl E$ - компакт. То есть в относительно компактных множествах не надо проверять замкнутость.

\bigskip 
\noindent\textasteriskcentered~Также существуют \textit{секвенциально компактные множества}. Замкнутое $K$ называется \textit{секвенциально компактным}, если из любой 
последовательности точек $\forall \{ x_n \} \subset K$ можно выделить сходящуюся подпоследовательность. 

\begin{theorem*}
Компактность и секвенциальная компактность в НП тождественны: $X$ - НП, $K$ - замкнутое множество в $X$. Тогда $K$ - компакт $\Longleftrightarrow K$ - секвенциальный 
компакт.
\end{theorem*}


\subsection*{$\epsilon$-сети. Вполне ограниченные множества.}

\noindent\textasteriskcentered~Для содержательного описания компактности в НП фундаментальную роль играют так называемые \textit{$\epsilon$-сети}
и \textit{вполне ограниченные множества}. Пусть имеется $A, B \subset X$, $\epsilon > 0, \forall a \in A \; \exists b \in B : \norm{a - b} \le \epsilon$. В этом случае 
множество $B$ называется \textit{$\epsilon$-сетью} для $A$. Если при этом множество $B$ состоит из конечного числа точек, то тогда оно называется \textit{конечной $\epsilon$-сетью} для $A$. 

\smallskip
\noindent\textasteriskcentered~$E$ - вполне ограничено, если для $\forall \epsilon$ у него $\exists$ конечная $\epsilon$-сеть. Если $E$ вполне ограничено, то $E$ 
ограничено, однако обратное в бесконечномерных пространствах в общем случае не верно. Пример, $\{ \overline{e_1}, \overline{e_2}, \dots\}$ - ограничено в $l_2$, если 
взять $\epsilon = \frac{\sqrt{2}}{10}$, то для него не построить конечной $\epsilon$-сети.


\subsection*{Теорема Хаусдорфа.}
\noindent\textbullet~Основное значение при исследовании множества на компактность имеет следующая классическая теорема Хаусдорфа.

\begin{theorem*}[Хаусдорф]
Пусть $X$ - В-пространство, $K$ - замкнутое в $X$ множество. Тогда $K$ - компакт $\Longleftrightarrow K$ - вполне ограничена. 
\end{theorem*}

\begin{proof}
\smallskip
\par\noindent\textbullet~Пусть $K$ - компакт, допустим $K$ не вполне ограничено. Тогда $\exists \epsilon_0 > 0$, для которой не будет существовать конечной 
$\epsilon$-сети. Тогда возьмем $\forall x_1 \in K$ и в силу отсутствия конечной $\epsilon_0$-сети обязательно $\exists x_2 \in K : \norm{x_1 - x_2} \ge \epsilon_0$. 
Если бы такой точки не было, то тогда бы множество, состоящее из одной точки $x_1$ было бы конечной $\epsilon_0$-сетью. Имея теперь $\{ x_1, x_2\}$ в силу отсутсвия 
конечной $\epsilon$-сети $\exists x_3 \in K : \norm{x_j - x_3} \ge \epsilon_0, \; j = 1, 2$. Если бы не было, то $\{ x_1, x_2\}$ было бы конечной $\epsilon_0$-сетью для $K$ 
и так далее по индукции. В результате выстраивается последовательность точек $\{x_1, x_2, \dots \; x_j \in K\} : \norm{x_n - x_m} \ge \epsilon_0 \; \forall n \neq m$. А из 
такой последовательности точек очевидно не выделить сходящуются подпоследовательность, значит множество $K$ не секвенциально компактно, а значит и не компактно. 
Противоречие.

\medskip
\noindent\textbullet~Теперь в другую сторону. Пусть $K$ - замкнуто и вполне ограничено. Проверим, что $K$ - секвенциальный компакт, а тогда $K$ - компакт. Для этого возьмем 
любую последовательность $X_n \in K$, тогда необходимо показать, что $\exists n_1 < n_2 < \dots : \{ x_{n_k}\}$ - сходится.

\smallskip
\noindent\textbullet~Возьмем последовательность $\epsilon_m = \frac{1}{m}$. Для $\epsilon_1 \; \exists \epsilon_1$-сеть, например, $a_1, \dots, a_p$. Значит само множество 
$K$ можно покрыть объединением шаров $K \subset \bigcup_{j = 1}^p \overline{V}_{\epsilon_1}(a_j)$. Так как их конечное число, то в одном из этих шаров окажется бесконечно 
много элементов последовательности $\{ x_n \}$. Обозначим этот шар $\overline{V}_{\epsilon_1}$. Возьмем теперь $\epsilon_2$ и ему будет существовать конечная $\epsilon_2$-сеть $b_1, b_2, \dots$. И тогда $K$ точно также будет конечно покрыт шарами радиуса $\epsilon_2$ с центром в точке $b_j$. А тогда по той же причине, что и 
выше, найдется шар, в котором будет бесконечно много элементов той части нашей последовательности, которые уже лежат в шаре $\overline{V}_{\epsilon_1}$, обозначим этот 
шар $\overline{V}_{\epsilon_2}$. И так далее до бесконечности. Так как шар $\overline{V}_{\epsilon_1}$ содержит бесконечно много $x_n$ обозначим один из них $x_{n_1}$. 
Так как шар $\overline{V}_{\epsilon_2}$ содержит бесконечно много $x_n$ из шара $\overline{V}_{\epsilon_1}$, то там найдется $x_n$ с номером, большим ${n_1}$ и 
обозначим его $x_{n_2} (n_1 < n_2)$. И так далее. В результате выстроится система номеров $(n_1 < n_2 < \dots): x_{n_k} \in \overline{V}_{\epsilon_j}$, в котором $j \le k$
.

\smallskip
\noindent\textbullet~Рассмотрим теперь точки $x_{n_k}, x_{n_{k + p}}$. $x_{n_k}, x_{n_{k + p}} \in \overline{V}_{\epsilon_k} \Rightarrow \norm{x_{n_{k + p}} - x_{n_k}} \le
2 \epsilon_k$ - расстояние не превосходит двух радиусов шара. $\epsilon_k \to 0 \Rightarrow x_{n_{k + p}} - x_{n_k} \to 0$, $k, p \to \infty$. Поскольку $X$ - В-пространство, то $\exists x = \lim x_{n_k}$. То есть из произвольной последовательности точек мы выделили сходящуюся подпоследовательность, значит $K$ - секвенциальный 
компакт, а значит и компакт.
\end{proof}