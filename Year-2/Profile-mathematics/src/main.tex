\documentclass{article}

%----geometry--------------------------------------------------------------------------------------
\usepackage{geometry}

% total - determines printable width, height
\geometry{ 
	a4paper, total={180mm,267mm}
}

%----text,fonts------------------------------------------------------------------------------------
\usepackage[T2A]{fontenc}
\usepackage[russian,english]{babel}
\usepackage[utf8]{inputenc}
\usepackage{setspace}

% \setstretch{0,9}

%----ToC-------------------------------------------------------------------------------------------
\usepackage{blindtext}
\usepackage{hyperref}

% table of contents
\addto\captionsenglish{
	\renewcommand{\contentsname}
	{Оглавление}
}

\hypersetup{
	allcolors=black
}

%----math,graphics---------------------------------------------------------------------------------
\usepackage{amsmath,amsfonts,amssymb}
\usepackage{amsthm}

\usepackage{graphicx}
\graphicspath{{images/}}

\usepackage{wrapfig}
\usepackage{tabularx}

% environments
\renewenvironment{proof}{\medskip\noindent\textbf{Доказательство}.}{\hfill$\blacksquare$}

% roman caps symbol
\newcommand{\romannumeralcaps}[1]
    {\MakeUppercase{\romannumeral #1}}
% norm symbol%
\newcommand{\norm}[1]{\left\lVert#1\right\rVert}
% module
\newcommand{\abs}[1]{\left\lvert#1\right\rvert}

%----title-----------------------------------------------------------------------------------------
\title{Профильная математика\\ Билеты к экзамену 3 семестр}
\date{2021-2022}
\author{}

%----document--------------------------------------------------------------------------------------
\begin{document}
\maketitle

\begin{sloppypar}
\tableofcontents

\newpage
\section{Матрицы и их основные свойства.}
\subsection*{Определение линейного пространства.}

\noindent \textasteriskcentered~Пусть $E$ - абстрактное \textit{линейное пространство} на поле вещественных или комплексных чисел. Это означает, что в $E$ определены 2 операции:

\romannumeralcaps{1}. Каждым двум элементам $x, y \in E$ поставлен в соответствие определенный элемент $x + y \in E$, называемый их \textit{суммой}.

\romannumeralcaps{2}. Каждому элементу $x \in E$ и каждому числу (скаляру) $\lambda$ поставлен в соответствие определенный элемент $\lambda x \in E$ - \textit{произведение} элемента на скаляр $\lambda$ - так что выполнены следующие свойства (аксиомы) для любых элементов $x, y, z \in E$ и любых скаляров $\lambda \mu$:

1) $x + y = y + x$;

2) $x + (y + z) = (x + y) + z$;

3) существует элемент $0 \in E$ такой, что $x + 0 = x$;

4) существование обратного элемента: $x + y = 0,~y$ - обратный элемент.

5) $\lambda(\mu x) = (\lambda \mu) x$;

6) $1 \cdot x = x, \; 0 \cdot x = 0$ (слева 0 - скаляр, а справа элемент множества $E$);

7) $\lambda (x + y) = \lambda x + \lambda y$;

8) $(\lambda + \mu) x = \lambda x + \mu x$;

\noindent В качестве числовых множителей (скаляров) $\lambda$, $\mu$, \dots в линейном пространстве берутся вещественные или комплексные числа. В первом случае $E$ называется \textit{вещественным} линейным многообразием, во втором - \textit{комплексным} линейным многообразием.


\subsection*{Линейные многообразия.}

\noindent \textasteriskcentered~Множество $\widetilde{E}$ в линейном пространстве $E$ называется \textit{линейным многообразием} (линейным множеством), если для любых $x, y \in \widetilde{E}$ и любых скаляров $\lambda, \mu$ линейная комбинация $\lambda x + \mu y \in \widetilde{E}$.

\noindent \textbullet~Поскольку $\widetilde{E}$ является частью линейного пространства $E$, то из определения линейного многообразия $\widetilde{E}$ также само является пространством.


\subsection*{Определение нормированного пространства и нормы.}

\noindent \textasteriskcentered~Линейное пространство $E$ называется \textit{нормированным пространством} (НП), если каждому $x \in E$ поставлено в соответствие неотрицательное число $\norm{x} \in \mathbb{R}$ (норма $x$) так, что выполнены следующие аксиомы:

\smallskip
1) $\norm{x} \ge 0; \; \norm{x} = 0$ в том и только в том случае, когда $x = 0$ (строгая положительная определенность или условие невырожденности);

2) $\norm{\lambda x} = \abs{\lambda} \cdot \norm{x}$ (однозначность или однородность);

3) $\norm{x + y} \le \norm{x} + \norm{y}$ (неравенство треугольника);


\subsection*{Метрическое пространство.}

\noindent \textasteriskcentered~Множество $X$ называется \textit{метрическим пространством}, если каждой паре его элементов $x$ и $y$ поставлено в соответствие вещественное число $\rho(x, y) = \norm{x - y}$, называемое \textit{расстоянием} между элементами $x$ и $y$, удовлетворяющее аксиомам:

\smallskip
1) $\rho(x, y) \ge 0;\; \rho(x, y) = 0$ тогда и только тогда, когда $x = y$;

2) $\rho(x, y) = \rho(y, x)$;

3) $\rho(x, y) \le \rho(x, z) + \rho(z, y)$;


\subsection*{Определение предела по норме.}
\noindent \textasteriskcentered~Элемент $x_0 \in E$ называется \textit{пределом} последовательности $\{x_n\}$, если $\norm{x_n - x_0} \rightarrow 0$ при $n \rightarrow \infty$. Если $x_0$ есть предел $\{x_n\}$, то будем писать $x_0 = \lim_{n \to \infty} x_n$ или $x_n \to x_0$ при $n \to \infty$ и говорить, что последовательность \textit{сходится} к $x_0$.\footnote{Очевидно, все это можно переписать через расстояния.}


\subsection*{Арифметика предела.}

\hspace{\parindent}1) $\lim (x_n + y_n) = \lim x_n + \lim y_n$;

2) $\lim (\alpha_n \cdot x_n) = \lim \alpha_n \cdot \lim x_n$;


3) $\lim \norm{x_n} = \norm{\lim x_n}$;

\medskip
\begin{proofexpr*}
$\lim (\alpha_n \cdot x_n) = \lim \alpha_n \cdot \lim x_n$
\end{proofexpr*}

\begin{proof}

\noindent \textbullet~$\alpha = \lim \alpha_n, \; x = \lim x_n$.

\smallskip
\noindent \textbullet~$\norm{\alpha_n x_n - \alpha x} = \norm{(\alpha_n - \alpha) x_n +\alpha (x_n - x)} \le
\norm{(\alpha_n - \alpha) x_n} + \norm{\alpha(x_n - x)} =
\abs{\alpha_n - \alpha}\norm{x_n} + \abs{\alpha} \norm{x_n - x}$ 

\smallskip
\noindent \textbullet~$\abs{\alpha_n - \alpha} \to 0$, $\norm{x_n - x} \to 0$, $\norm{x_n}$ - ограничена. Тогда и все последнее выражение стремится к 0. Тогда и $\norm{\alpha_n x_n - \alpha x}$ стремится к 0.

\smallskip 
\noindent \textbullet~Отсюда получаем, что $\lim \alpha_n x_n = \alpha x$

\end{proof}


\section{Ранг матрицы. Свойства.}
\subsection*{Неравенства Гельдера и Минковского.}
\noindent \textbf{Неравенство Гельдера.} \textit{Если $f(t) \in L_p(a, b)$, $p > 1$ и $g(t) \in L_q(a, b)$, где $\dfrac{1}{p} + \dfrac{1}{q} = 1$, то произведение 
$f(t)$ и $g(t)$ - суммируемая на $[a, b]$ функция и } 
\[
    \int_a^b \abs{f(t) g(t)} dt \le \left(\int_a^b \abs{f(t)}^p dt\right)^{\frac{1}{p}} \left(\int_a^b \abs{g(t)}^q dt\right)^{\frac{1}{q}}
\]

\medskip 
\noindent \textbf{Неравенство Минковского.} \textit{Если $f(t), g(t) \in L_p(a, b)$, то} 
\[
    \left(\int_a^b \abs{f(t) + g(t)}^p dt\right)^{\frac{1}{p}} \le 
    \left(\int_a^b \abs{f(t)}^p dt \right)^{\frac{1}{p}} + \left( \int_a^b \abs{g(t)}^p dt\right)^{\frac{1}{p}}
\]

\subsection*{Пространство последовательностей $l_p$.}

\noindent \textbullet~$l_p, \; p \ge 1$;
$l_p = \{\overline{x} = (x_1, x_2, \dots) : \sum_{n = 1}^{\infty} \abs{x_n}^p < +\infty\}$

\begin{proofexpr*}
    $\overline{x}$, $\overline{y} \in l_p \Rightarrow \overline{x} + \overline{y} \in l_p$
\end{proofexpr*}
\begin{proof}
\par \noindent \textbullet~$\overline{x} + \overline{y} = (x_1 + y_1, x_2 + y_2, \dots)$.

\smallskip
\noindent \textbullet~$\sum_{n = 1}^{\infty} \abs{x_n + y_n}^p \le 
\sum (\abs{x_n} + \abs{y_n})^p = 
\sum_{n : \abs{x_n} \ge \abs{y_n}} + \sum_{n : \abs{x_n} < \abs{y_n}} \le 
2^p \sum_{n : \abs{x_n} \ge \abs{y_n}} \abs{x_n}^p + 2^p \sum_{n : \abs{x_n} < \abs{y_n}} \abs{y_n}^p \le
2^p \left(\sum_{n = 1}^{\infty} \abs{x_n}^p + \abs{y_n}^p \right) < +\infty$ 

\medskip
\noindent \textbullet~Последнее выражение конечно по условию: $\sum_{n = 1}^\infty \abs{x_n}^p < +\infty$

\end{proof}


\begin{proofexpr*}
$\overline{x} \in l_p \Rightarrow \alpha \overline{x} \in l_p$
\end{proofexpr*}

\begin{proof}
\par\noindent \textbullet~$\alpha \overline{x} = (\alpha x_1, \alpha x_2, \dots)$. 

\smallskip
\noindent \textbullet~$\sum_{n = 1}^{\infty} \abs{\alpha x_n}^p =
\abs{\alpha}^p \sum_{n = 1}^{\infty}\abs{x_n}^p < +\infty$
\end{proof}

\subsection*{Пространство непрерывно-дифференцируемых функций $C[a, b]$.}

\noindent \textbullet~$C^{(p)}[a, b], \; p = 0, 1, 2, 3 \dots$;
$C^{(p)}[a, b] = \{ f : [a, b] \rightarrow \mathbb{R}$, которые $p$ раз непрерывно дифф на отрезке $[a, b]$$\}$

\medskip
\noindent \textbullet~При $p = 0$ $f^{(0)} = f$, по определению.

\smallskip
\noindent \textbullet~$(f + g)^{(k)} = f^{(k)} + g^{(k)}$.

\section{Обратная матрица. Теорема существования обратной матрицы.}
\subsection*{Невырожденная матрица.}

\noindent \textasteriskcentered~Квадратная матрица называется \textit{невырожденной}, если определитель $\Delta = det A$ не равен нулю: $\Delta = det A \neq 0$. В противном случае матрица называется \textit{вырожденной}.

\subsection*{Союзная матрица.}

\noindent \textasteriskcentered~Матрицей, \textbf{\textit{союзной}} к матрице $A$, называется матрица 
\[
    A^* = 
    \begin{pmatrix}
        A_{11} & A_{21} & \dots & A_{n1} \\
        A_{12} & A_{22} & \dots & A_{n2} \\
        \vdots & \vdots &\ddots & \vdots \\
        A_{1n} & A_{2n} & \dots & A_{nn}
    \end{pmatrix}
\]
, где $A_{ij}$ - алгебраическое дополнение элемента $a_{ij}$ данной матрицы.

\subsection*{Обратная матрица.}

\noindent \textasteriskcentered~Матрица $A^{-1}$ называется \textbf{\textit{обратной}} матрице $A$, если выполнено условие $A \cdot A^{-1} = A^{-1} \cdot A = E$

\bigskip
\noindent \textbf{Теорема}. Всякая невырожденная матрица имеет обратную. ($\exists A^{-1} \Longleftrightarrow detA \neq 0$)

\begin{proof}
\par \noindent \textit{Необходимость}. Пусть $\exists A^{-1} \Rightarrow A \cdot A^{-1} = E$. Так как $det(A \cdot A^{-1}) = det A * det A^{-1} \neq 0 \; (det E = 1 => det A \neq 0)$.

\smallskip
\noindent\textit{Достаточность}. Пусть $det A \neq 0$. Рассмотрим $A \cdot (A^{*})^T$.

\[
   A \cdot A^{*} = 
   \begin{pmatrix}
        A_{11} \cdot a_{11} + A_{12} \cdot a_{12} + \dots + A_{1n} \cdot a_{1n} & 
        0 & \dots & 0 \\ 
        0 & A_{21} \cdot a_{21} + \dots + A_{2n} \cdot a_{2n} &
        \dots & 0 \\ 
        \vdots & \vdots & \ddots & \vdots \\
        0 & 0 & \dots & 
        A_{n1} \cdot a_{n1} + \dots + A_{nn} \cdot a_{nn}  
   \end{pmatrix} = 
\]

$= det A \cdot E \Rightarrow \dfrac{A \cdot (A^*)^T}{det A} = E$.

\end{proof}
\section{Решение системы линейных уравнений методом Гаусса.}
\subsection*{Понятие предела на основе окрестности точки.}

\noindent \textasteriskcentered~$O(a)$ - окрестность точки $a$. Под этим понимается: $\exists G \in \tau : a \in G \subset O(a)$.

\smallskip
\noindent \textbullet~Тогда $x = \lim x_n$ в ТП $\Longleftrightarrow \forall O(x) \; \exists M \in \mathbb{N} : \forall n \ge M \Rightarrow x_n \in O(x)$. Это и есть определение предела в абстрактном ТП.

\medskip 
\noindent \textbullet~В НП все базируется на шарах, тогда то определение, которое давалось в НП ($x = \lim x_n \Longleftrightarrow \norm{x_n - x} \to 0$) $\Longleftrightarrow (x = \lim x_n$ на языке окрестностей).

\subsubsection*{Соотношение двойственности.}

\noindent \textbullet~$\overline{\bigcup\limits_\alpha A_\alpha} = \bigcap\limits_\alpha \overline{A_\alpha}, \;\;\; \overline{\bigcap\limits_\alpha A_\alpha} = \bigcup\limits_\alpha \overline{A_\alpha}$ - соотношение двойственности.


\subsection*{Важнейшие операции.}

$\forall A \subset X$ - ТП

\noindent \textasteriskcentered~Операция замыкания (closure): $Cl A = \bigcap_{A \subset F} F$ - замкн. Это наименьшее закрытое множество, в котором содержится $A$. Оно всегда не пусто.

\noindent \textasteriskcentered~$Int A = \bigcup\limits_{G \subset A} G$ - откр. Это наибольшее открытое множество, которое содержится в $A$. Может быть пустым.

\noindent \textasteriskcentered~$Fr A = Cl A \backslash Int A$ - граница.

\begin{theorem*}
    $X$ - НП, $A \subset X$, $Cl A = \{ x : \rho (x, A) = 0\}$, где $\rho(x, A) = \inf_{a \in A} \rho(x, a)$
\end{theorem*}
\begin{proof}
Пусть $B = \{ x : \rho (x, A) = 0\}$. Докажем, что $B = Cl A$.

\smallskip
\noindent \textbullet~$b \in B, \rho (b, A) = 0$. Проверим, что если $F \supset A \Rightarrow b \in F$, а тогда как $Cl A = \bigcap F \Rightarrow b \in Cl A$.

\smallskip
\noindent \textbullet~Допустим, что это не так. Тогда $b \in \overline{F}$ - откр, т.к $F$ - замкнуто. Тогда $\exists V_r(b) \subset \overline{F}$.

\smallskip
\noindent \textbullet~Поскольку $A \subset F, A \cap \overline{F} = \emptyset$. $\rho(b, A) = 0$. Тогда $\forall n \in \mathbb{N} \; \exists a_n \in A : \rho(b, a_n) < \dfrac{1}{n}$ - по определению расстояния. При $n \to \infty$ $\dfrac{1}{n} \to 0$. Тогда $\exists n_0 : \dfrac{1}{n_0} < r$. $\rho(b, a_{n_0}) < \dfrac{1}{n_0} < r \Rightarrow a_{n_0} \in V_r(b) \Rightarrow a_{n_0} \in \overline{F}$. Противоречие. Доказали $B \subset Cl A$.

\bigskip
\noindent \textbullet~Проверим, что $Cl A \subset B$. Для этого должно быть $B$ - замкнутое множество, содержащее $A$. Тогда по определению замыкания получим $Cl A \subset B$.

\smallskip
\noindent \textbullet~$B = \{ x : \rho(x, A) = 0\}$. Если взять $x \in A \Rightarrow \rho(x, A) = 0 \Rightarrow x \in B$. Доказали $A \subset B$. Теперь проверим, что $B$ - замкнутое множество. Для этого проверим, что $\overline{B}$ - открытое множество. 

\smallskip 
\noindent \textbullet~Для доказательства последнего должно выполниться: $\forall b \in \overline{B} \; \exists V_r(b) \subset \overline{B}$.

\smallskip 
\noindent \textbullet~$b \in \overline{B} \Rightarrow \rho(b, A) > 0$. Само расстояние $= inf_{a \in A} \norm{b - a}$. Обозначим $d = \rho(b, A) > 0$. $\forall a \in A \Rightarrow \norm{b - a} \ge d > 0$. 

\smallskip 
\noindent \textbullet~Возьмем в качестве $r = \dfrac{d}{3}$ и рассмотрим шар $V_r(b)$. Для того, чтобы проверить, что он содержится во множестве $\overline{B}$ необходимо проверить следующее: $\forall c \in V_r(b) \; \rho(c, A) > 0$. Тогда эта точка содержится в $\overline{B}$.

\smallskip 
\noindent \textbullet~$d \le \rho(b, a) \le \rho(b, c) + \rho(c, a)$. Значит $\rho(c, a) \ge d - \rho(b, c)$. Точка $c$ взята внутри шара, значит: $d - \rho(b, c) > d - r = \dfrac{2}{3} d$.

\smallskip 
\noindent \textbullet~$\forall a \in A \Rightarrow \rho(c, a) \ge \dfrac{2}{3} d > 0 \Rightarrow \rho(c, A) \ge \dfrac{2}{3} d > 0$.
\end{proof}

\bigskip
\noindent \textbullet~В нормированном пространстве $Cl A = \{ x : \rho(x, A) = 0\}$. $F$ - замкнутое $\Longleftrightarrow Cl F = F$.

\smallskip 
\noindent \textbullet~$F$ - замкнуто в НП $\Longleftrightarrow [x_n \in F$, $x = \lim x_n \Rightarrow x \in F]$.


\subsection*{Классификация множеств по Бэру для операторов.}

\noindent \textasteriskcentered~Классификация множеств в ТП по Бэру.

\smallskip 
\noindent \textasteriskcentered~$E \subset X$, $Cl E = X$, $E$ - всюдо плотное в X. Пример, $\overline{\mathbb{Q}} = \mathbb{R}$. Если $E$ - счетное, то само пространство $X$ называется \textit{сепарабельным}.

\smallskip
\noindent \textasteriskcentered~$E \subset X$, $Int Cl E = \emptyset$. Тогда $E$ называется нигде не плотным в $X$.

\smallskip 
\noindent \textasteriskcentered~$X = \bigcup\limits_{n = 1}^{\infty} E_n$ - объединение счетного числа нигде не плотных множеств $\Rightarrow X $ - \romannumeralcaps{1} категория Бэра. В противном случае - \romannumeralcaps{2} категория.

\medskip 
\noindent \textbullet~На языке шаров: $E$ - нигде не плотно в $X \Longleftrightarrow$ $\forall$ шаре $\overline{V} \; \exists \overline{V_1} \subset \overline{V} : \overline{V_1} \cap E = \emptyset$. То есть какого радиуса шар ни взять замыкание данного множества данный шар целиком не содержит - то есть всегда найдутся какие-то элементы из шара, которые не входят в замыкание данного множества. Как пример, на поле действительных чисел множество натуральных чисел нигде не плотно.

\begin{theorem*}[Лемма Рисса о почти перпендикуляре]
   $X$ - НП. $Y$ - собственное подпространство $X$ (замкнутое линейное многообразие). Тогда $\forall \epsilon \in (0, 1) \; \exists z_\epsilon :$ 1) $\norm{z_\epsilon} = 1$, 2) $\rho(z_\epsilon, Y) > 1 - \epsilon$.
\end{theorem*}
\begin{proof}

\noindent \textbullet~$Y = Cl Y$, $\exists \hat{x} \notin Y$, $d = \rho(\hat{x} - Y) = \inf_{y \in Y} \norm{\hat{x} - y}$.

\smallskip 
\noindent \textbullet~Допустим, что $d = 0$. Тогда по определению $\inf$ $\forall n \in \mathbb{N} \; \exists y_n \in Y : \norm{\hat{x} - y_n} < \dfrac{1}{n} \rightarrow 0 \Rightarrow \hat{x} = \lim y_n, Y$ - замкнутое $\Rightarrow \hat{x} \in Y$. Противоречие. $\hat{x} \notin Y$. Таким образом установили, что $d > 0$.

\smallskip 
\noindent \textbullet~Возьмем $\epsilon \in (0, 1) \Rightarrow \dfrac{1}{1 - \epsilon} > 1 \Rightarrow \dfrac{1}{1 - \epsilon} \cdot d > d = \inf$.

\smallskip 
\noindent \textbullet~Тогда по определению $inf$ найдется $y_\epsilon \in Y : d \le \norm{\hat{x} - y_{\epsilon}} < \dfrac{1}{1 - \epsilon} d$.

\smallskip 
\noindent \textbullet~Положим $z_\epsilon = \dfrac{\hat{x} - y_\epsilon}{\norm{\hat{x} - y_\epsilon}}$. $\norm{z_\epsilon} = 1$.

\smallskip 
\noindent \textbullet~Возьмем $\forall y \in Y$ и оценим норму разности $\norm{z_\epsilon - y}$. $\norm{z_\epsilon - y} = \norm{\dfrac{\hat{x} - y_\epsilon}{\norm{\hat{x} - y_\epsilon}} - y} = \dfrac{\norm{\hat{x} - (y_\epsilon + \norm{\hat{x} - y_\epsilon}) y}}{\norm{\hat{x} - y_\epsilon}}$.

\smallskip
\noindent \textbullet~$(y_\epsilon + \norm{\hat{x} - y_\epsilon})\cdot y \in Y$ - линейное многообразие. Тогда числитель того, что выше $\ge d$: $\dfrac{\norm{\hat{x} - (y_\epsilon + \norm{\hat{x} - y_\epsilon}) y}}{\norm{\hat{x} - y_\epsilon}} \ge \dfrac{d}{\dfrac{1}{1- \epsilon} d} = 1 - \epsilon$.

\smallskip
\noindent \textbullet~$\forall y \in Y \Rightarrow \norm{z_\epsilon - y} \ge 1 - \epsilon \Rightarrow \rho(z_\epsilon, Y) \ge 1 - \epsilon$. Что и требовалось доказать.
\end{proof}

\section{Решение системы линейных уравнений методом Крамера.}
\noindent \textbullet~Пусть дано какое-либо линейное многообразие $X$ и на нем определена не одна норма: $\norm{\cdot}_1$, $\norm{\cdot}_2$, $\norm{\cdot}_3 \dots$.
Если взять линейное многообразие $C[a, b]$ и определить на нем нормы $\norm{\delta} = max_{[a, b]} \abs{f}$, $\norm{f}_1 = \int_a^b \abs{f(x)} dx$.

\smallskip
\noindent \textbullet~Если брать $n$-мерное пространство: $\mathbb{R}^n = \{ \overline{x} = (x_1, x_2, \dots, x_n)\}$, то на нем можно например определить нормы:
$\norm{\overline{x}}_e = \sqrt{\sum_{j = 1}^n \abs{x_j}^2}$ - классическая евклидовская норма, $\norm{\overline{x}}_1 = \sum_{j = 1}^n \abs{x_j}$, $\norm{\overline{x}}_2 =
max{\abs{x_1}, \abs{x_2}, \dots, \abs{x_n}}$.

\smallskip 
\noindent \textbullet~Тогда встает вопрос, чем отличаются топологии?


\smallskip
\noindent \textbullet~Например, рассмотрим пространство $\mathbb{R}^2$. Задана евклидовская норма $\norm{\overline{x}}_e = \sqrt{x^2_1 + x^2_2}$ 
и норма $\norm{\overline{x}}_1 = \sum_{j = 1}^n \abs{x_j}$. Определим шар: $V_r(0) = \{ x_1^2 + x_2^2 < r\}$ по первой норме и квадрат по второй.

\begin{tikzpicture}
\begin{axis}[
    xmin=-11,xmax=11,
    ymin=-11,ymax=11,
    grid=both,
    grid style={line width=.1pt, draw=gray!10},
    major grid style={line width=.2pt,draw=gray!50},
    axis lines=middle,
    minor tick num=4,
    ticklabel style={font=\tiny,fill=white},
    xlabel style={at={(ticklabel* cs:1)},anchor=north west},
    ylabel style={at={(ticklabel* cs:1)},anchor=south west},
    axis equal,
    xlabel={$x$},
    ylabel={$y$}
]

\coordinate (O) at (0,0);
\node[fill=white,circle,inner sep=0pt] (O-label) at ($(O)+(-135:10pt)$) {$O$};
\node[fill=white,circle,inner sep=0pt] (r-label) at (5,3) {$r$};

\draw (axis cs: 5, 0) -- (axis cs: 0, 5);
\draw (axis cs: 0, 5) -- (axis cs: -5, 0);
\draw (axis cs: 5, 0) -- (axis cs: 0, -5);
\draw (axis cs: -5, 0) -- (axis cs: 0, -5);

\draw (O) circle [blue, radius=5];

\draw (O) circle [blue, radius=3];
\end{axis}
\end{tikzpicture}

\noindent \textbullet~В любом круге содержится квадрат, а в любом квадрате содержится соответствующий круг, а это сразу приводит к тому, что топология, порожденная
первой нормой, просто совпадет с топологией, порожденной второй, т.е. они тождественны. $\tau_e = \tau_1$.

\smallskip
\noindent \textbullet~Пример топологий, которые не совпадают: $\int$ и $max$, описанные выше.

\smallskip
\noindent \textbullet~В пространствах одной и той же размерности топологии все одинаковы, какую бы норму не писали. Именно поэтому конечномерные пространства одной 
размерности, они будут изоморфны друг другу.

\subsection*{Эквивалентных нормы.}
\noindent \textasteriskcentered~Пусть $X$ - линейное многообразие, на котором задано 2 нормы $\norm{\cdot}_1$, $\norm{\cdot}_2$. Они называются \textit{ эквивалентными} $\norm{\cdot}_1 \sim \norm{\cdot}_2$, 
тогда и только тогда когда существует пара положительных констант ($a$, $b$ $> 0$), таких что для любого $x \in X$ будет выполняться неравенство $a \cdot \norm{\cdot}_1 \le 
\norm{\cdot}_2 \le b \cdot \norm{\cdot}_1$.
\[
    \norm{\cdot}_1 \sim \norm{\cdot}_2 \Longleftrightarrow a \cdot \norm{\cdot}_1 \le \norm{\cdot}_2 \le b \cdot \norm{\cdot}_1
\]

\medskip
\noindent Бинарное отношение эквивалентности:
1) $a \sim a$;
2) $a \sim b \Longleftrightarrow b \sim a$;
3) $a \sim b, b \sim c \Rightarrow a \sim c$.

\bigskip 
\noindent \textbf{Проверим утверждение}. $\norm{\cdot}_1 \sim \norm{\cdot}_2 \Longleftrightarrow [x = \lim x_n (\norm{\cdot}_1) \Longleftrightarrow x = \lim x_n (\norm{\cdot}_2)]$.

\begin{proof}

\noindent \textbullet~$\norm{\cdot}_1 \sim \norm{\cdot}_2$, $a \cdot \norm{x}_1 \le \norm{x}_2 \le b \cdot \norm{x}_1$ $\Longleftrightarrow 
a \cdot \norm{x_n - x}_1 \le \norm{x_n - x}_2 \le b \cdot \norm{x_n - x}_1 \Longleftrightarrow
\lim \norm{x_n - x}_1 \to 0$, $\lim \norm{x_n - x}_2 \to 0$. Обратное доказывается точно так же.

\smallskip 
\noindent \textbullet~Докажем, что $\exists \; const \; a$. Для этого пойдем от противного: пусть $\nexists a $. Тогда $\forall n \in \mathbb{N} \; \exists x_n \in X : \dfrac{1}
{n} \norm{x_n}_1 > \norm{x_n}_2$. Возьмем точку $y_n = \dfrac{x_n}{\norm{x_n}_1}$. Из полувшегося равенства $\norm{y_n}_2 < \dfrac{1}{n} \to 0 \Rightarrow y_n \to 0$ 
по $\norm{\cdot}_2$. А тогда $y_n \to 0$ и по первой норме, однако $\norm{y_n}_1 = \dfrac{\norm{x_n}_1}{\norm{x_n}_1} = 1 \not\to 0$. Получили противоречие с тем, что 
$y_n \to 0$ по первой $\norm{\cdot}_1$. Аналогично доказывается и для  $b$.
\end{proof}

\section{Решение системы линейных уравнений матричным способом.}
\subsection*{Размерность.}

\noindent \textasteriskcentered~Пусть $X$ - линейное многообразие, $e_1, \dots, e_n$ - линейно-независимые векторы в $X : V(e_1, \dots, e_n) = \{ \sum_1^n 
\alpha_k e_k\}$\footnote{Чисто алгебраическое определение.}, где $V$ - линейная оболочка. Размерность $dim X = n$. Размерность отрезка = $1$, размерность квадрата - $2$.

\medskip
\noindent \checkmark~Есть еще понятие топологической размерности, но эта тема крайне трудная. Работы принадлежат Урысону (1930-е годы).

\bigskip
\begin{theorem*}[Фердинанд Рисс]
Пусть $dim X < +\infty \Rightarrow$ все нормы в $X$ эквивалентны.
\end{theorem*}

\begin{proof}

\noindent \textbullet~По условию в $X \; \exists e_1, \dots, e_n$ - линейно-нез. $: X = V(e_1, \dots, e_n)$, $\forall x \in X \Rightarrow$ Единственно($!$) $x = \sum_1^n 
\alpha_k e_k$.

\smallskip
\noindent \textbullet~Иксу соответствуют числа $x \leftrightarrow (\alpha_1, \dots, \alpha_n)$. Пусть $\norm{x}$ - норма в $X$. Помимо нее определим $\norm{x}_0$ -
как евклидовскую ($\sqrt{\sum_{k = 1} ^n \alpha_k^2}$).
Если исходная норма будет эквивалентна той, которую мы построили, то по транзитивности этого отношения эквивалентности любые две нормы на $x$ 
окажутся эквивалентными.

\smallskip 
\noindent \textbullet~По неравенству треугольника $\norm{x} \le \sum_1^n \abs{\alpha_k} \norm{e_k}$. По неравенству Гельдера, где $p = 2$: 
$\sum_1^n \abs{\alpha_k} \norm{e_k} \le \sqrt{\sum_{k = 1}^{n} \norm{e_k}^2} \cdot \sqrt{\sum_{k = 1}^n \abs{\alpha_k}^2}$. 

\smallskip
\noindent \textbullet~$\sqrt{\sum_{k = 1}^n \abs{\alpha_k}^2} = \norm{x}_0$, $ \sqrt{\sum_{k = 1}^{n} \norm{e_k}^2}$ - некоторая константа, обозначим $b$. Таким образом,
$\norm{x} \le b \norm{x}_0$.

\medskip 
\noindent \textbullet~Проверим $a \norm{x}_0 \le \norm{x}$. Для этого рассмотрим в $\mathbb{R}^n$ $f(\alpha_1, \dots, \alpha_n) = \norm{\sum_{k = 1}^n \alpha_k e_k}$.
Проверим, что $f$ - непрерывна в $\mathbb{R}^n$.

\smallskip
\noindent \textbullet~$\abs{f(\overline{\alpha} + \Delta \overline{\alpha}) - f(\overline{\alpha})} = \abs{\norm{\sum \alpha_k e_k + \sum \Delta\alpha_k e_k} - 
\norm{\sum \alpha_k e_k}} \le \norm{\sum \Delta \alpha_k e_k} \le \sum \abs{\Delta \alpha_k} \norm{e_k} \le b \cdot \norm{\Delta \overline{\alpha}}_0$.
Получили $\abs{\Delta f(\overline{\alpha})} \le b \cdot \norm{\Delta \overline{\alpha}}_0 \Rightarrow f$ - непрерывна в $\mathbb{R}^n$.

\smallskip 
\noindent \textbullet~Рассмотрим единичную сферу в $\mathbb{R}^n$ : $S_1 = \{ \overline{\alpha} : \sum_{k = 1}^n \alpha_k^2 = 1\}$. В силу непрерывности $f$ по 
т. Вейерштрасса в матанализе об экстремальных значениях непрерывной функции $\exists \overline{\alpha}^* \in S_1 : f(\overline{\alpha}^*) = 
\min_{\overline{\alpha } \in S} f(\overline{\alpha}) = a$.

\smallskip 
\noindent \textbullet~Если допустить, что $a = 0$, то $f(\overline{\alpha}^*) = 0$, а тогда по формуле для $f$ $\norm{\sum \alpha^*_k e_k} = 0 \Rightarrow
\sum \alpha^*_{k} e_k = 0$, а по линейной независимости $e_k \Rightarrow$ все $\alpha_k^* = 0$, а тогда эта точка не будет принадлежать сфере $\overline{\alpha}^* \notin
S_1$, что противоречит тому, что мы брали точку на сфере. То есть $a > 0$.

\smallskip
\noindent \textbullet~$\forall x \in X$, $ x = \sum \alpha_k e_k$. Рассмотрим соответствующее значение функции $f$ на этих коэффициентах: $f(\overline{\alpha}) = 
\norm{\sum \alpha_k e_k}$. Пусть $\beta_k = \dfrac{\alpha_k}{\norm{\overline{\alpha}}_e}$, $\sum \beta^2_k = \sum \dfrac{\alpha^2_k}{\norm{\overline{\alpha}}^2_e} = 1$
 т.е $\overline{\beta} = (\beta_1, \dots, \beta_n) \in S_1$. Тогда $f(\overline{\beta}) \ge a$.

\smallskip 
\noindent \textbullet~Если записать тождество $f(\overline{\alpha}) = \norm{\overline{\alpha}}_e \cdot \norm{\sum \dfrac{\alpha_k}{\norm{\overline{\alpha}}_e} e_k} 
= \norm{\overline{\alpha}}_e \cdot \norm{\sum \beta_k e_k}
\ge \norm{\overline{\alpha}}_e \cdot a = \norm{x}_0 \cdot a$. $f(\overline{\alpha}) = \norm{x}$. Таким образом, получаем $\norm{x} \ge a \cdot \norm{x}_0$.
\end{proof}
\section{Теорема Кронекера-Капелли.}
\noindent \textbf{Следствие}. \textit{Пусть $X$ - нормированное пространство, $Y$ - конечномерное линейное многообразие в $X \Rightarrow
Y$ - замкнутое в $X$ множество (или подпространство $X$).}

\begin{proof}

\noindent \textbullet~По условию $Y = V(e_1, \dots, e_n)$, $e_k \in X$. По свойствам замкнутых множеств необходимо доказать,что если $y_m \in Y$ и при этом 
$y_m \to y$ в $X \Rightarrow y \in Y$. 

\smallskip 
\noindent \textbullet~Так как $y_m \to y \Rightarrow y_m - y_p \to 0$. Раз $Y$ - линейное многообразие, то $y_m - y_p \in Y$. $\forall y \in Y \Rightarrow y = \sum_{k = 1}
^n \alpha_k e_k$. Также как и в теореме Рисса вместо исходной нормы $\norm{y}$ можно рассмотреть норму $\norm{y}_0 = \sqrt{\sum \alpha_k^2}$ и $\norm{y} \sim \norm{y}_0$.
А тогда по характеристическому свойству эквивалентностей раз $y_m - y_p \to 0$ по исходной норме пространства, то тогда $y_m - y_p \to 0$ по $\norm{\cdot}_0$.

\smallskip 
\noindent \textbullet~Тогда $\norm{y_m - y_p}_0 = \sqrt{\sum_{k = 1}^{n} \abs{\alpha_k^{(m)} - \alpha_k^{(p)}}^2} \to 0$. Тогда очевидно, что $\forall k = \overline{ 1,n}$
$\; \abs{\alpha_k^{(m)} - \alpha_k^{(p)}} \le \sqrt{\sum_j \abs{\alpha_j^{(m)} - \alpha_j^{(p)}}^2}$. Таким образом $\forall k = \overline{1, n} \Rightarrow
\alpha_k^{(m)} - \alpha_k^{(p)} \to 0$. То есть числовая последовательность $\{ \alpha_k^{(m)}\}$ сходится в себе (фундаментальная последовательность) по $m$. 
А тогда по критерию Коши существования предела
числовой последовательности\footnote{Для того чтобы последовательность имела конечный предел, необходимо и достаточно, чтобы она удовлетворяла условию Коши, т.е была 
фундаментальной $\norm{a_n - a_m} < \epsilon$.} $\exists \alpha_k = \lim_{m \to \infty} \alpha_k^{(m)}$, а так как таких чисел конечное число ($k = \overline{1, n}$),
$\Rightarrow \norm{\overline{\alpha}_m - \overline{\alpha}}_e \to 0$, а значит $\hat{y} = \sum \alpha_k e_k$ и если рассмотреть $\norm{y_m - \hat{y}}_0 = \norm{\overline
{\alpha}_m - \overline{\alpha}}_0 \to 0$. 

\smallskip
\noindent \textbullet~Таким образом окажется, что $\hat{y} = \lim y_m$ по норме $\norm{\cdot}_0$, а тогда в силу эквивалентности норм $\norm{\cdot} \sim
\norm{\cdot}_0$ $y_m \to \hat{y}$ по $\norm{\cdot}$. По условию $y_m \to y$ по основной норме, а тогда в силу единственности предела $\Rightarrow y = \hat{y}$. Тогда 
получается, что $y = \sum \alpha_k e_k$, а значит $y \in Y$.
\end{proof}

% Фундаментальность, ограниченность последовательности, критерий Коши
% http://nuclphys.sinp.msu.ru/mathan/p1/m0509.html
\section{Критерий линейной зависимости строк (столбцов) матрицы.}
\subsection*{Основные определения.}
\noindent \textasteriskcentered~Пусть $A \subset X$, $a \in X$. Тогда имеет место $\rho(a , A) = \inf_{b \in A} \norm{a - b}$. Существенный интерес представляет случай, 
когда $A$ - линейное подпространство $X$: пусть $Y$ - подпространство $X$, $\forall x \in X$ $E_Y(x) = \rho(x, Y)$ - \textit{наилучшее приближение $x$ подпространством
$Y$}. Если при этом $\exists y^* \in Y : E_Y(x) = \norm{x - y^*}$, то тогда $y^*$ - элемент наилучшего приближения $x$ подпространтсовм $Y$. Сам элемент может и не существовать.

\smallskip
\noindent \textasteriskcentered~Легко проверить, что $E_Y$ - \textit{полунорма} на $X$, т.е:

1) $E_Y(x) \ge 0$;

2) $E_Y(\alpha x) = \abs{\alpha} E_Y(x)$;

3) $E_Y(x_1 + x_2) \le E_Y(x_1) + E_Y(x_2)$.

\begin{theorem*}[Борель]
   Пусть $X$ - НП, $dim Y < + \infty \Rightarrow  \forall x \in X$ в $Y$ существует элемент наилучшего приближения $y^*$. 
\end{theorem*}

\begin{proof}

\noindent \textbullet~Рассмотрим $E_Y(x) = \inf_{y \in Y} \norm{x - y}$. Необходимо доказать, что $\exists y^* \in Y : E_Y(x) = \norm{x - y^*}$.

\smallskip
\noindent \textbullet~$Y = V(e_1, \dots, e_n)$ - линейная оболочка. Тогда рассмотрим $y = \sum_{k = 1}^{n} \alpha_k e_k$, а также функцию $f(\overline{\alpha}) = 
\norm{x - \sum_{k = 1}^{n} \alpha_k e_k}$. Также как при доказательстве теоремы Рисса проверяем непрерывность $f$ в $\mathbb{R}^n$. Обозначим для удобства $d = E_Y(x)$
и проверим, что вне некоторого замкнутого шара $\overline{V}_r(\overline{O})$ значение $f(\overline{\alpha}) \ge d + 1$. Если это сделать, то тогда достаточно искать инфиум функции в пределах этого
шара: $d = inf_{\overline{V} (\overline{O})} f(\overline{\alpha})$, где сам шар по теореме Рисса ограниченное замкнутое множество в $\mathbb{R}^n$, функция $f$
непрерывна на нем, а тогда по т. Вейерштрасса $\exists \overline{\alpha}^* \in \overline{V}_r(\overline{O}) : d = f(\overline{\alpha}^*)$, а тогда в качестве элемента 
наилучшего приближения мы и возьмем эту точку $y^* = \sum_{k = 1}^{n} \alpha_k^* e_k$ и теорема будет доказана. 

\smallskip 
\noindent \textbullet~$\norm{x - \sum_{k = 1}^{n} \alpha_k e_k} \ge d + 1$, где слева формула для $f$. $\norm{x - \sum_{k = 1}^{n} \alpha_k e_k} \ge
\norm{\sum \alpha_k e_k} - \norm{x}$, а тогда достаточно проверять неравенство $\norm{\sum \alpha_k e_k} \ge \norm{x} + d + 1$.

\smallskip 
\noindent \textbullet~В силу эквивалентности норм в конечномерном пространстве для некоторой константы $a > 0$ можно написать неравенство $\norm{\sum \alpha_k e_k} \ge 
a \sqrt{\sum \alpha_k^2} = a \cdot \norm{\overline{\alpha}}_0$ по т. Рисса, а тогда достаточно проверить $\norm{\overline{\alpha}}_0 \ge \dfrac{\norm{x} + d + 1}{a}$.
Тогда если обозначить $\dfrac{\norm{x} + d + 1}{a} = r$, то шар такого радиуса есть требуемый. Таким образом теорема доказана.  
\end{proof}
\section{Теорема о базисном миноре.}
\subsection*{Определение B-пространства.}

\noindent \textasteriskcentered~Последовательность $\{x_n\}$ элементов метрического пространства $X$ называется \textit{сходящейся к себе}, если для любого числа
 $\epsilon > 0$ найдется номер $n_0(\epsilon) : \rho(x_n, x_m) < \epsilon$ при $n, m \ge n_0(\epsilon)$.

\smallskip
\noindent \checkmark~Пусть $\mathbb{R}$. Есть критерий Коши существования предела числовой последовательности, в котором говорится: 
$\exists \lim a_n \Longleftrightarrow \lim (a_n - a_m) = 0$. Важным классом нормированных пространств являются те из них, в которых абстрактный вариант критерий 
Коши реализуется.

\smallskip
\noindent \checkmark~Из того факта, что $\exists a = \lim a_n$ сразу вытекает $\lim(a_n - a_m) = 0$. Достаточно написать: $\norm{a_n - a_m} = 
\norm{(a_n - a) + (a - a_m)} \le \norm{a_n - a} + \norm{a - a_m} \to 0$. Таким образом, если $\{ a_n \}$ в $X$ сходится $\Longleftrightarrow$ она будет сходиться в себе 
($a_n - a_m \to 0)$.

\smallskip
\noindent \textbullet~Пусть $X$ - НП, в котором если $lim(a_n - a_m) = 0$, то $\exists \, lim a_n$, то $X$ называется \textit{полным нормированным пространством} 
или \textit{пространством Банаха} или \textit{\textbf{B-пространством}}. Полнота означает, что из сходимости в себе следует сходимость.


\subsection*{Пространство $l_p$.}

\noindent \textbullet~Норма на $l_p$ определяется как $\norm{\overline{a}} = \left(\sum_{k = 1}^{\infty}(a_j)^p\right)^\frac{1}{p}$.

\bigskip
\noindent \textbf{Утверждение. }\textit{Пространство $l_p$ полное.\footnote{Проверьте доказательство, могу ошибаться.}}

\begin{proof}
\par\noindent \textbullet~$\overline{a} = \{ a_1^{(n)}, a_2^{(n)}, \dots\} \in l_p$; $\norm{\overline{a}_n - \overline{a}_m} \to 0$. То есть $\left(\sum_{j = 1}^\infty 
\abs{a_j^{(n)} - a_j^{(m)}}^p\right)^\frac{1}{p} \to 0$, $n, m \to \infty$. 

\smallskip
\noindent \textbullet~$\forall k = 1, 2, \dots \abs{a_k^{(n)} - a_k^{(m)}} \le \left(\sum_{j = 1}^\infty \abs{a_j^{(n)} - a_j^{(m)}}^p\right)^{\frac{1}{p}} \to 0$. 

\smallskip 
\noindent \textbullet~Таким образом, $\abs{a_k^{(n)} - a_k^{(m)}} \to  0$ при $n, m \to \infty$, $\; \forall k = 1, 2, \dots \{ a_k^{(n)}\}$ - сходится в себе,
где $k$ - фиксирована, $n$ - переменная.

\smallskip 
\noindent \textbullet~$\exists a_k = \lim a_k^{(n)} , n \to \infty$, $\overline{a} = (a_1, a_2, \dots)$. Теперь докажем переход.

\medskip
\noindent \textbullet~$\forall \epsilon > 0$ $\{ \overline{a}_n\}$ - сходится в себе.

\smallskip
\noindent \textbullet~$\exists \mathbb{N} : \forall n, m \ge \mathbb{N} \Rightarrow \norm{a_n - a_m} \le \sum$

\smallskip
\noindent \textbullet~$\sum_{j = 1}^\infty \abs{a_j^{(n)} - a_j^{(m)}}^p \le \sum^p$ $\Rightarrow$
$\sum_{j = 1}^k \abs{a_j^{(n)} - a_j^{(m)}}^p \le \sum^p \; \forall k = 1, 2, \dots \Rightarrow$
$\sum_{j = 1}^\infty \abs{a_j^{(n)} - a_j}^p \le \sum^p$;

\smallskip 
\noindent \textbullet~Так как $l_p$ - линейное многообразие, то $\overline{a} = \overline{a}_n - (\overline{a}_n - \overline{a}) \in l_p$, так как 
$\overline{a}_n \in l_p$, $\overline{a}_n - \overline{a} \in l_p$.

\end{proof}


\subsection*{Пространство $C[a, b]$.}

\noindent \textbf{Утверждение. }\textit{Пространство $C[a, b]$ полное.}

\begin{proof}
\par\noindent \textbullet~Пусть дана последовательность $\{ x_n(t)\}$, где $x_n(t) \in C[a, b]$, $n = 1, 2, \dots$, и пусть $\rho(x_n, x_m) \to 0$ при $n, m \to \infty$.
Это означает, что для последовательности $\{ x_n (t)\}$ выполняется условие Коши равномерной сходимости на $[a, b]$ и, следовательно, существует непрерывная на $[a, b]$
функция $x_0(t)$, к которой на $[a, b]$ равномерно сходится последовательность $\{x_n(t)\}$. Таким образом, $x_0(t) \in C[a, b]$ и $\rho(x_n, x_0) = \max_t \abs{x_n(t) 
- x_0(t)} \to 0$, т.е. $C[a, b]$ - полное пространство.
\end{proof}

\section{Свойства системы линейных однородных алгебраических уравнений.}
\noindent \textasteriskcentered~Пусть $\overline{V}_n = \overline{V}_{r_n}(a_n) = \{ \norm{x - a_n} \le r_n\}$. Если последующий шар содержится в предыдущем
$\overline{V}_{n + 1} \subset \overline{V}_{n}$, то тогда данная система называется \textit{вложенной}.  

\medskip
\noindent \textbf{Утверждение. }\textit{Пусть $\overline{V}_1 = \overline{V}_{r_1}(a_1)$, $\overline{V}_2 = \overline{V}_{r_2}(a_2)$ - замкнутые шары в $X$. Тогда $
\overline{V}_1 \subset \overline{V}_2 \Longleftrightarrow \norm{a_2 - a_1} \le r_2 - r_1$}. 

\begin{theorem*}[Принцип вложенных шаров]
Пусть $X$ - В-пространство и пусть система шаров $\{ \overline{V}_n = \overline{V}_{r_n}(a_n)\}$ - вложенная. Тогда их пересечение не пустое $\bigcap\limits_1^\infty 
\overline{V}_n \neq \emptyset$. 
\end{theorem*}

\begin{proof}
\par\noindent \textbullet~По вложенности и предыдущему утверждению $\norm{a_n - a_{n + 1}} \le r_n - r_{n + 1} \Rightarrow 0 \le r_{n + 1} \le r_n$, то есть 
последовательность $\{ r_n\}$ убывает и ограничена снизу, а тогда по т. Вейерштрасса о пределе монотонной последовательности $\exists r = \lim r_n$.

\noindent \textbullet~В силу вложенности можем написать неравенства с произвольными индексами: $\norm{a_m - a_{m + p}} \le r_m - r_{m+p} \to 0 \; m, p \to \infty$, 
а тогда последовательность центров $\{ a_n\}$ сходится в себе, а тогда по полноте $X$ $\exists a = \lim a_n$.

\noindent \textbullet~Опять таки в силу вложенности $\overline{V}_{m + p} \subset \overline{V}_m \Rightarrow a_{m + p} \in \overline{V}_m$, который является замкнутым 
множеством, поэтому если $p \to \infty$, то $a_{m + p} \to a \Rightarrow a \in \overline{V}_m$, а тогда в силу произвольности $m$: $a \in \bigcap\limits_1^\infty 
\overline{V}_m$.
\end{proof}
\section{Теорема о фундаментальной системе решений системы линейных однородных алгебраических уравнений.}
\begin{theorem*}[Бэр о категориях]
Любое В-пространство является множеством \romannumeralcaps{2} категории в себе. 
\end{theorem*}
\begin{proof}
\smallskip
\par\noindent \textbullet~Для доказательства применим принцип вложенных шаров и докажем от противного. Допустим $X$ имеет \romannumeralcaps{1} категорию. Значит мы его 
можем представить как $X = \bigcup\limits_1^\infty X_n$, где $X_n$ - нигде не плотное. 

\noindent \textbullet~Возьмем $\forall \overline{V}$. Раз $X_1$ - нигде не плотное, то $\exists \overline{V_1} \subset \overline{V} : \overline{V}_1 \cap X_1 = \emptyset$.
Теперь имея шар $\overline{V}_1$, принимая во внимание то, что $X_2$ - нигде не плотное, поэтому также $\exists \overline{V}_2 \subset \overline{V}_1 : \overline{V}_2 
\cap X_2 = \emptyset$. И так далее по индукции $\overline{V}_{n + 1} \subset \overline{V}_n$. Тогда по принципу вложенных шаров: $\exists a \in \bigcap\limits_1^\infty 
\overline{V}_n$.

\noindent \textbullet~Так как $X = \bigcup\limits_1^\infty X_n$ $\exists n_0 : a \in X_{n_0}$, однако $a \in \overline{V}_{n_0}$. Получили противоречие с тем, что $X_{n_0} \cap 
\overline{V}_{n_0} = \emptyset$.
\end{proof}

\section{Свойства решений системы линейных неоднородных алгебраических уравнений.}
\noindent\textbf{Следствие.\footnote{Следствие из теоремы Бэра о категориях.}} \textit{Любое В-пространство несчетно (его точки нельзя перенумеровать).}

\begin{proof}
\smallskip
\par\noindent \textbullet~Допустим, что удалось перенумеровать точки: $X = \{ x_1, x_2, \dots \}$. Очевидно, что любая точка $x_n$ как множество в $X$ нигде не плотно.
Получим $X = \bigcup\limits_{n =1}^\infty \{x_n\}$ - \romannumeralcaps{1} категория, что противоречит т. Бэра. Значит $X$ - несчетное множество.
\end{proof}
\section{Определение линейного пространства. Свойства линейного пространства. }
\subsection*{Определение основных операций в пространстве $\mathbb{R}^2$. }

\noindent \textbullet~Рассмотрим пространства $\mathbb{R}^2$. Зададим на него некоторые обозначения:

\smallskip
1) $(\overline{x}, \overline{y}) = x_1 y_1 + x_2 y_2$ - скалярное произведение векторов.

\smallskip
2) $\norm{\overline{x}} = \sqrt{x_1^2 + x_2^2} = \sqrt{(\overline{x}, \overline{x})}$ - длина вектора.

\smallskip
3) $\cos(\widehat{\overline{x}, \overline{y}}) = \dfrac{(\overline{x}, \overline{y})}{\norm{(\overline{x}, \overline{y})}}$. - косинус угла между векторами.

\smallskip
\noindent \textasteriskcentered~Ортогональные векторы - косинус между которыми равен $90^\circ$: 
$\overline{x} \bot \overline{y} \Longleftrightarrow \widehat{\overline{x}, \overline{y}} = 0$.

\medskip
\begin{tikzpicture}[scale=0.5]

\draw (0, 0) -- (5, 0) -- (5, 3) -- node[rotate=30, above]{$\overline{x} + \overline{y}$} (0, 0);
\draw (2.5, 0) node[anchor=north] {$\overline{x}$};
\draw (5, 1.5) node[anchor=west] {$\overline{y}$};
\draw (5, 0) rectangle (5, 0.5);

\end{tikzpicture}

\medskip
\noindent \textbullet~$\norm{\overline{x}}^2 + \norm{\overline{y}}^2 = \norm{\overline{x} + \overline{y}}^2$. Проверим.

\smallskip 
\noindent \textbullet~$\norm{\overline{x} + \overline{y}}^2 = (\overline{x} + \overline{y}, \overline{x} + \overline{y})$. Из формулы для скалярного произведения 
очевидно, что $(\alpha\overline{x} + \beta \overline{y}, \overline{z}) = \alpha(\overline{x}, \overline{z}) + \beta (\overline{y}, \overline{z})$. Тогда согласно 
этому соотношению: $(\overline{x} + \overline{y}, \overline{x} + \overline{y}) = (\overline{x}, \overline{x}) + (\overline{x}, \overline{y}) +
(\overline{y}, \overline{x}) + (\overline{y}, \overline{y}) = (\overline{x}, \overline{x}) + (\overline{y}, \overline{y}) = \norm{\overline{x}}^2 +
\norm{\overline{y}}^2$.

\medskip 
\noindent \textbullet~Также рассмотрим сложение векторов по правилу параллелограмма:
$2\norm{\overline{x}}^2 + 2\norm{\overline{y}}^2 = \norm{\overline{x} + \overline{y}}^2 + \norm{\overline{x} - \overline{y}}^2$.

\medskip 
\begin{tikzpicture}[scale=0.7]
    \draw[->] (0, 0) -- node[anchor=east]{$\overline{x}$} (1, 2);
    \draw[->] (0, 0) -- node[anchor=north]{$\overline{y}$} (5, 0);
    \draw (1, 2) -- (6, 2);
    \draw (5, 0) -- (6, 2);
    \draw[->] (0, 0) -- node[anchor=north, near end, rotate=25]{$\overline{x} + \overline{y}$} (6, 2);
    \draw[->] (5, 0) -- node[anchor=north, near end, rotate=-25]{$\overline{x} - \overline{y}$} (1, 2);
\end{tikzpicture}

\medskip
\noindent \textbullet~Имея линейное многообразие $\mathbb{R}^2 = \{ \overline{x} = (x_1, x_2)\}$ сложение точек и умножение точки на число выполняется покомпонентно и 
выстраивая в этом линейном многообразии величину $(\overline{x}, \overline{y}) = x_1 x_2 + y_1 y_2$ и назвав ее скалярное произведение мы дальше чисто формально можем 
определить длину вектора $\norm{\overline{x}} = \sqrt{(\overline{x}, \overline{x})}$, понятие ортогональности
$\overline{x} \bot \overline{y} \Longleftrightarrow \widehat{\overline{x}, \overline{y}} = 0$ и используя эти вещи формально выводить такие вещи как т.~Пифагора и 
равенство параллелограмма. И причем по формуле скалярного произведения заметим, что эта величина линейна по первой переменной:
$(\alpha \overline{x} + \beta \overline{y}, \overline{z}) = \alpha (\overline{x}, \overline{z}) + \beta (\overline{y}, \overline{z})$ - билинейный функционал, а также 
выполняется соотношение $(\overline{x}, \overline{y}) = (\overline{y}, \overline{x})$ поскольку $x_1, x_2 \in \mathbb{R}$. Также при умножении вектора на себя 
$(\overline{x}, \overline{x}) \ge 0$, $= 0 \Longleftrightarrow \overline{x} = 0$.


\subsection*{Пространство $\mathbb{R}^n$. }

\noindent \textbullet~Теперь рассмотрим $n$-мерные пространства: $\mathbb{R}^n = \{ \overline{x} = (x_1, x_2, \dots, x_n), x_k \in \mathbb{R}\}$. Операция и умножение
на число переносятся точно также, скалярное умножение задается как $(\overline{x}, \overline{y}) = \sum_{k = 1}^n x_k y_k$, длина вектора $\norm{\overline{x}} = 
\sqrt{(\overline{x}, \overline{x})}$. Теорема Пифагора и равенство параллелограмма сохраняются, мы задавали их формально.


\subsection*{Бесконечномерные пространства. }

\noindent \textbullet~Также хочется рассматривать бесконечномерные векторы $\overline{x} = (x_1, x_2, \dots, x_n, \dots)$. Сложение поординатное и умножение на множитель
точно также. Если взять ту же формулу для скалярного произведения, то получается бесконечная сумма слагаемых (результат не определен). Остается только заузить линейное 
многообразие бесконечномерных векторов таким образом, чтобы в рамках бесконечномерного многообразия формула для скалярного произведения сохранилась. С этой целью 
рассмотрим пространство $l_2 = \{ \overline{x} = (x_1, x_2, \dots) : \sum_{k = 1}^\infty x_k^2 < +\infty \} $. 

\smallskip 
\noindent \textbullet~По неравенству Гельдера для $p = 2$ получается $\abs{\sum_1^\infty x_k y_k} \le \sum_1^\infty \abs{x_k} \abs{y_k} \le 
\sqrt{\sum_1^\infty x_k^2} \cdot \sqrt{\sum_1^\infty y_k^2} \Rightarrow \sum_1^\infty x_k y_k$ - конечное. Таким образом, в $l_2$ скалярное произведение определено 
по аналогии с пространством $\mathbb{R}^n$ : $(\overline{x}, \overline{y}) = \sum_1^\infty x_k y_k$, норма определеляется как $\sqrt{\sum_1^\infty x_k^2} = 
\sqrt{(\overline{x}, \overline{x})}$. $\mathbb{R}^2 \rightarrow \mathbb{R}^n \rightarrow l_2$. 


\subsection*{Бесконечномерные пространства в поле комплексных чисел. Унитарное пространство. }

\noindent \textbullet~Теперь распространим скалярное произведение на абстрактную ситуацию. Пришли к абстрактной конструкции: 1) рассматриваем в поле комплексных чисел 
$\mathbb{C} = \{ z = x + i y, x, y \in \mathbb{R}, i^2 = -1\}$. Операции сложения $z_1 + z_2$ и умножения $z_1 \cdot z_2$ определены по соответствующим формулам, 
символ $\overline{z} = x - i y$ - сопряженный элемент, модуль - $\abs{z} = \sqrt{x^2 + y^2}$ и он обладает свойством $\abs{z_1 + z_2} \le \abs{z_1} + \abs{z_2}$. Также 
непосредственно проверяется: $z \cdot \overline{z} = \abs{z}^2, \; \overline{z_1 z_2} = \overline{z_1} \cdot \overline{z_2}, \; \overline{z_1 + z_2} = 
\overline{z_1} + \overline{z_2}, \; \abs{\overline{z_1} \cdot \overline{z_2}} = \abs{\overline{z_1}} + \abs{\overline{z_2}}$.

\bigskip 
\noindent \textasteriskcentered~Пусть $X$ - линейное многообразие над $\mathbb{C}$, $x + y$, $\alpha x$, $\alpha \in \mathbb{C}$, 
а также определена величина $\varphi(x, y)$, удовлетворяющая следующим аксиомам:

\smallskip
1) $\varphi(x, x) \ge 0$, $= 0 \Longleftrightarrow x = 0$;

\smallskip
2) $\varphi(x, y) = \overline{\varphi(y, x)}$;

\smallskip
3) $\varphi(\alpha x + \beta y, z) = \alpha \varphi(x, z) + \beta \varphi(y, z)$ - линейность по первому аргументу

\smallskip 
\noindent тогда $\varphi$ - \textit{скалярное произведение} на $X$ и обозначается $\varphi(x, y) = (x, y)$.

\medskip 
Как пример, возьмем $X = l_2$ над $\mathbb{C}$, $\overline{x} \in l_2$, $\overline{x} = (x_1, x_2, \dots) : x_k \in C$, $\sum_1^\infty \abs{x_k}^2 < + \infty$ - модуль 
потому что комплексное число, а также определено скалярное произведение $(\overline{x}, \overline{y}) = \sum_{k = 1}^\infty x_k \overline{y}_k$. Все аксиомы проверяются 
непосредственно.

\smallskip
\noindent \textasteriskcentered~Тогда пара объектов $(X, (x, y))$ - \textit{унитарное пространство} модели $l_2$ над полем $\mathbb{C}$.

\medskip
\begin{theorem*}[неравенство Шварца]
Пусть $X$ - унитарное пространство. Тогда $\forall x, y \in X \Rightarrow \abs{(x, y)}^2 \le (x, x) \cdot (y, y)$.    
\end{theorem*}

\begin{proof}
\smallskip
\par\noindent \textbullet~$\forall \lambda \in \mathbb{C}$ и по первой аксоиме $(\lambda x + y, \lambda x + y) \ge 0$. Раскроем данное неравенство по аксиомам, учитывая 
следующие моменты: 1) $(\lambda x, y) = \lambda(x, y)$, 2) $(x, \lambda y) = \overline{(\lambda y, x)} = \overline{\lambda (y, x)} =
\overline{\lambda} \cdot \overline{(y, x)} = \overline{\lambda} \cdot (x, y)$, т.е множитель выносится с комплексным сопряжением со второй позиции, 3) $(x, y + z) = 
\overline{(y + z, x)} = \overline{(y, x) + (z, x)} = \overline{(y, x)} + \overline{(z, x)} = (x, y) + (x, z)$.

\smallskip
\noindent \textbullet~Таким образом, $(\lambda x + y, \lambda x + y) = (\lambda x, \lambda x + y) + (y, \lambda x + y) = (\lambda x, \lambda x) + (\lambda x, y) + 
(y, \lambda x) + (y, y) = \lambda \overline{\lambda} (x, x) + \lambda(x, y) + \overline{\lambda} (y, x) + (y, y) \ge 0 \; \forall \lambda \in \mathbb{C}$. В частности 
это неравенство выполнится, если $\lambda = - \dfrac{(y, x)}{(x, x)}$; $\overline{\lambda} = - \dfrac{\overline{(y, x)}}{(x, x)} = - \dfrac{(x, y)}{(x, x)}$. Подставим и 
посчитаем.

\[
    \left( - \dfrac{(y, x)}{(x, x)}\right) \cdot \left( - \dfrac{(x, y)}{(x, x)}\right) \cdot (x, x) - \left(\dfrac{(y, x)}{(x, x)}\right) \cdot (x, y) - 
    \left(\dfrac{(x, y)}{(x, x)}\right) \cdot (y, x) + (y, y) \ge 0
\]

\smallskip
\noindent \textbullet~Домножим на неотрицательный $(x, x)$ и сократим:

\[
    \abs{(x, y)}^2 - \abs{(x, y)}^2 - \abs{(x, y)}^2 + (x, x) \cdot (y, y) \ge 0 \Rightarrow \abs{(x,y)}^2 \le (x, x) \cdot (y, y)
\]
\end{proof}

\medskip
\noindent \textbf{Утверждение.} \textit{Функционал $\norm{x} = \sqrt{(x, x)}$ удовлетворяет всем 3-м аксиомам абстрактной нормы. }

\medskip
\begin{proof}
   
\noindent1)~$\norm{x} \ge 0$, $= 0 \Longleftrightarrow x = 0$ - очевидно;

\smallskip 
\noindent2)~$\norm{\alpha x} = \sqrt{(\alpha x, \alpha x)} = \sqrt{\abs{\alpha}^2 \cdot (x, x)} = \abs{\alpha} \cdot \sqrt{(x, x)} = \abs{\alpha} \cdot \norm{x}$;

\smallskip 
\noindent3)~По неравенству Шварца: $\norm{x + y}^2 = (x + y, x+ y) = (x, x) + (x, y) + (y, x) + (y, y) = \norm{x}^2 + (x, y) + (y, x) + \norm{y}^2 \le 
\norm{x}^2 + \abs{(x, y)} + \abs{(y, x)} + \norm{y}^2 \le \norm{x}^2 + 2 \norm{x} \norm{y} + \norm{y}^2 = (\norm{x} + \norm{y})^2$. Извлекаем корень и 
получаем неравенство: $\norm{x + y} \le \norm{x} + \norm{y}$.
\end{proof}

\medskip 
\noindent \textbullet~В любом унитарном пространстве $X$, $\norm{x} = \sqrt{(x, x)}$ определяет норму на $X$. Таким образом, унитарное пространство - частный случай 
общих нормированных пространств. Оно отличается тем, что в этом пространстве норма задается конкретной формулой через скалярное произведение. Поэтому все понятия, которые 
были раньше для нормированных пространств, они могут быть использованы и здесь.

\section{Базис линейного пространства. Теорема о разложении элемента линейного пространства по базису.}
\subsection*{Ортонормальные системы.}

\noindent \textasteriskcentered~$x \bot y$ - ортогональные, если $(x, y) = 0$.

\smallskip
\noindent \textbullet~В связи с ортогональностью, вводится понятиe ОНС (ортонормальная систем точек) - $\{ e_1, \dots, e_n, \dots \}$ в $X$ : $\norm{e_n} = 1$, $i \neq 
j \Rightarrow e_i \bot e_j$. $(e_i, e_j) = \delta_{i, j}$, где $\delta_{i, j}$ - символ Кронекера, равный 1 тогда и только тогда когда $i \neq j$. Если взять пространство $l_2$, $e_n = (0, \dots, 0, 1, 0, \dots)$, где $1$ на $n$-ом месте, то тогда ${e_n}$ - ОНС в $l_2$.

\smallskip 
\noindent \textasteriskcentered~Пусть $\{e_n\}$ - ОНС, $\forall x \in X$ можно задать скалярное произведение $(x, e_n)$, обозначаемое как \textit{коэффициент Фурье} 
$x$ по ОНС $\{e_n\}$.

\smallskip 
\noindent \textbullet~Пусть $X$ - НП, в котором заданы абстрактные ряды $\sum_{k = 1}^\infty x_k$, $S_n = \sum_{k = 1}^n x_k$ - частичная сумма.
Тогда если $\exists S = \lim S_n$ в $X$, то $S$ - сумма ряда, а сам ряд сходится.

\medskip
\noindent \textbullet~Если $X$ - B-пространство, $\sum_{k = 1}^\infty \norm{x_k} < +\infty \Rightarrow \sum_{k = 1}^\infty x_k$ сходится в $X$.

\medskip 
\noindent \textasteriskcentered~Абстрактный ряд $\sigma(x) = \sum_{k = 1}^\infty (x, e_k) e_k$ называется \textit{рядом Фурье} точки $x$.

\medskip 
\noindent \textasteriskcentered~Пусть $X_n = V(e_1, \dots, e_n) = \{\sum_{k = 1}^\infty \alpha_k e_k, \alpha_k \in \mathbb{C} \}$ - линейная оболочка. $\forall x \in X$, 
$E_{X_n}(x) = \inf_{y \in X_n}\norm{x - y}$ - \textit{наилучшее приближение}.

\begin{theorem*}[Экстремальное свойство частичных сумм ряда Фурье\footnote{Доказательство этой теоремы можно также найти в книге Люстерника-Соболева на стр.82
    (параграф 3, пункт 3.)}]
Пусть $\{ e_n \}$ - ОНС, $S_n(x) = \sum_{k = 1}^n (x, e_k) e_k$ - частичная сумма. Тогда $E_{X_n}(x) = \norm{x - S_n(x)}$ - наилучшее приближение. Другими словами,
частичная сумма ряда $S_n(x)$ будет наилучшим приближением $x$ ряда $S_n(x)$.
\end{theorem*}

\begin{proof}
\smallskip
\par\noindent \textbullet~$\forall y \in X_n$, $y = \sum_{k = 1}^n \alpha_k e_k$. Проверим, что $\norm{x - y} \to \min$ по $y \in X_n$, то этот минимум будет 
достигаться на соответствующей сумме ряда Фурье $S_n(x)$.

\smallskip
\noindent \textbullet~Распишем $\norm{x - y}^2 = (x - y, x - y) = \norm{x}^2 - (x, y) - (y, x) + \norm{y}^2$ - делали для неравенства Шварца. Так как норма неотрицательна,
то, взяв квадрат, соотношение никак не нарушится.

\smallskip
\noindent \textbullet~$\norm{y}^2 = (\sum_{k = 1}^n \alpha_k e_k, \sum_{k = 1}^n \alpha_k e_k) = \sum_{i, j = 1}^n \alpha_i \overline{\alpha_j}(e_i, e_j)$ (
$(e_i, e_j) = 0 \Longleftrightarrow (i \neq j)$, иначе 1) $= \sum_{i = 1}^{n} \alpha_i \overline{\alpha_i}$.

\smallskip
\noindent \textbullet~$(x, y) = (x, \sum_1^n \alpha_j e_j) = \sum_1^n \overline{\alpha_j} (x, e_j)$. Обозначим коэф. Фурье $(x, e_j) = \beta_j$. 
То есть $(x, y) = \sum_1^n \beta_j \overline{\alpha_j}$

\smallskip
\noindent \textbullet~$(y, x) = \sum_1^n \alpha_j (e_j, x) = \sum_1^n \overline{\beta_j} \cdot \alpha_j$.

\smallskip 
\noindent \textbullet~$\norm{x - y}^2 = \norm{x}^2 - \sum_1^n \beta_j \cdot \overline{\alpha_j} - \sum_1^n \overline{\beta_j} \cdot \alpha_j + \sum_1^n \alpha_j \cdot 
\overline{\alpha_j}$.

\smallskip 
\noindent \textbullet~Заметим, что $\abs{\alpha_j - \beta_j}^2 = (\alpha_j - \beta_j) \cdot \overline{(\alpha_j - \beta_j)} = 
(\alpha_j - \beta_j) \cdot (\overline{\alpha_j} - \overline{\beta_j}) = \alpha_j \overline{\alpha_j} - \beta_j \overline{\alpha_j} - 
\overline{\beta_j} \alpha_j + \beta_j \overline{\beta_j} = \abs{\alpha_j}^2 - \beta_j \overline{\alpha_j} - \overline{\beta_j} \alpha_j + \abs{\beta_j}^2$

\smallskip 
\noindent \textbullet~$\norm{x - y}^2 = \norm{x}^2 - \sum_1^n \abs{\beta_j}^2 + \sum_1^n \abs{\alpha_j - \beta_j}^2 \to \min$ по $\alpha_j$. Минимум достигается при 
$\alpha_j = \beta_j \Rightarrow \sum_1^n \abs{\alpha_j - \beta_j}^2 = 0$.

\smallskip
\noindent Можете убедиться, что $S_n(x) = \sum_{j = 1}^n (x, e_j) e_j$.
\end{proof}

\bigskip 
\noindent\textbf{Следствие.} \textit{$E_{X_n}(x) = \norm{x}^2 - \sum_1^n \abs{(x, e_j)}^2$.}

\begin{proof}
Очевидно из последнего выведенного соотношения в теореме выше.
\end{proof}

\bigskip
\noindent \textasteriskcentered~Так как само наилучшее приближение неотрицательно, то по правилу треугольника $\sum_1^n \abs{(x, e_j)}^2 \le \norm{x}^2$, $n \to \infty$.
Данное неравенство называется \textbf{\textit{неравенством Бесселя}}.

\smallskip
\noindent \textbullet~Для того, чтобы данное неравенство обращалось в равенство, требуется полнота унитарного пространства $X$, как НП.

\section{Размерность линейного пространства. Теоремы о связи базиса и размерности.}
\subsection*{Определение гильтертового пространства.}

\noindent \textasteriskcentered~Полное, бесконечномерное, унитарное пространство назывется \textit{пространством Гильберта} или \textit{гильбертовым пространством}.

\smallskip
\noindent \textasteriskcentered~$H$ - гильбертовое пространство, пространство в котором определено скалярное произведение $(x, y)$, которое позволяет говорить об 
ортогональности точек, 
понятие нормы $\norm{x} = \sqrt{(x, x)}$ и пространство является полным или Банаховым ($x_n - x_m \to 0 \Rightarrow \exists \lim x_n$).

\bigskip
\noindent \textasteriskcentered~Пусть $H_1$ - подпростраство в $H$. Пусть $A \subset H$, тогда $A^{\bot} = \{ x \in H : x \bot a, \forall a \in A\}$.
Получившееся множество, проходящее через нуль, называется \textit{ортогональным дополнением} для множества $A$. $A$ - может быть любым, однако $A^\bot$ всегда подпространство $H$.

\begin{tikzpicture}[scale=0.2]
    \draw[dashed] (-10, 0) -- (10, 0);
    \draw[thick] (2, 0) --node[anchor=north]{$A$} (7, 0);
    \draw[solid] (0, 5) --node[anchor=east, near end]{$A^\bot$} (0, -5);
\end{tikzpicture}


\subsection*{Основная теорема теории гильбертовых пространств.}

\begin{theorem*}[Основная теорема теории гильбертовых пространств]
   Пусть $H_1$ - подпространство $H$, тогда $\forall x \in H$ $\exists$ единственные $x_1 \in H_1$ и $x_1^\bot \in H_1^\bot : x = x_1 + x_1^\bot$, то есть можно 
   разложить с сумму взаимноортогональных точек.
\end{theorem*}

\begin{proof}
\smallskip
\par\noindent \textbullet~Докажем единственность. Пусть $x'_1 \in H_1, \; x_1^{\bot'} \in H_1^\bot : x = x_1' + x_1^{\bot'}, \; x = x_1 + x_1^\bot \Rightarrow
x_1 - x_1' = x_1^{\bot'} - x_1^\bot$. Так как $H$ - подпространство, то $x_1 - x_1' \in H_1$, $x_1^{\bot'} - x_1^{\bot} \in H_1^\bot$. Так как $H_1^\bot$ - 
ортогональное дополнение $\Rightarrow H_1^\bot \cap H_1 = {0}$ - тривиальное, а тогда из последнего равенства $x_1 - x_1' = x_1^{\bot'} - x_1^\bot = 0$ получаем, что 
эти точки равны. 

\medskip 
\noindent \textbullet~Так как $H_1$ - подпространство, то $H_1$ - выпуклое замкнутое множество, тогда по теореме из пункта 17 $\forall x \in H \; \exists$ минимизирующий
элемент $x_1 \in H_1$, значит $\forall y \in H_1 \Rightarrow \norm{x - x_1} \le \norm{x - y}$. Обозначим $x_1^\bot = x - x_1$ и проверим, что $x_1^\bot \in H_1^\bot$. 

\smallskip
\noindent \textbullet~$\forall y \in H_1$, $\forall t \in \mathbb{C}$. Тогда $x_1 + t y \in H_1$, потому что это подпространство, и в силу неравенства выше получим 
$\norm{x - x_1} \le \norm{x - (x_1 + t y)}$ - верно для всех $t$. Перейдя к обозначениям выше $\norm{x_1^\bot} \le \norm{x_1^\bot - t y}^2 = \norm{x_1^\bot}^2 - 
(x_1^\bot, ty) - (ty, x_1^\bot) + \norm{t y}^2$.

\smallskip
\noindent \textbullet~Сокращаем и получаем $\overline{t} (x_1^\bot, y) + t (y, x_1^\bot) \le \abs{t}^2 \norm{y}^2$. В частности, неравенство выполняется при 
$t = \dfrac{\overline{(y, x_1^\bot)}}{\norm{y}^2}$.

\[
    \dfrac{(y, x_1^\bot)(x_1^\bot, y)}{\norm{y}^2} + \dfrac{\overline{(y, x_1^\bot)}(y, x_1^\bot)}{\norm{y}^2} \le \dfrac{\abs{(y, x_1^\bot)}^2}{\norm{y}^4} \cdot 
    \norm{y}^2
\]

\[
    2 \abs{(y, x_1^\bot)} \le \abs{(y, x_1^\bot)}^2
\]
\noindent \textbullet~Если допустить, что $(y, x_1^\bot) \neq 0 \Rightarrow 2 \le 1$. Таким образом, $(y, x_1^\bot) = 0 \Rightarrow \forall y \in H_1 \Rightarrow 
y \bot x_1^\bot \Rightarrow x_1^\bot \in H_1^\bot$. Разложение получено.
\end{proof}

\bigskip
\noindent \textasteriskcentered~$x = x_1 + x_1^\bot$, $x_1$ - проекция $x$ на $H_1$. Верно для любых абстрактных гильбертовых пространств.

\begin{tikzpicture}[scale=0.2]
    \draw[solid] (-10, 0) --node[anchor=north, very near start]{$H_1$} (10, 0);
    \draw[solid] (0, -7) --node[anchor=east, very near start]{$H_1^\bot$} (0, 7);
    \draw[->] (0, 0) --node[anchor=north, very near end]{$x$} (8, 6);
    \draw[dotted](8, 6) -- (8, 0);
    \draw[dotted](8, 6) -- (0, 6);
    \draw[very thick, ->] (0, 0) -- node[anchor=north, very near end]{$x_1$} (8, 0);
    \draw[very thick, ->] (0, 0) -- node[anchor=east, very near end]{$x_1^\bot$} (0, 6);
\end{tikzpicture}


\subsection*{Ортогональные ряды.}

\noindent \textbullet~Пусть есть $\{ e_j \}$ - ОНС. Мы хотим понять, при каком условии $\forall x \; \norm{x}^2 = \sum \abs{(x, e_j)}^2$ - выполняется уравнение замкнутости.
Для этого рассмотрим ортогональный ряд $\sigma(x) = \sum_1^\infty (x, e_j) e_j$ и установим критерий сходимости. 

\bigskip 
\noindent\textbf{Утверждение \textnormal{(Критерий сходимости ортогонального ряда)}.} \textit{Пусть $H$ - гильбертовое пространство, $\sum_1^\infty x_j$ -
ортогональный ряд. Тогда он сходится $\Longleftrightarrow \sum_1^\infty \norm{x_j}^2 < +\infty$.} 

\begin{proof}

\smallskip
\noindent1)~Пусть ряд сходится $\Rightarrow \exists S = \lim_{n \to \infty} S_n$, $S_n = \sum_1^n x_j$. Из того, что $S_n \to S \Rightarrow \norm{S_n} \to \norm{S}$.
Сосчитаем квадрат нормы частичных сумм $\norm{S_n}^2 = \left(\sum_1^n x_j , \sum_1^n x_j\right) = \sum_{i, j = 1}^n (x_i, x_j)$, где если $i \neq j \Rightarrow = 0$.
Тогда $\norm{S_n}^2 = \sum_1^n \norm{x_j}^2 \to \norm{S}^2$. Поскольку $\sum_1^n \norm{x_j}^2$ - частичная сумма $\sum_1^\infty \norm{x_k}^2$, значит сходится. 

\medskip
\noindent \textbullet~Также в силу того, что есть соотношение $\sum_1^n \norm{x_j}^2 \to \norm{S}^2$ приходим, что $\norm{\sum_1^\infty x_j}^2 = \sum_1^\infty \norm{x_j}^2$
, что можно назвать теоремой Пифагора, для пространства гильберта.

\noindent2)~Пусть $\sum_1^\infty \norm{x_k}^2 < +\infty$. Тогда $\norm{S_{n+p} - S_n}^2 = \norm{\sum_{k = n + 1}^{n + p} x_k}^2$. Точно также как считалось выше за счет 
ортогональности $\norm{\sum_{k = n + 1}^{n + p} x_k}^2 = \sum_{k = n + 1}^{n+p} \norm{x_k}^2 \to 0$, $n, p \to \infty$ поскольку числовой ряд из квадратов норм сходится.
В результе получаем, что $\norm{S_{n +p} - S_n} \to 0$, а по полноте $H$ $\exists S = \lim S_n$. Доказано.
\end{proof}
\section{Подпространство линейного пространства. Свойства.}
\subsection*{Полная и замкнутая ортонормированная система.}

\noindent \textasteriskcentered~Пусть $\{ e_n \}$ - ОНС в $H$.

\smallskip 1)~ Если из $(x, e_n) = 0 \; \forall n \Rightarrow x = 0$. Тогда $\{ e_n\}$ - \textit{замкнутая ОНС}.

\smallskip 2)~ Если $H = Cl V(e_1, e_2, \dots)$, то $\{ e_n\}$ - \textit{полная ОНС}. 


\subsection*{Теорема Рисса-Фишера.}

\begin{theorem*}[Теорема Рисса-Фишера]
   Пусть $H$ - гильбертово пространство, $a_n \in \mathbb{C} : \sum_1^\infty \abs{a_n}^2 < + \infty$. Тогда для $\forall$ ОНС $\{ e_n \} \; \exists y \in H : (y, e_n) = a_n$ -
   коэффициенты Фурье по этой системе равны $a_n$.
\end{theorem*}

\begin{proof}
\smallskip
\par\noindent \textbullet~Рассмотрим ортогональный ряд вида $\sum_1^\infty a_n e_n$. $\norm{a_n \cdot e_n}^2 = \abs{a_n}^2$, $\sum_1^\infty \abs{a_n}^2 < + \infty$,
тогда по критерию сходимости ортогональных рядов $\Rightarrow \sum_1^\infty a_n e_n$ - сходится в $H$. Обозначим $y = \sum_1^\infty a_n e_n$ его сумму. 

\smallskip
\noindent \textbullet~Воспользуемся тем, что скалярное произведение точек - функционал непрерывный, то есть если $x_n \to x, y_n \to y \Rightarrow (x_n, y_n) \to (x, y)$.
Это получается из неравенства Шварца: $(x_n, y_n) = (x, y_n) + (x_n - x, y_n) = (x, y_n) + (x_n - x, y_n - y) + (x_n - x, y) = (x, y_n - y) + (x, y) + (x_n - x, y_n -y) + 
(x_n - x, y)$. Переносим и получаем: $\abs{(x_n, y_n) - (x, y)} \le \abs{(x, y_n - y)} + \abs{(x_n - x, y_n - y)} + \abs{(x_n - x, y)} \le $ (по неравенству Шварца) $\le
\norm{x} \norm{y_n - y} + \norm{x_n - x} \norm{y_n - y} + \norm{x_n - x} \norm{y} \to 0$. Поскольку правая часть стремится к нулю, то и левая тоже будет стремиться.

\smallskip
\noindent \textbullet~Раз скалярное произведение непрерывно, то рассмотрим коэффициенты Фурье ряда $y = \sum_1^\infty a_j e_j$. $(y, e_k) = (\sum_1^\infty a_j e_j, 
e_k) = \sum_1^\infty a_j (e_j, e_k) = a_k$, ну и понятно что $(e_j, e_k) = 1$ только в том случае, когда $k = j$. Сумму ряда мы можем выносить, поскольку 
он непрерывный и сумма ряда равна пределу частичных сумм. 
\end{proof}

\bigskip
\begin{theorem*}
Пусть $\{ e_n\}$ - замкнутая ОНС. Тогда $\forall x \in H$ разлагается в ряд Фурье по этой системе. То есть выполняется равенство: $x = \sum_{j = 1}^\infty (x, e_j)e_j$.
\end{theorem*}

\begin{proof}
\smallskip\par\noindent \textbullet~По неравенству Бесселя $\sum_{j = 1}^\infty \abs{(x, e_j)}^2 \le \norm{x}^2$, то есть ряд сходится. Тогда беря в т. Рисса-Фишера $a_j 
= (x, e_j)$ получаем, что они удовлетворяют т. Рисса-Фишера, а значит по этой теореме $\exists y \in H : (y, e_j) = a_j = (x, e_j)$. Знаем, что $y = \sum_1^\infty a_j e_j
= \sum_1^\infty (x, e_j) e_j$. В силу $(y, e_j) = (x, e_j)$ видим, что $(y - x, e_j) = 0$, а тогда по замкнутости системы $y - x = 0$ - нулевая точка, а значит $y = x$,
тогда будем в $y = \sum_1^\infty (x, e_j) e_j$ писать вместо $y$ $x$ и получим нужную формулу. 
\end{proof}

\medskip
\noindent \textbullet~Таким образом, по любой замкнутой системе мы может точку разложить в ряд Фурье по этой системе. Это позволяет установить следующее утверждение:

\bigskip
\noindent \textbf{Утверждение.} \textit{ Пусть $\{ e_j \}$ - ОНС в $H$. Тогда она замкнута $\Longleftrightarrow$ она полная (классы тождественны).}

\begin{proof}
\smallskip
\par\noindent \textbullet~Если $\{ e_j \}$ - замкнута $\Rightarrow \forall x = \sum_1^\infty (x, e_j) e_j$, а значит $S_n = \sum_1^n (x, e_j) e_j \to x \Rightarrow 
Cl V(e_1, e_2, \dots) = H$ - замыкание линейное оболочки есть все $H$. Тогда получится, что $ \{ e_j\}$ - полная система.

\medskip
\noindent \textbullet~Если система полная, то $\forall \epsilon > 0 \; \exists \sum_{k = 1}^n \alpha_k e_k : \norm{x - \sum_1^n \alpha_k e_k} \le \epsilon$, но по 
экстремальному свойству частичных сумм ряда Фурье, если $S_n = \sum_1^n (x, e_k)e_k \Rightarrow \norm{x - S_n} \le \norm{x - \sum_1^n \alpha_k e_k} \le \epsilon$. Тогда 
в силу произвольности $\epsilon$ $x$ окажется пределом этих частичных сумм или разложится в ряд Фурье: $x = \lim S_n$, $x = \sum_1^\infty (x, e_k) e_k$. Тогда если 
все $(x, e_k) = 0 \Rightarrow x = 0$, а значит $\{ e_n\}$ - замкнуто.
\end{proof}


\subsection*{Равенство Парсеваля.}

\noindent \textbullet~Таким образом, для того, чтобы $\forall x$ выполнялось равенство $\norm{x}^2 = \sum_1^\infty \abs{(x, e_k)}^2 \Longleftrightarrow$ $\{ e_n\}$
замкнута.

\smallskip
\noindent \textbullet~$\{ e_n \}$ - замкнутая ОНС. Берем пару точек $x = \sum_1^\infty (x, e_j) e_j$, $y = \sum_1^\infty (y, e_j) e_j$. Если вычислить скалярное 
произведение, то за счет непрерывности скалярного произведения можно написать двойную сумму, за счет ортогональности все обнуляется кроме одинаковых индексов
$(x, y) = \sum_{i, j = 1}^\infty (x, e_i) \overline{(y,e_j)} (e_i, e_j) = \sum_{i = 1}^\infty (x, e_i)\overline{(y, e_i)}$. Полученное равенство называется \textit{
равенством Парсеваля}.

\smallskip
\noindent \textbullet~В качестве примера, $l_2$, $\overline{e_n} = (0, \dots, 0, 1, 0, \dots)$. Теперь если взять некоторую последовательность $\overline{x} = (x_1, 
x_2, \dots)$ и начать считать ее коэффициенты Фурье по этой системе: $(\overline{x}, \overline{e}_n) = \sum_1^\infty x_k \overline{e}_k^{(n)} = x_k$. Поэтому если 
все $(\overline{x}, \overline{e_n}) = 0 \Rightarrow $ все $x_n = 0 \Rightarrow \overline{x} = \overline{0}$. Тогда данная система является замкнутой. Поэтому любая 
точка $\overline{x} = \sum_1^\infty x_k \overline{e_k}$ в $l_2$.
\section{Линейный оператор. Матрица линейного оператора.}
\subsection*{Определение выпуклого и невыпуклого множества.}

\noindent \textasteriskcentered~Множество является \textit{выпуклым}, если $x, y \in M$, $\lambda \in [0, 1] \Rightarrow \lambda x + (1 - \lambda) y \in M$. В противном 
случае, множество называется \textit{невыпуклым}. Любое подпространство обязательно является выпуклым, однако обратное неверно.

\begin{tikzpicture}
    \draw (0, 0) .. controls (0, 1) and (1, 1) .. (1, 0.5)
                 .. controls (1, 0.25) and (2, 0.25) .. (2, 0.5)
                 .. controls (2, 1) and (3, 1) .. (3, 0.5)
                 .. controls (3, -1) and (0, -1) .. (0, 0);
    \draw (1.5, 0) node{невыпуклое};

    \draw (5, 0) .. controls (5, 1) and (8, 1) .. (8, 0)
                 .. controls (8, -1) and (5, -1) .. (5, 0);

    \draw (6.5, 0) node{выпуклое};


    \draw (12, 0) node{$M$} circle (1);
    \fill (14, 0) node[anchor=north]{$x$} circle[radius = 0.05];
    \draw (14, 0) -- (12, 0);
    \fill (13, 0) node[anchor=south west]{$\hat{x}$} circle[radius = 0.05];
\end{tikzpicture}


\subsection*{Теорема о наилучшем приближении в Н для случая выпуклого, замкнутого множества.\footnote{По идее данная теорема должна была быть в пункте 15.}}

\begin{theorem*}
Пусть $H$ - гильбертовое пространство, $M$ - замкнутое выпуклое в $H$. Тогда $\forall x \in H$ в $M$ $\exists !$(единственный) элемент $\hat{x} : 
\rho(x, M) = \norm{x - \hat{x}}$.
\end{theorem*}

\begin{proof}
\smallskip
\par\noindent \textbullet~Обозначим $d = \rho(x, M) = \inf_{y \in M} \norm{x - y}$. По определению точной нижней грани $\forall n \in \mathbb{N} \; \exists y_n \in M : 
d \le \norm{x - y_n} < d + \dfrac{1}{n}$.

\smallskip 
\noindent \textbullet~Рассмотрим точки $y_n, y_m, x - y_n, x - y_m$ и применим к ним равенство параллелограмма $2 \norm{x - y_n}^2 + 2 \norm{x - y_m}^2 = \norm{2x - y_n 
-y_m}^2 + \norm{y_n - y_m}^2$ - сумма квадратов диагоналей равна сумме квадратов его сторон.

\smallskip 
\noindent \textbullet~Из полученого выше $\norm{2x - y_n - y_m}^2 + \norm{y_n - y_m}^2 < 2(d + \dfrac{1}{n})^2 + 2(d + \dfrac{1}{m})^2$.

\smallskip
\noindent \textbullet~$\norm{2x - y_n - y_m}^2 = 4 \norm{x - \dfrac{y_n + y_m}{2}}^2$. $M$ - выпуклое, $y_n, y_m \in M \Rightarrow \dfrac{y_n + y_m}{2}$ - по выпуклости 
в $M$ будет лежать и их середина, то есть $\norm{x - \dfrac{y_n + y_m}{2}} \ge d$.

\smallskip
\noindent \textbullet~Подставляя в равенство, получаем $4d^2 + \norm{y_n - y_m}^2 < 2(d + \dfrac{1}{n})^2 + 2(d + \dfrac{1}{m})^2 = 4d^2 + \dfrac{4d}{n} + \dfrac{4d}{m} + 
\dfrac{2}{n^2} + \dfrac{2}{m^2} \Longleftrightarrow \norm{y_n - y_m}^2 < 2(d + \dfrac{1}{n})^2 + 2(d + \dfrac{1}{m})^2 = \dfrac{4d}{n} + \dfrac{4d}{m} + \dfrac{2}{n^2} +
\dfrac{2}{m^2}$. Выражение в правой части $\to 0$ при $n, m \to \infty$, а тогда и $\norm{y_n - y_m} \to 0$. По полноте $H$ $y_n, y_m \to y$. $M$ - замкнутое, $y_n \in M 
\Rightarrow y \in M$.

\noindent \textbullet~$d \le \norm{x - y_n} < d + \dfrac{1}{n}$, $y_n \to y \in M$. Устремляя $n \to \infty$ получаем, что $\norm{x - y} = d = \rho(x, M)$.

\noindent \textbullet~Докажем единственность. От противного: пусть нашлось $y^* \in M : \norm{x - y^*} = d$. Пишем равенство параллелограмма $2 \norm{x - y}^2 + 
2 \norm{x - y^*}^2 = \norm{2x - y - y^*}^2 + \norm{y - y^*}^2$. $\norm{x - y} = d$, $\norm{x - y^*} = d$, $\norm{2x - y - y^*} \ge 4 d^2$. Тогда получаем 
$4d^2 + \norm{y - y^*} \le 4d^2 \Rightarrow \norm{y - y^*} = 0 \Rightarrow y^* = y$. Найденный $y$ есть $\hat{x}$, который мы искали.
\end{proof}


\section{Координаты образа линейного оператора.}
\subsection*{Определение прямой суммы.}

\subsubsection*{Для 2-х попарно ортогональных подпростанств.}

\noindent\textbullet~В теории гильбертовых пространств важное значение имеет операция \textit{прямой суммы} попарно ортогональных подпространств. Пусть $H_1 \bot H_2$ 
в $H$, то есть $\forall x \in H_1, \forall y \in H_2 \Rightarrow (x, y) = 0 \; [x \bot y = 0]$. Полагаем $H_1 \oplus H_2 = \{ x_1 + x_2, x_1 \in H_1, x_2 \in H_2 \}$ 
- линейное многообразие в $H$. Проверим, что это множество замкнутое. Для этого берем последовательность точек $x_n \in H_1 \oplus H_2$, считаем, что $\exists x = \lim 
x_n$. Необходимо проверить, что $\lim x_n \Rightarrow x \in H_1 \oplus H_2$. 

\smallskip 
\noindent\textbullet~Так как $x_n \in H_1 \oplus H_2 \Rightarrow x_n = x_1^{(n)} +x_2^{(n)}$, где $x_1^{(n)} \in H_1, x_2^{(n)} \in H_2$, $x_1^{(n)} \bot x_2^{(n)}$.

\smallskip 
\noindent\textbullet~Составляем $x_n - x_m = (x_1^{(n)} - x_1^{(m)}) + (x_2^{(n)} - x_2^{(m)})$, тогда по теореме Пифагора $\norm{x_n - x_m}^2 = \norm{x_1^{(n)} - 
x_1^{(m)}}^2 + \norm{x_2^{(n)} - x_2^{(m)}}^2$. Левая часть $\to 0$, тогда и пара слагаемых $\norm{x_1^{(n)} - x_1^{(m)}}^2, \norm{x_2^{(n)} - x_2^{(m)}}^2$ $\to 0$. 
Тогда по полноте $H$ $\exists  x_1 = \lim x_1^{(n)}$, $\exists x_2 = \lim x_2^{(n)}$, причем $H_i$ - замкнуты, поскольку подпространства, тогда $x_i \in H_i$.

\smallskip 
\noindent\textasteriskcentered~Если вернуться к $x_n = x_1^{(n)} + x_2^{(n)}$, $x_n \to x$, $x_1^{(n)} \to x_1$, $x_2^{(n)} \to x_2$. Тогда $x = x_1 + x_2$, а в силу 
$x_i \in H_i \Rightarrow$ $x \in H_1 \oplus H_2$. Таким образом, это многообразие - замкнутое множество, а значит оно подпространство. Это позволяет определить прямую
сумму взаимноортогональных подпростанств, то есть линейное многообразие $H_1 \oplus H_2 = \{x_1 + x_2, x_j \in H_j \}$, которое является подпространством и называется
\textit{прямой суммой} $H_1$ с $H_2$.

\medskip
\noindent\textbullet~Далее в терминах прямой суммы если вернуться к основной теореме теории гильбертовых пространств, то тогда ясно, что эту теорему можно записать 
формулой $H = H_1 \oplus H_1^\bot$. Таким образом, любое гильбертовое пространство может быть разложено в прямую сумму $H_1$ и $H_1^\bot$.


\subsubsection*{Для n попарно ортогональных подпростанств.}

\noindent\textasteriskcentered~Если $H_1, \dots, H_p$ - подпространства и $i \neq j \; H_i \bot H_j$, то $H_1 \oplus \dots \oplus H_p = \{ x_1 + x_2 + \dots + x_p , 
x_i \in H_i \}$. Как выше устанавливается то, что это подпространство и называется \textit{прямой суммой} $H_1, \dots, H_p$ попарно ортогональных подпростанств.


\subsubsection*{Для последовательности попарно ортогональных подпростанств.}

\noindent\textbullet~Теперь перенесем эту операцию на целую последовательность $H_1, H_2, \dots$ попарно ортогональных подпространств. Рассматривать $x_1, x_2, \dots, x_j
\in H_j$ бессмысленно, потому что это ортогональный ряд, его сходимость равносильна сходимости $\sum_1^\infty \norm{x_j}^2$, а этот ряд оказывается расходящимся, потому 
что нормы не стремятся к нулю.

\smallskip 
\noindent\textasteriskcentered~Имея последовательность ортогональных подпространств $H_1, H_2, \dots$ создаем линейное многообразие $\hat{H} : 
\{ \sum_1^n x_k, x_k \in H_k\}$. После этого переходим к замыканию $Cl \hat{H}$ - подпространство $H$. Тогда это замыкание и обозначают $H_1 \oplus H_2 \oplus \dots \oplus \dots$ и называют \textit{ прямой суммой последовательности попарно ортогональных подпространств}. Таким образом обычно в функциональном анализе переносят операцию
с конечным числом слагаемых на операции с бесконечным числом слагаемых.


\subsubsection*{Математический смысл прямой суммы. }

\noindent\textbullet~В следующей теореме приводится без доказательства математический смысл прямой суммы.

\begin{theorem*}
Пусть $H_1, H_2, \dots$ - попарно ортогональные подпространства $H$, $\hat{H} = H_1 \oplus H_2 \oplus \dots$. Тогда $\forall x \in H$ его проекция на $\hat{H}$ $\hat{x}
= x_1 + x_2 + \dots$, где $x_n$ - проекция $x$ на $H_n$.
\end{theorem*}
\section{Действия с линейным операторами. Преобразование матрицы линейного оператора при переходе к новому базису.}
\subsection*{Компактные множества.}

\noindent\textbullet~Пусть $X$ - НП, $\norm{x}$ - норма точки, $V_r(a) = \{ \norm{x - a} < r \}$ - открытый шар, $\tau = \{ G = \bigcup\limits_\alpha V_\alpha\}$
- топология, где $\forall G \in \tau$ - открытое множество. Нормированное пространство рассматриваем как частный случай топологического пространства, $V_r(a) \in \tau$.
Семейство замкнутых множеств $F = \overline{G}$, где $G \in \tau$, это класс двойственен открытому, определены операции, $E \subset X$, $Cl E$, $Int E$, $Fr E$.

\smallskip
\noindent\textbullet~В этом пункте выделим один из самых важный классов множеств в НП, которые называются \textit{компактные множества}. На базе этих множеств строится
теория вполне непрерывных или компактных операторов.

\medskip
\noindent\textasteriskcentered~Пусть $K \subset X$, $\{ G_\alpha\}$ - семейство открытых множеств, $K \subset \bigcup_\alpha G_\alpha$. Тогда это семейство 
открытых множеств называется \textit{открытым покрытием} множества $K$.

\smallskip 
\noindent\textasteriskcentered~Множество $K$ называется \textit{компактом} или \textit{компактным множеством}, если из любого его открытого покрытия можно выделить 
конечное подпрокрытие. То есть $K \subset \bigcup_\alpha G_\alpha \Rightarrow \exists \alpha_1, \alpha_2, \dots, \alpha_p : K \subset \bigcup_{j = 1}^p G_{\alpha_j}$.

\medskip 
\noindent\textasteriskcentered~Важным свойством НП является то, что с точки зрения топологии такое пространство \textit{хаудсорфово}. Под этим понимается то, что для $\forall x
\neq y; x, y \in X \; \exists O(x) \cap O(y) = \emptyset$ - для любых неравных точек существуют непересекающиеся окрестности. Если пространство обладает таким свойством,
то оно называется \textit{хаудсорфовым}.

\smallskip 
\noindent\textbullet~НП всегда хаудсорфово, так как если $x \neq y \Rightarrow d = \norm{x - y} \neq 0 \Rightarrow V_{\frac{d}{3}}(x) \cap V_{\frac{d}{3}}(y) = \emptyset$.
Если бы $\exists z \in V_{\frac{d}{3}}(x) \cap V_{\frac{d}{3}}(y) \Rightarrow d = \norm{x - y} = \norm{(x - z) + (z - y)} \le \norm{x - z} + \norm{z - y}$. Так как 
точка $z$ входит в оба шара, то получилось бы $\norm{x - z} + \norm{z - y} \le \frac{d}{3} + \frac{d}{3}$, а тогда бы было $d < \frac{2d}{3}$, что невозможно.

\medskip
\noindent\textbullet~$E \subset X, E$ - \textit{ограничено}, если $\forall e \in E \Rightarrow \norm{e} \le M$ - const.

\bigskip 
\noindent\textbf{Утверждение.}\textit{ $K$ - компакт в $X \Rightarrow K$ ограничено и замкнуто.}
\par\begin{proof}
\smallskip
\par\noindent\textbullet~Сначала докажем ограниченность. $\forall x \in X$ рассмотрим открытые шары $V_1(x), r = 1$. Ясно, что $K \subset \bigcup_{x \in X} V_1(x)$. Тогда по 
компактности $K \subset \bigcup_{j = 1}^p V_1(x_j)$. Возьмем $\forall y \in K$, для него $\exists j : y \in V_1(x_j) \Rightarrow \norm{y - x_j} < 1$. Тогда если $\norm{y} =
\norm{(y - x_j) + x_j} \le \norm{y - x_j} + \norm{x_j} < 1 + \norm{x_j}$. Точек $x_j$ конечное число: $x_1, \dots, x_p$, а значит $\norm{x_j} \le d = \max \{ \norm{x_1}, 
\dots, \norm{x_p}\}$. Тогда из неравенства получится $\norm{y} \le 1 + d, \; \forall y \in K \Rightarrow$ $K$ - ограниченное множество.

\medskip
\noindent\textbullet~Теперь проверим замкнутость. Для этого достаточно доказать, что $\overline{K}$ - открыто. Открытость в НП означает, что вместе с любой своей точкой 
содержится и некоторый открый шар с центром в этой точке. Берем $x \in \overline{K}$, $x \notin K$, поэтому $\forall y \in K \Rightarrow x \neq y$, а тогда по 
хаусдорфовости $X$ подбираем пару открытых множеств $G_y(x), G_y : x \in G_y(x), y \in G_y, G_y(x) \cap G_y = \emptyset$. Так как $y$ любое, то $K \subset 
\bigcap_{y \in K}G_y$, а это открытое покрытие, тогда по компактности $K$ $\exists y_1, \dots, y_p : K \in \bigcup_{j = 1}^p G_{y_j}$.  

\smallskip
\noindent\textbullet~Положим $G(x) = \bigcap_{j = 1}^p G_{y_j}(x)$, по определению оно открытое множество. Допустим $\exists z : z \in \bigcup_{j = 1}^p G_{y_j}, \;z = \bigcap_{j = 1}^p G_{y_j}(x)$. Значит для некоторого $j_0$ $z \in G_{y_{j_0}}$ и автоматически $z \in 
G_{y_{j_0}}(x)$, потому что она взята из пересечения. А тогда получится $G_{y_{j_0}} \cap G_{y_{j_0}}(x) \neq \emptyset$, что противоречит тому, что любая такая пара
множеств не пересекается, напр. $(G_y(x) \cap G_y = \emptyset)$. Таким образом множество $\bigcup_{j = 1}^p G_{y_j} \cap G(x) = \emptyset$, а значит автоматически $K \cap G(x) 
= \emptyset$, а тогда $G(x) \in \overline{K}$. 

\smallskip
\noindent\textbullet~Итак, мы взяли любую точку $x$ из дополнительного множества, построили открытое множество $G(x) \ni x$ и при этом $G(x) \subset \overline{K}$. 
Значит $\overline{K}$ - открыто, а значит $K$ - замкнуто.
\end{proof}

\bigskip 
\noindent\textbullet~В НП любой компакт замкнут и ограничен, в общем случае обратное неверно. Например, возьмем пространство $l_2$, $K = \{ \overline{e_1}, \overline{e_2},
\dots\}$, $\overline{e_n} = (0, \dots, 0, 1, 0, \dots)$. Норма каждой точки $\norm{\overline{e_n}}_{l_2} = 1 \Rightarrow K$ - ограничено; $K$ - замкнуто, потому что 
это дискретная последовательность. С другом стороны, если замерить расстояние между разными точками этого множества, то $\norm{\overline{e_n} - \overline{e_m}} = \sqrt{2}$
. Если теперь рассмотреть шары $\bigcup_{n = 1}^p V_{\frac{\sqrt{2}}{10}}(\overline{e_n}) \supset K$, однако никакого конечного подпокрытия здесь не выбрать, потому что 
ни одна из точек множества, кроме самого центра этого шара в эти шары не входит (бесконечно много непересекающихся шаров), поэтому это множество хоть и ограничено и 
замкнуто, но оно не является компактом.

\smallskip
\noindent\textbullet~Если рассматривать конечномерные НП, то в них компакность равносильна ограниченности и замкнутости.


\subsection*{Относительно компактные множества. Секвенциально компактные множества.}

\noindent\textasteriskcentered~Помимо компактных множеств удобно говорить о так называемых \textit{относительно компактных} множествах. $E$ - относительный компакт, если 
$Cl E$ - компакт. То есть в относительно компактных множествах не надо проверять замкнутость.

\bigskip 
\noindent\textasteriskcentered~Также существуют \textit{секвенциально компактные множества}. Замкнутое $K$ называется \textit{секвенциально компактным}, если из любой 
последовательности точек $\forall \{ x_n \} \subset K$ можно выделить сходящуюся подпоследовательность. 

\begin{theorem*}
Компактность и секвенциальная компактность в НП тождественны: $X$ - НП, $K$ - замкнутое множество в $X$. Тогда $K$ - компакт $\Longleftrightarrow K$ - секвенциальный 
компакт.
\end{theorem*}


\subsection*{$\epsilon$-сети. Вполне ограниченные множества.}

\noindent\textasteriskcentered~Для содержательного описания компактности в НП фундаментальную роль играют так называемые \textit{$\epsilon$-сети}
и \textit{вполне ограниченные множества}. Пусть имеется $A, B \subset X$, $\epsilon > 0, \forall a \in A \; \exists b \in B : \norm{a - b} \le \epsilon$. В этом случае 
множество $B$ называется \textit{$\epsilon$-сетью} для $A$. Если при этом множество $B$ состоит из конечного числа точек, то тогда оно называется \textit{конечной $\epsilon$-сетью} для $A$. 

\smallskip
\noindent\textasteriskcentered~$E$ - вполне ограничено, если для $\forall \epsilon$ у него $\exists$ конечная $\epsilon$-сеть. Если $E$ вполне ограничено, то $E$ 
ограничено, однако обратное в бесконечномерных пространствах в общем случае не верно. Пример, $\{ \overline{e_1}, \overline{e_2}, \dots\}$ - ограничено в $l_2$, если 
взять $\epsilon = \frac{\sqrt{2}}{10}$, то для него не построить конечной $\epsilon$-сети.


\subsection*{Теорема Хаусдорфа.}
\noindent\textbullet~Основное значение при исследовании множества на компактность имеет следующая классическая теорема Хаусдорфа.

\begin{theorem*}[Хаусдорф]
Пусть $X$ - В-пространство, $K$ - замкнутое в $X$ множество. Тогда $K$ - компакт $\Longleftrightarrow K$ - вполне ограничена. 
\end{theorem*}

\begin{proof}
\smallskip
\par\noindent\textbullet~Пусть $K$ - компакт, допустим $K$ не вполне ограничено. Тогда $\exists \epsilon_0 > 0$, для которой не будет существовать конечной 
$\epsilon$-сети. Тогда возьмем $\forall x_1 \in K$ и в силу отсутствия конечной $\epsilon_0$-сети обязательно $\exists x_2 \in K : \norm{x_1 - x_2} \ge \epsilon_0$. 
Если бы такой точки не было, то тогда бы множество, состоящее из одной точки $x_1$ было бы конечной $\epsilon_0$-сетью. Имея теперь $\{ x_1, x_2\}$ в силу отсутсвия 
конечной $\epsilon$-сети $\exists x_3 \in K : \norm{x_j - x_3} \ge \epsilon_0, \; j = 1, 2$. Если бы не было, то $\{ x_1, x_2\}$ было бы конечной $\epsilon_0$-сетью для $K$ 
и так далее по индукции. В результате выстраивается последовательность точек $\{x_1, x_2, \dots \; x_j \in K\} : \norm{x_n - x_m} \ge \epsilon_0 \; \forall n \neq m$. А из 
такой последовательности точек очевидно не выделить сходящуются подпоследовательность, значит множество $K$ не секвенциально компактно, а значит и не компактно. 
Противоречие.

\medskip
\noindent\textbullet~Теперь в другую сторону. Пусть $K$ - замкнуто и вполне ограничено. Проверим, что $K$ - секвенциальный компакт, а тогда $K$ - компакт. Для этого возьмем 
любую последовательность $X_n \in K$, тогда необходимо показать, что $\exists n_1 < n_2 < \dots : \{ x_{n_k}\}$ - сходится.

\smallskip
\noindent\textbullet~Возьмем последовательность $\epsilon_m = \frac{1}{m}$. Для $\epsilon_1 \; \exists \epsilon_1$-сеть, например, $a_1, \dots, a_p$. Значит само множество 
$K$ можно покрыть объединением шаров $K \subset \bigcup_{j = 1}^p \overline{V}_{\epsilon_1}(a_j)$. Так как их конечное число, то в одном из этих шаров окажется бесконечно 
много элементов последовательности $\{ x_n \}$. Обозначим этот шар $\overline{V}_{\epsilon_1}$. Возьмем теперь $\epsilon_2$ и ему будет существовать конечная $\epsilon_2$-сеть $b_1, b_2, \dots$. И тогда $K$ точно также будет конечно покрыт шарами радиуса $\epsilon_2$ с центром в точке $b_j$. А тогда по той же причине, что и 
выше, найдется шар, в котором будет бесконечно много элементов той части нашей последовательности, которые уже лежат в шаре $\overline{V}_{\epsilon_1}$, обозначим этот 
шар $\overline{V}_{\epsilon_2}$. И так далее до бесконечности. Так как шар $\overline{V}_{\epsilon_1}$ содержит бесконечно много $x_n$ обозначим один из них $x_{n_1}$. 
Так как шар $\overline{V}_{\epsilon_2}$ содержит бесконечно много $x_n$ из шара $\overline{V}_{\epsilon_1}$, то там найдется $x_n$ с номером, большим ${n_1}$ и 
обозначим его $x_{n_2} (n_1 < n_2)$. И так далее. В результате выстроится система номеров $(n_1 < n_2 < \dots): x_{n_k} \in \overline{V}_{\epsilon_j}$, в котором $j \le k$
.

\smallskip
\noindent\textbullet~Рассмотрим теперь точки $x_{n_k}, x_{n_{k + p}}$. $x_{n_k}, x_{n_{k + p}} \in \overline{V}_{\epsilon_k} \Rightarrow \norm{x_{n_{k + p}} - x_{n_k}} \le
2 \epsilon_k$ - расстояние не превосходит двух радиусов шара. $\epsilon_k \to 0 \Rightarrow x_{n_{k + p}} - x_{n_k} \to 0$, $k, p \to \infty$. Поскольку $X$ - В-пространство, то $\exists x = \lim x_{n_k}$. То есть из произвольной последовательности точек мы выделили сходящуюся подпоследовательность, значит $K$ - секвенциальный 
компакт, а значит и компакт.
\end{proof}
\section{Ядро и область значений линейного оператора.}
\subsection*{Пространство $l_p$.}

\noindent\checkmark~Если рассматривать конкретные В-пространства, то базируясь на теореме Хаусдорфа, то есть записываю ее в терминах этого конкретного пространства, 
можно получать конструктивные критерии компактности в этих пространсвах. Рассмотрим пространство $l_p$.

\smallskip 
\noindent\textbullet~$l_p = \{ \overline{x} = (x_1, x_2, \dots) : \sum_1^\infty \abs{x_k}^p < +\infty\}, \; p \ge 1$. Норма $\norm{\overline{x}}_p = \left( \sum_1^\infty 
\abs{x_k}^p\right)^\frac{1}{p}$. Это пространство Банахово. 

\begin{theorem*}
Пусть $K \subset l_p$. Тогда $K$ - относительный компакт $\Longleftrightarrow$ :

\smallskip
1)~$K$ - ограничена в  $l_p$;

\smallskip 
2)~$\forall \epsilon > 0 \; \exists n_\epsilon \in \mathbb{N} : \forall \overline{x} = (x_1, x_2, \dots) \in K \Rightarrow \sum_{k = n_\epsilon + 1}^\infty \abs{x_k}^p \le 
\epsilon^p$.

\end{theorem*}

\begin{proof}
\smallskip\par\noindent\textbullet~Пусть $K$ - относительный компакт в $l_p$. $K$ - ограничен, поскольку это общий факт нормированных пространств. Проверим свойство 2. 
По т.~Хаусдорфа $\forall \epsilon > 0$ у $K \; \exists$ конечная $\epsilon$-сеть, состоящая из точек $\overline{b}_1, \dots, \overline{b}_p$, $\overline{b_j} = (b_1^{(j)},
b_2^{(j)}, \dots)$. Возьмем $\forall \overline{x} \in K$, подбираем соответствующую $\overline{b_j} : \norm{\overline{x} - \overline{b_j}}_p \le \epsilon$. 

\smallskip 
\noindent\textbullet~Рассмотрим сумму: (3) - неравенство Минковского
\begin{align*}
    \left( \sum_{k = n}^\infty \abs{x_k}^p\right)^\frac{1}{p} = \left( \sum_{k = n}^\infty \abs{(x_k - b_k^{(j)}) + b_k^{(j)}}^p \right)^\frac{1}{p} \le 
    \left( \sum_{k = n}^\infty \left(\abs{(x_k - b_k^{(j)})} + \abs{b_k^{(j)}}\right)^p \right)^\frac{1}{p} \le^{(3)} 
    \left( \sum_{k = n}^\infty \abs{(x_k - b_k^{(j)})}^p\right)^\frac{1}{p} + \\ \left(\sum_{k = n}^\infty \abs{b_k^{(j)}}^p \right)^\frac{1}{p} \le 
    \norm{\overline{x} - \overline{b}_j}_p + \left(\sum_{k = n}^\infty \abs{b_k^{(j)}}^p \right)^\frac{1}{p} 
\end{align*}

\noindent\textbullet~$\norm{\overline{x} - \overline{b}_j}_p \le \epsilon$, поскольку $\overline{b}_j \in l_p$, то $\sum_{k = 1}^\infty \abs{b_k^{(j)}} < + \infty$ - 
сходится, то и хвост ряда $\left(\sum_{k = n}^\infty \abs{b_k^{(j)}}^p \right)^\frac{1}{p} \to 0$, $n \to \infty$. А значит $\forall n \ge N_j$ сумма ряда $ \sum_{k = n}^\infty \abs{b_k^{(j)}}^p \le \epsilon^p$, а тогда если взять номер $n_\epsilon = N_1 + \dots + N_p + 10$, где 10 - произвольное число $\Rightarrow n_\epsilon \ge N_j$,
а тогда для этого номера $\forall j = \overline{1, p}$ суммы $\sum_{k = n_\epsilon}^\infty \abs{b_k^{(j)}}^p \le \epsilon^p$, а тогда если в
$\left(\sum_{k = n}^\infty \abs{b_k^{(j)}}^p \right)^\frac{1}{p}$ подставить вместо $n$ $n_\epsilon$, то тогда получится, что мы взяли $\forall x \in K$ и нашли номер $n_\epsilon : \sum_{k = n_\epsilon}^\infty \abs{x_k}^p \le 2 \epsilon$. Доказали необходимое условие.

\medskip 
\noindent\textbullet~Докажем достаточность. Не умаляя общности считаем, что $p = 1$. Установим, что из свойств 1, 2 вытекает вполне ограниченное множество $K$. Тогда по т.~Хаусдорфа $K$ будет относительно 
компактно. По свойству 1 $\exists const \; a > 0 : \norm{\overline{x}}_1 \le a$ для $\forall x \in K$. Однако $\abs{x_j} \le \norm{\overline{x}}_1 \Rightarrow K$ - 
покоординатно ограничено. Пусть $M$ - натуральное число из условия 2. Обозначим $S$ - конечное множество точек $\overline{y}_j$ из $l_1$ вида $y_{M + 1}^{(j)} = 
y_{M+2}^{(j)} = \dots = 0$, а при $i = \overline{1, M}$ $y_i^{(j)}$ одна из точек $a_s$ разбиения $[-a, a]$ на конечное число частей длиной не больше $\frac{\epsilon}{M}$.

\smallskip
\noindent\textbullet~Теперь для $\forall \overline{x} = (x_1, x_2, \dots) \in K$ будет $:$ если $i = \overline{1, M}$, то покоординатной ограниченности $x_i \in [-a, a]$ 
а значит $\exists a_s : \abs{x_i - a_s} \le \frac{\epsilon}{M}$. При $i > M$ все $y_i^{(j)} = 0 \Rightarrow \abs{x_i - y_i^{(j)}} = \abs{x_i}$. Имеем $\norm{\overline{x} -
\overline{y}_j}_1 = \sum_{i = 1}^M \abs{x_i - y_i^{(j)}} + \sum_{i = M + 1}^\infty \abs{x_i} \le \sum_{i = 1}^M \frac{\epsilon}{M} + \epsilon = 2 \epsilon$.

\smallskip
\noindent\textbullet~Поэтому $S$ - конечная $2\epsilon$-сеть для $K$.
\end{proof}
\section{Собственные векторы и собственные значения линейного оператора.}
\noindent\textasteriskcentered~Множество $S \subset C[a, b]$ называется равностепенно непрерывным семейством функций, если $\forall \epsilon > 0 \; \exists \delta > 0 : 
\abs{t'' - t'} \le \delta$, $t', t'' \in [a, b] \Rightarrow \abs{f(t'') - f(t')} \le \epsilon$ для $\forall f \in S$. 

\smallskip
\noindent\checkmark~Приведем без доказательства классическим критерий компактности в $C[a, b]$.

\begin{theorem*}[Арцела-Асколи]
Множество $K \subset C[a,b]$ - относительно компактно $\Longleftrightarrow$:

\smallskip 
\noindent 1)~$K$ - ограничено в $C[a, b]$;

\smallskip
\noindent 2)~$K$ - равностепенно непрерывно в $C[a, b]$.
\end{theorem*}
\section{Критерий представления матрицы в диагональном виде.}
\subsection*{Линейные оператор. Непрерывность. Ограниченность.}

\noindent\textasteriskcentered~Пусть $X, Y$ - НП, $A : X \rightarrow Y$ - отображение $X$ в $Y$, удовлетворяющее условию $A(\alpha x_1 + \beta x_2) = \alpha A x_1 + 
\beta A x_2$.  В этом случае говорят, что $A$ - \textit{линейный оператор}. Так как пространства нормированные, то можно говорить о непрерывности линейного оператора, 
что означает $x_n \to x \Rightarrow A_{x_n} \to A x$. Тогда говорят, что линейный оператор непрерывен в точке $x$. 

\smallskip
\noindent\textbullet~За счет линейности, если оператор непрерывен хотя бы в
одной точке, тогда он будет непрерывен и в любой точке. Пусть $A$ - \textit{непрерывен} в $x^*$, это означает, что если $x_n \to x^* \Rightarrow A x_n \to A x^*$. По арифметике
предела это означает, что $A x_n - A x^* \to 0$. По линейности оператора $A x_n - A x^* = A(x_n - x^*)$. Так как $x_n \to x^* \Rightarrow 
x_n - x^* \to 0$, тогда получается $A(x_n - x^*) \to 0$, что равно $A 0$ - значение оператора в нуле. Таким образом, из непрерывности в одной точке будет вытекать
непрерывность в нуле, а значит и в любой точке. 

\smallskip
\noindent\textbullet~Пояснение про нуль. Пусть $z_n \to 0 \Rightarrow A z_n \to A 0 = 0$. 
Поскольку $0 = \alpha 0$, тогда $A 0 = A(\alpha 0) = \alpha A(0)$. Тогда получится, что равенство $A(0) = \alpha A(0)$ верно при $\forall \alpha \Rightarrow A(0) = 0$.
Если знаем, что из $z_n \to 0 \Rightarrow A z_n \to 0$, тогда возьмем $\forall x \in X$, $x_n \to x \Rightarrow x_n - x \to 0$, тогда по непрерывности в нуле $A(x_n - x) \to 0$. По
линейности оператора $A(x_n - x) = A x_n - A x \to 0 \Longleftrightarrow A x_n \to Ax$ - значит линейный оператор непрерывен в точке $x$. Итого, линейный оператор
непрерывен в точке $x$ $\Longleftrightarrow$ оператор непрерывен в $0$.

\smallskip
\noindent\textasteriskcentered~Если $\exists M$ - $const > 0 : \forall x \in X \Rightarrow \norm{Ax} \le M \cdot \norm{x} \Rightarrow$ $A$ - \textit{ограниченный} оператор. 

\begin{theorem*}
    Линейный оператор $A$ - непрерывен $\Longleftrightarrow$ $A$ - ограничен
\end{theorem*}
\begin{proof}
\smallskip
\par\noindent\textbullet~Пусть оператор $A$ ограничен $\Rightarrow \norm{A x} \le M \cdot \norm{x}$. Если $x_n \to 0$, то написав неравенство $\norm{A x_n} \le 
M \cdot \norm{x_n}$ можно заметить, что правая часть $ \to 0$ $\Rightarrow \norm{A x_n} \to 0 \Rightarrow A x_n \to 0$. Таким образом, из ограниченности оператора 
вытекает непрерывность в нуле, а значит и в любой точке тоже.

\medskip
\noindent\textbullet~Пусть $A$ - непрерывен, допустим, что он не ограничен. Тогда $\forall n \in \mathbb{N}$ всегда $\exists x_n \in X : \norm{A x_n} > n \cdot 
\norm{x_n}$. Значит отсюда получится по линейности оператора и свойствам нормы $\norm{A \left( \dfrac{x_n}{n \cdot \norm{x_n}}\right)} > 1$. Если рассмотреть точки 
$y_n = \dfrac{x_n}{n \cdot \norm{x_n}}$, то очевидно окажется, что $\norm{y_n} = \dfrac{1}{n} \to 0$. Таким образом, $y_n \to 0 \Rightarrow$ по непрерывности 
$A y_n \to 0$, однако у нас выполняется неравенство $\norm{A y_n} > 1$, что противоречит тому, что $A y_n \to 0$. Значит оператор ограничен.
\end{proof}

\subsection*{Норма оператора. Аксиомы нормы.}

\noindent\textasteriskcentered~Для ограниченных операторов, если $\norm{x} \le 1$, то так как $\norm{A x} \le M \cdot \norm{x} \le M$, то тогда получаем, что $\sup \norm{A x}$ на 
единичном шаре конечен: $\sup_{\norm{x} \le 1} \norm{A x} < + \infty$. Эта величина называется \textit{нормой оператора} и обозначается $\norm{A}$. То есть $\norm{A} = 
\sup_{\norm{x} \le 1} \norm{A x}$. Если взять $\forall x \in X$, то точка $\dfrac{x}{\norm{x}}$ будет иметь единичную норму, то тогда по определению нормы оператора 
$\norm{A (\dfrac{x}{\norm{x}})} \le \norm{A}$, так как $\norm{x} \le 1$. С другой стороны $\norm{A (\dfrac{x}{\norm{x}})} = \dfrac{1}{\norm{x}} \cdot \norm{A x}$. А тогда получится, $\norm{A x} 
\le \norm{A} \cdot {\norm{x}} \; \forall x \in X$.

\medskip
\noindent\textbullet~Проверим, что $\norm{A}$ удовлетворяет всем аксиомам нормы на линейном многообразии. Рассмотрим линейное многообразие ограниченных операторов $V(X, Y)
= \{ A $ - линейный ограниченный оператор $: X \to Y\}$ ($V$ - знак линейной оболочки). Арифметические действия с операторами определяем поточечно: $(\alpha A)(x) = \alpha \cdot A(x)$, $(A + B)(x) = 
A x + B x$, при этом каждый раз будут получаться ограниченные операторы. Действительно если начать смотреть $\norm{(A + B) \cdot x} = \norm{A x + B x} \le \norm{A x} + 
\norm{B x}$, а так как операторы $A, B$ ограничены, то $\norm{A x} + \norm{B x} \le \norm{A} \norm{x} + \norm{B} \norm{x} = (\norm{A} + \norm{B}) \cdot \norm{x}$ = $const
\cdot \norm{x}$, значит оператор ограничен. В частности, если $\norm{x} \le 1 \Rightarrow \norm{(A + B) \cdot x} \le \norm{A} + \norm{B}$. Переходя к $\sup$ по $\norm{x}
\le 1 \Rightarrow \norm{A + B} \le \norm{A} + \norm{B}$ - доказали неравенство треугольника.

\medskip
\noindent\textbullet~Докажем аксиому $\norm{\alpha A} = \abs{\alpha}\norm{A}$. Считаем, что $\norm{x} \le 1$, вычисляем $\norm{(\alpha A)x} = \norm{\alpha A x} = 
\abs{\alpha} \cdot \norm{A x} \le \abs{\alpha} \cdot \norm{A}$. Таким образом, получили $\norm{\alpha A} \le \abs{\alpha} \cdot \norm{A}$.

\smallskip
\noindent\textbullet~Проверим противоположное неравенство. Запишем тождество $\norm{A} = \norm{\alpha (\dfrac{1}{\alpha} A)}$. В силу только что доказанного неравенства
подставляем $\dfrac{1}{\alpha}: \norm{\alpha (\dfrac{1}{\alpha} A)} \le \dfrac{1}{\abs{\alpha}} \cdot \norm{\alpha A} \Rightarrow \abs{\alpha} \cdot \norm{A} \le  
\norm{\alpha A}$. Получили противоположное неравенство, значит $\abs{\alpha} \cdot \norm{A} = \norm{\alpha A}$. Проверили вторую аксиому. Первая аксиома очевидна. 

\medskip 
\noindent\textbullet~Теперь можно вести разговор о линейном многообразии ограниченных оператор. В этом многообразии величина $\norm{A} = \sup_{\norm{x} \le 1} \norm{A x}$
задает норму. И значит наше линейное многообразие превращается в линейное нормированное пространство. Значит мы можем говорить об операторе $A = \lim A_n$, понимая под 
этим тот факт, что $\norm{A_n - A} \to 0$, то есть $\forall \epsilon > 0 \; \exists N : \forall n \ge N \Rightarrow \norm{A_n - A} \le \epsilon$. Норма разности - 
$\sup_{\norm{x} \le 1} \norm{A_n x - A x}$, а тогда получается, что $\norm{A_n - A} \to 0$ можно перезаписать в форме $\forall \epsilon > 0 \; \exists N : \forall n \ge 
N$ и $\forall x : \norm{x} \le 1 \Rightarrow \norm{A_n x - A x} \le \epsilon$. Последнее поточеное неравенство должно выполняться сразу для всех иксов для единичного 
шара, начиная с какого-то номера.

\section{Евклидово пространство. Определение. Неравенство Коши-Буняковского. Неравенство треугольника.}
\noindent\textbf{Утверждение.}\textit{ Пусть $X$ - НП, $Y$ - B-пространство. Тогда пространство линейных ограниченных операторов из $X$ в $Y$ $V(X, Y)$ будет 
B-пространством.}

\begin{proof}
\smallskip
\par\noindent\textbullet~Необходимо убедиться в том, что $A_n - A_m \to 0$ в $V(X, Y) \Rightarrow \exists A \in V(X, Y) : A_n \to A$, тогда пространство будет Банаховым.

\smallskip
\noindent\textbullet~Так как $A_n - A_m \to 0$, то тогда $\forall \epsilon \; \exists N : \forall n, m \ge N, \forall x : \norm{x} \le 1 \Rightarrow \norm{A_n x - A_m x} 
\le \epsilon$. 

\smallskip
\noindent\textbullet~Также можем написать, что $\forall x \in X \norm{A_n x - A_m x} \le \norm{A_n - A_m} \cdot \norm{x} \le \epsilon \cdot \norm{x}$ по свойствам нормы.

\smallskip
\noindent\textbullet~Из последнего неравенства видно, что $\forall x$ последовательность $\{ A_n x\}$ сходится в себе в $Y$, а так как $Y$ - полное пространство, то 
тогда у такой последовательности 
должен $\exists \lim A_n x$ в $Y$. Обозначим этот предел $A x$ и проверим, что оператор $A$ ограничен и является пределом оператора $A_n$. Если мы это проверим, тогда 
у сходящейся последовательности операторов будет существовать предел. 

\smallskip
\noindent\textbullet~Имеется неравенство $\norm{A_n x - A_m x} \le \epsilon$ $\forall \epsilon > 0$ c $N$ и $\forall x : \norm{x} \le 1$ - на единичном шаре. Устремим 
$n \to \infty \Rightarrow A_n x \to A x$ по определению оператора $A$, а тогда по свойствам предела $\norm{A x - A_m x} \le \epsilon \; \forall \epsilon > 0, \forall m 
\ge N, \forall x : \norm{x} \le 1$. Отсюда получается $\norm{A x} = \norm{(A x - A_m x) + A_m x} \le \norm{A x - A_m x} + \norm{A_m x}$. Если взять, например, $\epsilon 
> 1$, то нашлось бы $N$, такое что $m > N, \norm{x} \le 1$, а тогда из получившегося неравенства $\norm{A x} \le 1 + \norm{A_{m_0}} \Rightarrow \norm{A} \le 1 + \norm{A_{m_0}}$, 
а значит она конечна и оператор окажется ограниченным. Тогда если вернуться к факту $\norm{A - A_m} \le \epsilon$ $\forall m \ge N$, это будет однозначно обозначать, что
$A = \lim A_m$ в пространстве $V(X, Y)$, а значит оно окажется полным.
\end{proof}

\section{Дифференциальные уравнения первого порядка. Понятия уравнения и его решения. Поле направлений. Задача Коши. Теорема Пикара. Общее, частное и особое решение.}
\subsection*{Теорема о продолжении по непрерывности.}

\noindent\textbullet~Пусть $Z$ - линейное многообразие в $X$. Также пусть имеется $Y$ (это все НП), а также оператор $A : Z \to Y$. Раз $Z$ - линейное многообразие, то будем предполагать, что $A$ - ограничен на $Z$. Норма $\norm{A}$ на $Z$, $\norm{A} = \sup_{x \in Z} \norm{A x}, \norm{x} \le 1$. Возникает 
вопрос, при каких условиях на $Z$ и $Y$ $\exists \hat{A} \in V(X, Y) : $

1)$\hat{A} \big|_z = A$;

2)$\norm{\hat{A}} = \norm{A}$. 

\noindent\textasteriskcentered~Тогда говорят, что оператор $\hat{A}$ является \textit{продолжением ограниченного оператора $A$ по непрерывности}. 

\medskip
\begin{theorem*}
Пусть $Z$ всюдо плотно в $X$, $Y$ - Банахово пространство. Тогда оператор $\hat{A}$ существует и только один. 
\end{theorem*}

\begin{proof}
\smallskip
\par\noindent\textbullet~По условию $Cl Z = X$, это означает, что $\forall x \in X \; \exists z_n \in Z : x = \lim z_n$ в $X$. Рассмотрим последовательность значений 
оператора $A : Z \to Y$ на точках $z_n$. По условию оператор $A$ - ограничен на $Z$, то есть $\norm{A} = \sup_{z \in Z} \norm{A z} < + \infty, \norm{z} \le 1$.

\smallskip
\noindent\textbullet~Рассмотрим $\norm{A z_m - A z_n} = \norm{A (z_m - z_n)} \le \norm{A} \cdot \norm{z_m - z_n}$. Норма $\norm{z_m - z_n} \to 0$, так как $z_m \to x$, а 
тогда из написанного неравенства $\norm{A z_m - A z_n} \to 0 \Rightarrow A z_m $ сходится в себе в пространстве $Y$, которое полное. Значит у этой последовательности 
существует предел, обозначим его $\hat{A}x$. 

\smallskip
\noindent\textbullet~Проверим, что наше определение является корретным в том смысле, если помимо $z_m \to x$ найдется последовательность $z_m' \to 
x$, то тогда $\lim A z_m = \lim A z_m'$. Для того, чтобы это проверить составим $\norm{A z_m - A z_m'} = \norm{A (z_m - z_m')} \le \norm{A} \norm{z_m - z_m'}$, где 
$\norm{z_m - z_m'} \to 0$, а тогда $\norm{A z_m - A z_m'} \to \Rightarrow \lim A z_m = \lim A z_m'$.

\smallskip
\noindent\textbullet~Итак мы проверили, что формула $z_m \in Z, z_m \to x, \hat{A} x = \lim A z_m$ корректно определяет некоторый оператор $\hat{A}$, заданный на всем $X$.
По арифметике предела последовательностей ясно, что оператор $\hat{A}$ является линейным. Если при этом в этой формуле $x \in Z$, то последовательность $z_m = x \to x$, 
а тогда получается, что значение оператора $\hat{A} x = A x$, то есть этот линейный оператор является продолжением линейного оператора $A$ со всюду плотного линейного 
многообразия $Z$ на все $X$. Осталось проверить, что оператор $\hat{A}$ ограничен и его норма $\norm{\hat{A}} = \sup_{x \in X, \norm{x} \le 1} \norm{\hat{A} x} 
= \norm{A}$.

\medskip
\noindent\textbullet~Доказывая ограниченность, воспользуемся формулой, с помощью которой мы продолжали оператор. $\forall x \in X, z_m \to x, z_m \in Z$, $A$ - ограничен.
Тогда можем писать $\norm{A z_m} \le \norm{A} \cdot \norm{z_m}$, где $\norm{z_m} \to \norm{x}$, $\norm{A z_m} \to \norm{\hat{A}x}$, в пределе это неравенство сохранится, 
тогда получится неравенство $\norm{\hat{A} x} \le \norm{A} \cdot \norm{x} \Rightarrow \hat{A}$ - ограничен и подставляя в это неравенство $\norm{x} \le 1$ получаем 
$\norm{\hat{A} x } \le \norm{A} \Rightarrow \norm{\hat{A}} \le \norm{A}$. Противоположное неравенство $\norm{\hat{A}} \ge \norm{A}$ - очевидно, так как оператор $\hat{A}$ 
является продолжение оператора $A$ на все $X$ ($\hat{A}\big|_z = A$). Таким образом, мы проверили, что построенный оператор ограничен и его норма совпадает с нормой 
исходного оператора. 

\medskip 
\noindent\textbullet~Если бы существовал другой оператор $\tilde{A}$ с такими же свойствами, то есть 1) $\tilde{A}\big|_z = A$, 2) $\norm{\tilde{A}} = \norm{A}$, то 
тогда бы получилось, что мы опять берем $\forall z \in X, \; \exists z_m \to z, z_m \in Z$. Тогда оба продолженных оператора $\hat{A}, \tilde{A}$ были бы непрерывными, 
поэтому $\hat{A} z_m \to \hat{A} x$, $\tilde{A} z_m \to \tilde{A} x$ по непрерывности. Но так как оба этих оператора продолжают оператор $A$ с многообразия $Z$, то точки 
$\tilde{A}z_m = \hat{A} z_m = A z_m$, а тогда в этих двух предельных соотношения слева можно подставлять $A z_m \to \hat{A} x, \tilde{A} x$, а тогда по единственности 
предела точки $\hat{A} x = \tilde{a} x$ совпадут, что верно для всех $x$, а значит это два одинаковых оператора.
\end{proof}


\subsection*{Иллюстрация неограниченного оператора. Операторы в пространствах $l_p$, $C[0, 1]$.}

\noindent\checkmark~Убедимся, что есть линейные неограниченные операторы. Для этого рассмотрим в качестве $X = \{$непрерывно дифф на $[0, 1] f, \norm{f} = \max_{[0, 1]}
\abs{f}\}$. Ясно, что это линейное многообразие. В качестве $Y$ возьмем $C[0,1] = \{$ непр $[0, 1] f$ с sup-нормой, определенной выше $\}$. Рассмотрим оператор $A : X \to 
Y$, $A(f) = f'$, этот оператор линейный по правилам дифференциирования. Если бы он был ограничен, то $\norm{A(f)} \le M \cdot \norm{f}$, $M$ - $const$. Или, подставлял 
значение оператора, что $\norm{f'} \le M \cdot \norm{f}$. Очевидно, что это невозможно, потому что можно взять асцилирующую фунцию, соответственно значение производной 
у нее может быть сколь угодно большой. Соотетсвенно, этот линейный оператор не ограничен. 

\smallskip
\noindent\checkmark~Мы определили $\norm{A} = \sup_{\norm{x} \le 1} \norm{A x}$, но следует понимать, что в конкретных ситуациях само вычисление нормы оператора может 
оказаться неподъемной задачей, поэтому в большинстве приложений можно судить о норме оператора только записывая оценки этой нормы, потому что точное значение нормы не 
сосчитать.

\bigskip 
\noindent\textbullet~В пространствах $l_p$ операторы возникают на основе стандартных базисных точек $\overline{e}_n = \{ 0, 0, \dots, 0, 1, 0, \dots \}$. Например, 
рассмотрим $\lambda_j \in \mathbb{R} : \abs{\lambda_j} \le M$. Возьмем $\forall \overline{x} = (x_1, x_2, \dots) \in l_p$ и рассмотрим формальный ряд $\sum_1^\infty 
\lambda_n x_n \overline{e}_n$. В НП должны смотреть, что частичные суммы ряда $S_n = \sum_{k = 1}^\infty \lambda_k x_k \overline{e}_k$ и определить есть ли у него
предел или нет, тогда $S_n = (\lambda_1 x_1, \lambda_2 x_2, \dots, \lambda_n x_n, 0, \dots)$. 

\medskip
\noindent\textbullet~Если теперь рассмотреть точку $S = (\lambda_1 x_1, \lambda_2 x_2, \dots, \lambda_n x_n , \dots)$ и составить сумму $\sum_1^\infty 
\abs{\lambda_n x_n}^p \le M^p \le \sum_1^\infty \abs{x_n}^p < +\infty$, так как $\sum_1^\infty \abs{x_n}^p < +\infty$. Тогда эта последовательность $S \in l_p$. Если 
теперь рассмотреть $\norm{S - S_n}_p = \norm{(0, \dots, 0, \lambda_{n+1} x_{n + 1}, \dots)}_p = (\sum_{k = n + 1}^\infty \abs{\lambda_k}^p \abs{x_k}^p)^\frac{1}{p} \le 
M (\sum_{k = n+ 1}^\infty \abs{x_k}^p)^\frac{1}{p}$, где $\sum_{k = n + 1}^\infty \abs{x_k}^p \to 0$ по условию вхождения в пространство $l_p$, так как 
$\sum_{k = 1}^\infty \abs{x_k}^p < +\infty$. Тогда $S = \lim S_n$. Таким образом, $A\overline{x} = \sum_{n = 1}^\infty \lambda_n x_n \overline{e}_n, \abs{\lambda_n} \le M$ 
определеляет линейный оператор $l_p \to l_p$. Предыдущие вычисления показывают, что $\norm{A \overline{x}}_p \le M \norm{\overline{x}}_p \Rightarrow \norm{A} \le M < 
+\infty$.

\medskip
\noindent\textbullet~Теперь рассмотрим $C[0, 1]$, $K(u,v)$ - непрерывная фукнция двух переменных на $[0,1] \times [0, 1]$, $M = \max_{[0, 1]^2} \abs{K(n, v)} < +\infty$, 
это значение конечно по т.~Вейерштрасса о непрерывных функциях. Теперь возьмем любую функцию $f \in C[0, 1]$, $A(f, x) = \int_0^1 K(x, t) f(t) dt$, определяющая некоторую 
функцию переменной $x$, она непрерывна по свойствам интеграла и является линейным оператором $C[0, 1] \to [0, 1]$. Также по свойствам интеграла $\abs{A(f, x)} \le 
\int_0^1 \abs{K(x, t)} \abs{f(t)} dt \le \int_0^1 M \norm{f} dt = M \cdot \norm{f}$. Написанный оператор $\norm{A f} \le M \norm{f}, \norm{A}$ ограничен и его 
норма  $\le M$. В функциональном анализе такой класс обозначается как операторы Фредгольма.

\section{Методы интегрирования уравнений первого порядка. Уравнения с разделяющимися переменными. Однородные уравнения и уравнения, приводящиеся к однородным.}
\noindent\textbullet~Далее, не оговаривая, мы считаем, что все пространства являются B-пространствами, то есть $x_n - x_m \to 0 \Rightarrow \exists \lim x_n$.
Обратное верно всегда.

\smallskip 
\noindent\textbullet~Для доказательства потребуется принцип вложенных шаров: $X$ - B-пространство, $\overline{V}_n$ - замкнутые шары в $X$, $\overline{V}_{n+1} \subset 
\overline{V}_n \Rightarrow \bigcap_{n = 1}^\infty \overline{V}_n \neq \emptyset$, общих точек не обязательно должна быть единственной. 
Если $r \to 0$, то общих точек только 1. Например, $\overline{V}_n = [-1 - \frac{1}{n}, 1 + \frac{1}{n}]$.

\begin{theorem*}[Банах, Штейнгауз]
Пусть дана последовательность линейных ограниченных операторов $A_n \in V(X, Y)$, про которую известно, что $\forall x \in X$ $\sup_{n \in \mathbb{N}}\norm{A_n x} < 
+ \infty$ (то есть последовательность операторов поточечно равномерно ограничена). Тогда $\sup_{n \in \mathbb{N}} \norm{A_n} < + \infty$ (то есть последовательность 
операторов просто равномерно ограничена). Равномерность означает конечность соответствующих супремумов.
\end{theorem*}

\begin{proof}
\smallskip 
\par\noindent\textbullet~Доказательство разобьем на 2 этапа.

\smallskip
\noindent\textbullet~Допустим $\exists \overline{V} = \overline{V}_r(a) : \sup_{x \in \overline{V}, n \in \mathbb{N}} \norm{A_n x} < +\infty$. Покажем тогда, что можно 
утверждать, что из этого факта будет вытекать, что $sup_{n \in \mathbb{N}} \norm{A_n} < +\infty$. Обозначим для удобства $M = \sup_{x \in \overline{V}, n \in \mathbb{N}}
\norm{A_n x}$. 

\smallskip
\noindent\textbullet~Рассмотрим единичный шар $\overline{V}_1 = \overline{V}_1(0)$. По нему считаются нормы операторов. Возьмем $\forall x \in \overline{V}_1$ и определяем 
$y = a + r x$. Если составить норму разности $\norm{y - a} = \norm{r x} = r \norm{x} \le r$. Таким образом, точка $y \in \overline{V}$. Значит $\forall n \in \mathbb{N}
\Rightarrow \norm{A_n y} \le M$. Из формулы $y = a + rx \Rightarrow x = \dfrac{y - a}{r}$, начинаем смотреть, что представляет норма значения $n$-ого оператора над точкой 
$x$, которая является любой.

\smallskip
\noindent\textbullet~$\norm{A_n x} = \norm{a_n (\dfrac{y - a}{r})} = \dfrac{1}{r} \norm{A_n y - A_n a} \le \dfrac{1}{r}(\norm{A_n y} + \norm{A_n a}) \le \dfrac{1}{r} 
(M + \norm{A_n a})$. Ясно, что норма $\norm{A_n a} \le \sup_{m \in \mathbb{N}} \norm{A_m a} = N < + \infty$ по условию теоремы (поточечно равномерно ограничена). 
Подставляя это в последнее неравенство $\norm{A_n x} \le \dfrac{1}{r} (M + N)$ - не зависит от $n$ и $x \in \overline{V}_1$. Тогда сначала переходим к $\sup$ по $x \in 
\overline{V}_1$, а тогда получаем, что $\norm{A_n} \le \dfrac{1}{r}(M + N)$. Теперь переходим к супремому по номерам $n \Rightarrow \sup_{n \in N} \norm{A_n} < + \infty$,
то есть будет выполняться утверждение теоремы Банаха-Штейнгауза. Первый этап проделали.

\medskip
\noindent\textbullet~Допустим, что $\nexists$ шара $\overline{V}$ из первого этапа и убедимся в том, что тогда появится противоречие 
(такой шар хотя бы один должен существовать). Возьмем $\forall \overline{V}$, тогда по предположению должна $\exists x_1 \in V, \exists n_1 \in \mathbb{N} : 
\norm{A_{n_1}(x_1)} > 1$. 

\smallskip
\noindent\textbullet~Оператор $A_{n_1}$ непрерывен (поскольку ограничен), тогда по стандартному свойству непрерывности $\exists \overline{V}_{r_1}(x_1) = \overline{V}_1 : 
\overline{V}_1 \subset \overline{V}, \forall y \in \overline{V}_1 \Rightarrow \norm{A_{n_1}(y)} > 1$. Построенный шар $\overline{V}_1$ не может быть шаром из первого 
этапа по нашему предположению, тогда $\exists x_2 \in V_1, \exists n_2 \in \mathbb{N} : \norm{A_{n_2}(x_2)} > 2$, при этом можно считать, что $n_2 > n_1$. Тогда опять по 
непрерывности $\exists \overline{V}_{r_2} (x_1) = \overline{V}_2 : \overline{V}_2 \subset V_1$, $r_2 < \frac{r_1}{2}$, $\forall y \in \overline{V}_2 \Rightarrow 
\norm{A_{n_2}(y)} > 2$ и так далее продолжаем это построение.

\smallskip
\noindent\textbullet~В результате выстраивается последовательность замкнутые вложенных шаров: $\overline{V}_{k+1} \in \overline{V}_k$, $r_k \to 0$, радиус каждый раз 
уменьшается в 2 раза, и при этом $\forall x \in \overline{V}_k \Rightarrow \norm{A_{n_k}(x)} > k$. По принципу вложенных шаров существует точка $x^* \in \bigcap_{k = 1}^\infty \overline{V}_k$. В частности эта точка принадлежит шару $\overline{V}_k$, а тогда $\norm{A_{n_k}(x^*)} > k$. Если в этом неравенстве $k \to \infty \Rightarrow
\norm{A_{n_k}(x^*)} \to +\infty$. А это противоречит тому, что $\sup_{m \in \mathbb{N}}\norm{A_n(x^*)} < +\infty$. Полученное противоречие доказывает, что шар из первого 
этапа существует, значит теорема доказана.
\end{proof}


\subsection*{Следствие из теоремы. Интерпретации $A  = \lim A_n$.}

\noindent\textbullet~Пусть $A_n \in V(X, Y), A = \lim A_n$. В функциональном анализе есть 3 разных понимания этого равенства.

\smallskip 
1) $\norm{A_n - A} \to 0$ - оператор $A$ является пределом по операторной норме. Это тоже самое, что $\forall \epsilon > 0 \exists n_0 : \forall n \ge n_0 \Rightarrow
\norm{A_n x - A_x} \le \epsilon$ сразу для всех $x$ из замкнутого единичного шара. - равномерная сходимость.

\smallskip 
2) $\forall x \in X \Rightarrow A x = \lim A_n x$ - сильная (поточечная) сходимость последовательности операторов.

\smallskip 
3) $\forall f$ - линейного ограниченого функционала, $\forall x \Rightarrow f(A x) = \lim f(A_n x)$ - слабая сходимость последовательности операторов.

\bigskip\noindent\textbf{Следствие.}\textit{ Пусть $A_n \in V(X, Y)$, про которую известно, что $\forall x \in X \Rightarrow \exists \lim A_n x = A x$. Тогда предельный 
оператор $A \in V(X, Y)$, то есть тоже ограничен (по сильному пределу). }

\begin{proof}
\smallskip
\par\noindent\textbullet~Возьмем $x \in \overline{V}_1 \Rightarrow \norm{A x} \le M$ - $const$. $\norm{A x} = \norm{(A x - A_n x) + A_n x} \le \norm{A x - A_n x} + 
\norm{A_n x}$. Для имеющегося $x$, так как можно написать $A x = \lim A_n x$, возьмем $\epsilon = 1, \; \exists n_0 : \forall n \ge n_0 \Rightarrow \norm{A x - A_n x} \le 
1$. В частности, получится $\norm{A x} \le 1 + \norm{A_{n_0} x}$. Норма $\norm{A_{n_0}x} \le \norm{A_{n_0}} \cdot \norm{x} \le \norm{A_{n_0}}$.

\smallskip
\noindent\textbullet~Так как $\forall x \; \exists \lim A_n x$ по условию следствия, тогда по стандартным свойствам предела $\{ \norm{A_n x}\}$ - ограничена. То есть для 
любого $x$ выполняется условия Банаха-Штейнгауза, а тогда $S = \sup \norm{A_n} < +\infty$. Тогда возвращаясь к неравенству $\norm{A x} \le 1 + \norm{A_{n_0} x}$ получаем, 
что $\norm{A x} \le 1 + \norm{A_{n_0}} \le 1 + S$ - $const$. А следовательно неравенство верно $\forall x \in \overline{V}_1 \Rightarrow \norm{A} < +\infty$.
\end{proof}
\section{Линейные уравнения первого порядка. Уравнения Бернулли.}
\input{tasks/task-26}
\section{Уравнения в полных дифференциалах. Интегрирующий множитель.}
\input{tasks/task-27}
\section{Уравнения первого порядка, не разрешенные относительно производной. Уравнения Лагранжа и Клеро.}
\input{tasks/task-28}
\section{Дифференциальные уравнения высших порядков. Основные понятия и определения. Задача Коши. Теорема Пикара. Понижение порядка уравнения. Уравнения, не содержащие искомой функции и последовательных первых производных. Уравнения, не содержащие независимой переменной.}
\input{tasks/task-29}
\section{Линейные дифференциальные уравнения n-го порядка. Свойства решений линейного однородного уравнения. Фундаментальная система решений и определитель Вронского. Признак линейной независимости решений. Формула Остроградского–Лиувилля.}
\input{tasks/task-30}
\section{Построение общего решения линейного однородного уравнения по фундаментальной системе решений. Структура общего решения неоднородного уравнения. Принцип наложения. Метод вариации произвольных постоянных (метод Лагранжа) для уравнения 2-го порядка. Случай уравнения  n-го порядка.}
\input{tasks/task-31}
\section{Системы дифференциальных уравнений. Основные понятия и определения. Нормальная система. Задача Коши. Механическое истолкование нормальной системы и ее решения. Теорема Пикара. Связь между уравнениями высшего порядка и системами дифференциальных уравнений 1-го порядка.}
\input{tasks/task-32}
\section{Линейные системы. Свойства линейных систем. Фундаментальная матрица. Определитель Вронского. Критерий линейной независимости вектор-функций. Формула Остроградского–Лиувилля.}
\input{tasks/task-33}
\section{Построение общего решения линейной однородной системы по фундаментальной системе решений. Интегрирование линейной однородной системы с постоянными коэффициентами методом Эйлера.}
\input{tasks/task-34}
\section{Структура общего решения неоднородной линейной системы. Метод вариации произвольных постоянных (метод Лагранжа).}
\input{tasks/task-35}
\section{Числовые ряды. Сходимость. Необходимый признак сходимости.}
\input{tasks/task-36}
\section{Свойства сходящихся рядов.}
\input{tasks/task-37}
\section{Признаки сравнения рядов с положительными членами.}
\input{tasks/task-38}
\section{Признаки Даламбера, Коши и интегральный сходимости рядов.}
\input{tasks/task-39}
\section{Знакочередующиеся ряды. Теорема Лейбница.}
\input{tasks/task-40}
\section{Знакопеременные ряды. Абсолютная и условная сходимость. Свойства абсолютно сходящихся рядов.}
\input{tasks/task-41}
\section{Функциональные ряды. Область сходимости. Мажорируемые ряды. Равномерная сходимость. Признак Вейерштрасса.}
\input{tasks/task-42}
\section{Непрерывность суммы ряда.}
\input{tasks/task-43}
\section{Интегрирование функционального ряда.}
\input{tasks/task-44}
\section{Дифференцирование функционального ряда.}
\input{tasks/task-45}
\section{Степенные ряды. Теорема Абеля. Интервал сходимости.}
\input{tasks/task-46}
\section{Радиус сходимости. Формулы Даламбера и Коши–Адамара.}
\input{tasks/task-47}
\section{Непрерывность суммы степенного ряда. Интегрирование и дифференцирование степенного ряда.}
\input{tasks/task-48}
\section{Ряды Тейлора и Маклорена. Критерий сходимости ряда Тейлора.}
\input{tasks/task-49}
\section{Тригонометрическая система функций. Ряд Фурье. Разложение периодической функции в ряд Фурье. Теорема Дирихле.}
\input{tasks/task-50}
\section{Ряды Фурье для чётных и нечётных функций. Ряд Фурье непериодической функции.}
\input{tasks/task-51}
\section{Ряд Фурье с комплексными членами.}
\input{tasks/task-52}

\end{sloppypar}
\end{document}