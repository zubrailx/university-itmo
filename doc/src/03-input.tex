\begin{markdown}
# Этап 3

## Запросы

### Для администраторов системы
* Редактирование доступных едениц измерения
* Редактирование доступных ингредиентов

### Для поставщика
* Создание и удаление аккаунта поставщика
* Размещение цен на товары

### Для владельца кафе
* Добавление, изменение, удаление кафе 
* Добавление, удаление, изменение меню
* Добавление, удаление меню в кафе
* Добавление, изменение, удаление рецептов
* Добавление, удаление рецептов в меню
* Поиск товаров у поставщиков
* Покупка товара у поставщиков
* Получение информации о купленных товаров, т.е. кассового чека, и обновление склада при изменении статуса чека на получено
* Редактирование и удаление записей об инвенторе

### Для пользователя
* Просмотр меню кафе
* Просмотр рецептов блюд

## Создание объектов

### Таблицы
\end{markdown}
\lstinputlisting[
  style = User-SQL,
  % caption = {create-functions.sql}
  ]{../../backend/sql/create-scheme.sql}

\begin{markdown}
### Функции
\end{markdown}
\lstinputlisting[
  style = User-SQL,
  % caption = {create-functions.sql}
  ]{../../backend/sql/create-functions.sql}

\begin{markdown}
### Триггеры
\end{markdown}
\lstinputlisting[
  style = User-SQL,
  % caption = {create-triggers}
  ]{../../backend/sql/create-triggers.sql}

\begin{markdown}
### Процедуры
\end{markdown}
\lstinputlisting[
  style = User-SQL,
  % caption = {create-procedures}
  ]{../../backend/sql/create-procedures.sql}

\begin{markdown}
## Удаление объектов

### Таблицы
\end{markdown}
\lstinputlisting[
  style = User-SQL,
  % caption = {drop-scheme}
  ]{../../backend/sql/drop-scheme.sql}

\begin{markdown}
### Функции
\end{markdown}
\lstinputlisting[
  style = User-SQL,
  % caption = {drop-functions}
  ]{../../backend/sql/drop-functions.sql}

\begin{markdown}
### Процедуры
\end{markdown}
\lstinputlisting[
  style = User-SQL,
  % caption = {drop-procedures}
  ]{../../backend/sql/drop-procedures.sql}

\begin{markdown}
## Скрипты
\end{markdown}

