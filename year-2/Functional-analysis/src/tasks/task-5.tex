\noindent \textbullet~Пусть дано какое-либо линейное многообразие $X$ и на нем определена не одна норма: $\norm{\cdot}_1$, $\norm{\cdot}_2$, $\norm{\cdot}_3 \dots$.
Если взять линейное многообразие $C[a, b]$ и определить на нем нормы $\norm{\delta} = max_{[a, b]} \abs{f}$, $\norm{f}_1 = \int_a^b \abs{f(x)} dx$.

\smallskip
\noindent \textbullet~Если брать $n$-мерное пространство: $\mathbb{R}^n = \{ \overline{x} = (x_1, x_2, \dots, x_n)\}$, то на нем можно например определить нормы:
$\norm{\overline{x}}_e = \sqrt{\sum_{j = 1}^n \abs{x_j}^2}$ - классическая евклидовская норма, $\norm{\overline{x}}_1 = \sum_{j = 1}^n \abs{x_j}$, $\norm{\overline{x}}_2 =
max{\abs{x_1}, \abs{x_2}, \dots, \abs{x_n}}$.

\smallskip 
\noindent \textbullet~Тогда встает вопрос, чем отличаются топологии?


\smallskip
\noindent \textbullet~Например, рассмотрим пространство $\mathbb{R}^2$. Задана евклидовская норма $\norm{\overline{x}}_e = \sqrt{x^2_1 + x^2_2}$ 
и норма $\norm{\overline{x}}_1 = \sum_{j = 1}^n \abs{x_j}$. Определим шар: $V_r(0) = \{ x_1^2 + x_2^2 < r\}$ по первой норме и квадрат по второй.

\begin{tikzpicture}
\begin{axis}[
    xmin=-11,xmax=11,
    ymin=-11,ymax=11,
    grid=both,
    grid style={line width=.1pt, draw=gray!10},
    major grid style={line width=.2pt,draw=gray!50},
    axis lines=middle,
    minor tick num=4,
    ticklabel style={font=\tiny,fill=white},
    xlabel style={at={(ticklabel* cs:1)},anchor=north west},
    ylabel style={at={(ticklabel* cs:1)},anchor=south west},
    axis equal,
    xlabel={$x$},
    ylabel={$y$}
]

\coordinate (O) at (0,0);
\node[fill=white,circle,inner sep=0pt] (O-label) at ($(O)+(-135:10pt)$) {$O$};
\node[fill=white,circle,inner sep=0pt] (r-label) at (5,3) {$r$};

\draw (axis cs: 5, 0) -- (axis cs: 0, 5);
\draw (axis cs: 0, 5) -- (axis cs: -5, 0);
\draw (axis cs: 5, 0) -- (axis cs: 0, -5);
\draw (axis cs: -5, 0) -- (axis cs: 0, -5);

\draw (O) circle [blue, radius=5];

\draw (O) circle [blue, radius=3];
\end{axis}
\end{tikzpicture}

\noindent \textbullet~В любом круге содержится квадрат, а в любом квадрате содержится соответствующий круг, а это сразу приводит к тому, что топология, порожденная
первой нормой, просто совпадет с топологией, порожденной второй, т.е. они тождественны. $\tau_e = \tau_1$.

\smallskip
\noindent \textbullet~Пример топологий, которые не совпадают: $\int$ и $max$, описанные выше.

\smallskip
\noindent \textbullet~В пространствах одной и той же размерности топологии все одинаковы, какую бы норму не писали. Именно поэтому конечномерные пространства одной 
размерности, они будут изоморфны друг другу.

\subsection*{Эквивалентных нормы.}
\noindent \textasteriskcentered~Пусть $X$ - линейное многообразие, на котором задано 2 нормы $\norm{\cdot}_1$, $\norm{\cdot}_2$. Они называются \textit{ эквивалентными} $\norm{\cdot}_1 \sim \norm{\cdot}_2$, 
тогда и только тогда когда существует пара положительных констант ($a$, $b$ $> 0$), таких что для любого $x \in X$ будет выполняться неравенство $a \cdot \norm{\cdot}_1 \le 
\norm{\cdot}_2 \le b \cdot \norm{\cdot}_1$.
\[
    \norm{\cdot}_1 \sim \norm{\cdot}_2 \Longleftrightarrow a \cdot \norm{\cdot}_1 \le \norm{\cdot}_2 \le b \cdot \norm{\cdot}_1
\]

\medskip
\noindent Бинарное отношение эквивалентности:
1) $a \sim a$;
2) $a \sim b \Longleftrightarrow b \sim a$;
3) $a \sim b, b \sim c \Rightarrow a \sim c$.

\bigskip 
\noindent \textbf{Проверим утверждение}. $\norm{\cdot}_1 \sim \norm{\cdot}_2 \Longleftrightarrow [x = \lim x_n (\norm{\cdot}_1) \Longleftrightarrow x = \lim x_n (\norm{\cdot}_2)]$.

\begin{proof}

\noindent \textbullet~$\norm{\cdot}_1 \sim \norm{\cdot}_2$, $a \cdot \norm{x}_1 \le \norm{x}_2 \le b \cdot \norm{x}_1$ $\Longleftrightarrow 
a \cdot \norm{x_n - x}_1 \le \norm{x_n - x}_2 \le b \cdot \norm{x_n - x}_1 \Longleftrightarrow
\lim \norm{x_n - x}_1 \to 0$, $\lim \norm{x_n - x}_2 \to 0$. Обратное доказывается точно так же.

\smallskip 
\noindent \textbullet~Докажем, что $\exists \; const \; a$. Для этого пойдем от противного: пусть $\nexists a $. Тогда $\forall n \in \mathbb{N} \; \exists x_n \in X : \dfrac{1}
{n} \norm{x_n}_1 > \norm{x_n}_2$. Возьмем точку $y_n = \dfrac{x_n}{\norm{x_n}_1}$. Из полувшегося равенства $\norm{y_n}_2 < \dfrac{1}{n} \to 0 \Rightarrow y_n \to 0$ 
по $\norm{\cdot}_2$. А тогда $y_n \to 0$ и по первой норме, однако $\norm{y_n}_1 = \dfrac{\norm{x_n}_1}{\norm{x_n}_1} = 1 \not\to 0$. Получили противоречие с тем, что 
$y_n \to 0$ по первой $\norm{\cdot}_1$. Аналогично доказывается и для  $b$.
\end{proof}
