\subsection*{Определение прямой суммы.}

\subsubsection*{Для 2-х попарно ортогональных подпростанств.}

\noindent\textbullet~В теории гильбертовых пространств важное значение имеет операция \textit{прямой суммы} попарно ортогональных подпространств. Пусть $H_1 \bot H_2$ 
в $H$, то есть $\forall x \in H_1, \forall y \in H_2 \Rightarrow (x, y) = 0 \; [x \bot y = 0]$. Полагаем $H_1 \oplus H_2 = \{ x_1 + x_2, x_1 \in H_1, x_2 \in H_2 \}$ 
- линейное многообразие в $H$. Проверим, что это множество замкнутое. Для этого берем последовательность точек $x_n \in H_1 \oplus H_2$, считаем, что $\exists x = \lim 
x_n$. Необходимо проверить, что $\lim x_n \Rightarrow x \in H_1 \oplus H_2$. 

\smallskip 
\noindent\textbullet~Так как $x_n \in H_1 \oplus H_2 \Rightarrow x_n = x_1^{(n)} +x_2^{(n)}$, где $x_1^{(n)} \in H_1, x_2^{(n)} \in H_2$, $x_1^{(n)} \bot x_2^{(n)}$.

\smallskip 
\noindent\textbullet~Составляем $x_n - x_m = (x_1^{(n)} - x_1^{(m)}) + (x_2^{(n)} - x_2^{(m)})$, тогда по теореме Пифагора $\norm{x_n - x_m}^2 = \norm{x_1^{(n)} - 
x_1^{(m)}}^2 + \norm{x_2^{(n)} - x_2^{(m)}}^2$. Левая часть $\to 0$, тогда и пара слагаемых $\norm{x_1^{(n)} - x_1^{(m)}}^2, \norm{x_2^{(n)} - x_2^{(m)}}^2$ $\to 0$. 
Тогда по полноте $H$ $\exists  x_1 = \lim x_1^{(n)}$, $\exists x_2 = \lim x_2^{(n)}$, причем $H_i$ - замкнуты, поскольку подпространства, тогда $x_i \in H_i$.

\smallskip 
\noindent\textasteriskcentered~Если вернуться к $x_n = x_1^{(n)} + x_2^{(n)}$, $x_n \to x$, $x_1^{(n)} \to x_1$, $x_2^{(n)} \to x_2$. Тогда $x = x_1 + x_2$, а в силу 
$x_i \in H_i \Rightarrow$ $x \in H_1 \oplus H_2$. Таким образом, это многообразие - замкнутое множество, а значит оно подпространство. Это позволяет определить прямую
сумму взаимноортогональных подпростанств, то есть линейное многообразие $H_1 \oplus H_2 = \{x_1 + x_2, x_j \in H_j \}$, которое является подпространством и называется
\textit{прямой суммой} $H_1$ с $H_2$.

\medskip
\noindent\textbullet~Далее в терминах прямой суммы если вернуться к основной теореме теории гильбертовых пространств, то тогда ясно, что эту теорему можно записать 
формулой $H = H_1 \oplus H_1^\bot$. Таким образом, любое гильбертовое пространство может быть разложено в прямую сумму $H_1$ и $H_1^\bot$.


\subsubsection*{Для n попарно ортогональных подпростанств.}

\noindent\textasteriskcentered~Если $H_1, \dots, H_p$ - подпространства и $i \neq j \; H_i \bot H_j$, то $H_1 \oplus \dots \oplus H_p = \{ x_1 + x_2 + \dots + x_p , 
x_i \in H_i \}$. Как выше устанавливается то, что это подпространство и называется \textit{прямой суммой} $H_1, \dots, H_p$ попарно ортогональных подпростанств.


\subsubsection*{Для последовательности попарно ортогональных подпростанств.}

\noindent\textbullet~Теперь перенесем эту операцию на целую последовательность $H_1, H_2, \dots$ попарно ортогональных подпространств. Рассматривать $x_1, x_2, \dots, x_j
\in H_j$ бессмысленно, потому что это ортогональный ряд, его сходимость равносильна сходимости $\sum_1^\infty \norm{x_j}^2$, а этот ряд оказывается расходящимся, потому 
что нормы не стремятся к нулю.

\smallskip 
\noindent\textasteriskcentered~Имея последовательность ортогональных подпространств $H_1, H_2, \dots$ создаем линейное многообразие $\hat{H} : 
\{ \sum_1^n x_k, x_k \in H_k\}$. После этого переходим к замыканию $Cl \hat{H}$ - подпространство $H$. Тогда это замыкание и обозначают $H_1 \oplus H_2 \oplus \dots \oplus \dots$ и называют \textit{ прямой суммой последовательности попарно ортогональных подпространств}. Таким образом обычно в функциональном анализе переносят операцию
с конечным числом слагаемых на операции с бесконечным числом слагаемых.


\subsubsection*{Математический смысл прямой суммы. }

\noindent\textbullet~В следующей теореме приводится без доказательства математический смысл прямой суммы.

\begin{theorem*}
Пусть $H_1, H_2, \dots$ - попарно ортогональные подпространства $H$, $\hat{H} = H_1 \oplus H_2 \oplus \dots$. Тогда $\forall x \in H$ его проекция на $\hat{H}$ $\hat{x}
= x_1 + x_2 + \dots$, где $x_n$ - проекция $x$ на $H_n$.
\end{theorem*}