\documentclass[12pt,a4paper]{report}

\usepackage{setspace, graphicx, fancyhdr, rotating, color}
\usepackage{amsmath, amsfonts}
\usepackage[margin=2cm]{geometry}

\usepackage{tikz}

\usepackage[utf8]{inputenc}
\usepackage[russian]{babel}
\usepackage{setspace}

\usepackage{comment}

\pagestyle{fancy}
\setcounter{tocdepth}{3}

\setlength{\parindent}{0em}
%\setlength{\parskip}{0.5em}

\begin{document}
	
\begin{comment}
	https://www.opennet.ru/docs/RUS/latex/node2.html - секционирование
	https://www.linux.org.ru/forum/general/6926046 - многострочные комментарии
	https://www.opennet.ru/docs/RUS/latex/node2.html - верстка текста
	http://blog.harrix.org/article/661 - ссылки, гиперссылки
	http://mydebianblog.blogspot.com/2011/05/latex.html - оглавления (нумерация, переименование)	
	https://tex.stackexchange.com/questions/45948/what-is-the-difference-between-hspace-fill-and-hfill - горизонтальное заполнение.
	
	http://mif.vspu.ru/books/ASYfb.pdf - рисунки
	https://www.mccme.ru/free-books/llang/newllang.pdf - книжка
	http://www.ict.nsc.ru/jspui/bitstream/ICT/1488/1/pgf-ru-all-method.pdf - tikz
	https://engraver.wordpress.com/2011/08/27/tikz-%D1%87%D0%B5%D1%80%D1%82%D0%B8%D0%BC-%D0%B2-%D0%BB%D0%B0%D1%82%D0%B5%D1%85%D0%B5/ - coordinates
\end{comment}

\thispagestyle{empty}
{\Large Конспект}

\vspace*{230pt}
\begin{center}
\medskip
{\LARGE КОММУНИКАЦИЯ И КОМАНДООБРАЗОВАНИЕ}
\end{center}

\vspace{300pt}
\begin{flushright}
	{\LARGE Кулаков Никита}
\end{flushright}
\vspace*{\fill}
		
\newpage
\setcounter{page}{1}
\section*{Урок 1. Работа в команде.}
$\ast$ Soft Skills -- что-то, что помогает в профессии вне зависимости от специальности
 (коммуникация, коммандообразование, уменить представить себя и т.д).

\medskip
$\ast$ Hard Skills -- то, что относится к специальности (знание языков, математика, физика $\dots$)

\medskip\medskip
STEM - Science Technology Engeneering Mathematics

\medskip
SMART - Specific Measurable Alternable Relevant Timeboard

\bigskip
$\bullet$ Лучше не давать слишком сложные или простые задачи, поскольку сотрудникам будет
тяжело или скучно.

\bigskip
Навыки "умения работать в команде":
\begin{enumerate}
	\item Ответственность (брать на себя)
	\begin{enumerate}
		\item работать над установками и мыслями (проактивное мышление).
		\item быть активными участниками.
		\item перестать обвинять и оправдываться.
		\item проанализировать свое поведение в работе.
		\item делать вывода на основе работы.
	\end{enumerate}
	\item Постановка целей
	\begin{enumerate}
		\item формулировка задач.
		\item соотнести их с целями команды.
		\item определять задачи по достижению целей.
		\item прояснять цели с другими участниками.
	\end{enumerate}
	\item Мотивация
	
	\medskip
	\begin{minipage}{0.4\textwidth}
		$\bullet$ интерес
		
		$\bullet$ профессиональный рост
		
		$\bullet$ стремление к достижениям
		
		$\bullet$ стремление к власти
	\end{minipage}
	\hfill
	\begin{minipage}{0.4\textwidth}
		$\bullet$ признание
		
		$\bullet$ социальная значимость
		
		$\bullet$ цели
		
		$\bullet$ высокая степень ответственности
	\end{minipage}

\medskip
$\ast$ Ценности -- идеальное значение и смысл вещей и явлений, происходящих в жизни человека.

\medskip
Мотивация коллег:

	$\hspace{1em}\circ$ узнать, что их мотивирует
	
	$\hspace{1em}\circ$ быть достойным их ожиданиям
	
	$\hspace{1em}\circ$ признание и учитывание их ценностей
	
\item Доверие
\begin{enumerate}
	\item соблюдение договоренностей.
	\item выстраивание доверия (предупреждайте о задержках).
	\item выполняйте свои обещания.
\end{enumerate}

\item Понимание своей роли в команде

Роли:
\begin{itemize}
	\item Технические.
	\item Социальные (эмоциональный интеллект, коммуникация).
\end{itemize}
\end{enumerate}

\medskip
$\bullet$ Групповые эффекты -- закономерности, характерные для поведения человека в 
группе. Свойственны большим и малым группам, но сила и характер их протекания всегда 
различны, так как поведение в каждой группе уникально и обусловлено множеством факторов.

\bigskip
$\diamond$ Норман Триплетт: обнаружил особенности реакции человека на присутствие других людей.

\medskip
$\ast$ Социальная фасилитация -- простые и примитивные действия лучше выполняются в присутствии других людей.

\medskip 
$\ast$ Социальная ингибиция -- трудные задачи в присутствии других людей делаются хуже.

\medskip
$\rightarrow$ Советы:
\begin{itemize}
	\item научиться управлять своим вниманием и концентрироваться.
	\item создавать ритуалы, чтобы настроиться на работу.
	\item перед групповым обсуждением обдумать решение индивидуально.
	\item добавлять в деятельность элементы геймификации.
\end{itemize}

\bigskip
$\diamond$ Эффект социальной лени (Марк Рингельман): уменьшение индивидуальных усилий при совместной работе.

\medskip
$\rightarrow$ Как снижать социальная лень:
\begin{itemize}
	\item оценивать не только суммарный результат, но и индивидуальный.
	\item создвавать дружеские отношения в командной работе.
	\item оптимизировать численность группы.
	\item учитывать гендерные различия. 
\end{itemize}

\bigskip
$\diamond$ Эффект группомыслия: единство мнений важнее мнения отдельного человека.

\medskip 
$\rightarrow$ Как преодолеть эффект группомыслия:
\begin{itemize}
	\item создавать динамическую активность, предлагая противоположные идеи.
	\item вводить правила оценки всех предложений 
	\item предлагать самым авторитетным участникам высказываться последними.
\end{itemize}

\bigskip
$\diamond$ Эффект комформизма -- категорическое мнение большинства цленов группы относительно какого-либо вопроса приводит к узменению поведения остальных членов группы, даже если они считают данное мнение ошибочным.

\newpage
\medskip 
$\rightarrow$ Как преодолеть эффект комформизма:
\begin{itemize}
	\item повышать собственную компетентность.
	\item брать дополнительное время на обдумывание.
	\item формировать уверенность в себе.
	\item развивать критическое мышление.
	\item использовать письменное голосование.
\end{itemize}

\bigskip 
$\diamond$ Эффект групповой идентичности (своя группа лучшая, а другая враждебная; тенденция поддерживать всегда членов своей группы, даже если она сформировалась только что; успехи приписывать своей группе, а неудачи - внешним причинам):

\medskip
$\rightarrow$ Решения:
\begin{itemize}
	\item не огриничиваться членством в одной группе.
	\item разнообразие задач и ролей.
	\item стараться отмечать положительные стороны даже в командах-конкурентах.
\end{itemize}

\bigskip
$\diamond$  Эффект синергии -- продукт командной работы больше суммы индивидуальных каждого.

\bigskip 
$\ast$ Фрилансер -- нанимается на разовые или краткосрочные задачи.


\medskip
$\ast$ Удаленный сотрудник -- человек, который нацелен на сотрудничество с компанией в течение долгого времени.

\bigskip 
$\rightarrow$ Предпосылки перехода на удаленный формат:
\begin{itemize}
	\item развитие технологий.
	\item желание специалистов работать удаленно.
	\item необходимость оптимизации расходов на офис.
	\item присутствуют обстоятельства, вынуждающие работать в удаленке. 
\end{itemize}

\smallskip
$\rightarrow$ Минусы:
\begin{enumerate}
	\item слабый командный дух. 
	\item разница часовых поясов.
	\item снижение ответственности.
	\item разница в опыте.
	\item высокая текучесть кадров.
	\item трудность в оценке мотивации сотрудников.
	\item сторание границы "работа-дом"
\end{enumerate}

\newpage
$\rightarrow$ Рекомендации:
\begin{itemize}
	\item не ограничиваться текстовыми сообщениями.
	\item создавать фреймворк взаимодействий.
	\item максимально дробить большие задачи. 
	\item закладывать дополнительное время на каждую задачу.
	\item использовать методы Agile/Scrum/Kanban.
\end{itemize}

\medskip
$\rightarrow$ Правила проведения удаленных совещаний:
\begin{enumerate}
	\item говорить меньше, чем обычно.
	\item озвучивайте правила поведения на совещании.
	\item вовлекайте в обсуждение всех участников.
	\item смотрите в камеру.
	\item поддерживайте видимость рабочей атмосферы.
	\item привлекайте помощников.
	\item остерегайтесь "троллей".
\end{enumerate}

\bigskip
$\rightarrow$ Инструменты для работы в удаленной команде:
\begin{enumerate}
	\item Мессенджеры и коммуникаторы:
	\begin{itemize}
		\item Slack
		\item Zoom
	\end{itemize}
	\item Аналитические инструменты:
	\begin{itemize}
		\item Mentimeter - интерактивные презентации.
		\item Prezi - динамические презентации.
		\item Miro - онлайн-доски.
	\end{itemize}
	\item Редакторы данных и хранилища информации:
	\begin{itemize}
		\item Google Docs
		\item Skitch - фото, заметки, комментарии.
		\item Google Drive, Yandex.Disk, Dropbox.
	\end{itemize}
	\item Планеры:
	\begin{itemize}
		\item Trello
		\item Microsoft Teams - чат, заметки, общий планер.
		\item Teamer
	\end{itemize}
\end{enumerate}

\newpage
\section*{Урок 2. Роли в команде.}
$\diamond$ Мередит Белбин: исследование ролей в командах (9 командных социальных ролей).

\medskip 
$\rightarrow$ Роли у человека:

\smallskip
$\bullet$ Основная роль

$\bullet$ Дополнительная роль

\medskip 
$\rightarrow$ Роли:
\begin{enumerate}
	\item Ориентация на идеи:
	\begin{itemize}
		\item генератор идей
		\item аналитик
		\item специалист (не комендуют на длительной основе заниматься)
	\end{itemize}
	\item Ориентация на достижение целей:
	\begin{itemize}
		\item формирователь
		\item реализатор
		\item завершитель
	\end{itemize}
	\item Ориентация на работу с людьми:
	\begin{itemize}
		\item дипломат
		\item исследователь ресурсов
		\item координатор
	\end{itemize}
\end{enumerate}

\medskip 
$\rightarrow$ Выводы:
\begin{enumerate}
	\item Одноролевые команды неэффективны.
	\item Эффективное число участников 4-6.
	\item Уровень интеллекта - фактор успеха.
	\item Любая командная роль сопоставима с понятием лидерства.
	\item Человек обладает 2 соц.ролями: основной и дополнительной.
\end{enumerate}

\bigskip
$\ast$ Социальная группа -- устойчивая совокупность людей, которая имеет отличные, только ей присущие признаки.

\medskip 
$\ast$ Малая социальная группа -- группа непосредственно контакцирующих индивидов, объединенных общими интересами, целями и групповыми нормами поведения.

\bigskip 
$\diamond$ Брюс Такман "Стадии развития команды".

\bigskip 
\begin{minipage}{0.4\textwidth}
	$\rightarrow$ Стадии: 
	\begin{enumerate}
		\item Формирование.
		\item Шторм.
		\item Нормализация.
		\item Работоспособность.
		\item Распад.
	\end{enumerate}
\end{minipage}
\hspace*{\fill}
\begin{minipage}{0.4\textwidth}
	\begin{tikzpicture}
		\draw (-0.25,0) -- (7,0);
		\draw (0,-0.25) -- (0,3.5);
		
		\draw (0,0) .. controls (1,3) and (2,3) .. (2.5,1)
		.. controls (3,-0.5) and (3.5, 0.5) .. (5,2)
		.. controls (5.8,2.6) .. (7, 3);
		
		\coordinate [label=above:$1$] (1) at (0.7,2);
		\coordinate [label=above:$2$] (2) at (1.6,2.5);
		\coordinate [label=above:$3$] (3) at (3.1,0.6);
		\coordinate [label=above:$4$] (4) at (5,2.2);
		\coordinate [label=above:$5$] (5) at (6.5,3);
	\end{tikzpicture}
\end{minipage}
\newpage

$\bullet$ Формирование:
\begin{itemize}
	\item[$\circ$] знакомство, определение норм поведения, целей и задач, вежливость и осторожность в общении
	\item[$\circ$] осознание причин, целей, ролей.
\end{itemize}

$\bullet$ Шторм:
\begin{itemize}
	\item[$\circ$] конфликты, превосходство личных целей над командными, низкая эффективность работы, закрепление норм и ролей.
	\item[$\circ$] установление и принятие групповых норм.
\end{itemize}

$\bullet$ Нормализация:
\begin{itemize}
	\item[$\circ$] сотрудничество и синергия, конструктивное планирование работы и задач.
\end{itemize}

$\bullet$ Работоспособность:
\begin{itemize}
	\item[$\circ$] ответственные и открытые коммуникации, работа над задачей, пик эффективности.
	\item[$\circ$] результат работы достигается.
	\item[$\circ$] делегирование полномочий, оценка результатов деятельности и обратная связь - задачи лидера.
\end{itemize}

$\bullet$ Распад:
\begin{itemize}
	\item[$\circ$] подведение итогов, обратная связь, планирование будущего.
\end{itemize}

\subsection*{Ответы к видео}
\begin{enumerate}
	\item Мнение самого себя важнее мнения других: мы неохотно договариваемся. Нарушение баланса между слушание и говорением связано с развитием технологий (телефон: проще напечатать, чем сказать в лицо).
	
	 Нет смысла делать вид, что вы находитесь в разговоре, если вы и так в нем находитесь.
	\item Правило <<присутствия в разговоре>>: участие в общении, осознавание происходящего, способность что-то понять после разговора.
	
	\smallskip 
	$\bullet$ Не мультитаскить во время разговора.
	
	$\bullet$ Не умничать и важничать.
	
	\item  В процессе разговора не следует думать, что твое мнение единственное важное. Научиться слушать других и осознавать, что твоя точка зрения в каких-то моментах может уступать другим.
	
	\smallskip
	$\bullet$ Задавайте открытые (специальные) вопросы.
	
	$\bullet$ Плывите по течению (мысли пришли - высказаться сразу или отпустить).
	
	$\bullet$ Если вы знаете, то сказать, иначе - описать незнание.
	
	$\bullet$ Не нужно проводить параллели опыта собеседника с собой, поскольку каждый опыт уникален.
	
	$\bullet$ Conversations aren't the promotion opportunity.
	
	\item Да, я согласен с тем, что написано чуть выше, так как иначе ваши собеседник будет думать, что из всего разговора вы пытаетесь только получить выгоду для себя.
	
	\smallskip
	$\bullet$ Не повторять сказанное.
	
	$\bullet$ Избегать лишних деталей.
	
	$\bullet$ Слушать (слушать не для того, чтобы что-то сказать).
	
	$\bullet$ Быть краткими.
	
	$\bullet$ Быть готовыми удивляться.
	
	\item Для успешного разговора необходимо не только умение говорить, но и способность слушать окружающих, удивляться.
\end{enumerate}

\section*{Урок 3. Цикл коммуникации}
$\ast$ Реципиент -- тот, кто принимает информацию.

\bigskip
$\rightarrow$ Цикл коммуникации (Шаги):
\begin{enumerate}
	\item Проясните цель коммуникации. (способ общения, настроение, уровень знаний аудитории, манера речи и т.п)
	\item Сформулируйте ваше сообщение.
	\item Передайте сообщение (учитывать время отправки информации, контекст, отвлекающие факторы)
	\item Получение обратной связи. (невербальный отклик, ответ)
	\item Декодирование обратной связи, получение информации.
	\item Улучшайте вашу коммуникацию. 
\end{enumerate}

\medskip
$\ast$  Конгруэнтность в коммуникации -- совпадение ваших мыслей, слов, невербальных проявлений и действий.

\medskip
$\ast$ Зеркальные нейроны -- нейроны головного мозга, которые возбуждаются как при выполнении определенного действия, так и при наблюдении за выполнением этого действия другим человеком.

\medskip
Виды коммуникации:
\begin{itemize}
	\item Вербальное (устная, письменная речь).
	\item Невербальное (мимика, жесты, позы, интонация, походка, взгляд).
\end{itemize}
\medskip 
Слушание:
\begin{itemize}
	\item Активное (попытки побудить партнера к разговору, точно воспринять сказанное партнером, убедиться в точности восприятия, удержать партнера в ходе данной темы, не давая ему уйти в сторону).
	\item Пассивное (терпеливое ожидание того, чтобы партнер заговорил, отвлечение внимания на что-либо другое при сохрании маски внимания, ожидание, пока партнер вернется к теме разговора).
\end{itemize}

\medskip 
$\rightarrow$ Техники, способствующие пониманию:
\begin{itemize}
	\item Вербализация через повторение.
	\item Вербализации через перефразирование.
	\item Вербализация через интерпритацию и развитие идеи.
\end{itemize}

\bigskip 
$\rightarrow$ Типичные ошибки вербализации и способы их преодоления:
\begin{itemize}
	\item безапелляционность (констатация правильности вместо проверки ясного понимания).
	\item навязчивое повторение (диалог становится односторонним, другой человек не ощущает вашего вклада в разговор).
	\item ложная интерпретация (неточное предположение о намерениях, мыслях или чувствах другого человека).
	\item слишком точная интерпретация (лучше говорить в форме уточняющего вопроса или пробной гипотезы).
\end{itemize}

\bigskip
$\rightarrow$ {\large Стили коммуникации}:
	 
\bigskip
\begin{minipage}{0.45\textwidth}
		{\large Прямой}
	
	\smallskip
	$\circ$ Прямолинеен в намерениях
	
	$\circ$ Возможна категоричность
	
	$\circ$ Часто использует местоимение <<Я>>
\end{minipage}
\hfill
\begin{minipage}{0.45\textwidth}
		{\large Непрямой}
		
	\smallskip
	$\circ$ Немерения коммуникации скрыто
	
	$\circ$ Использование неопределенных слов
	
	$\circ$ Редко использует местоимение <<Я>>
\end{minipage}

\bigskip
\begin{minipage}{0.45\textwidth}
	{\large Настойчивый}
	
	\smallskip
	$\circ$ Лидирует в коммуникации, начинает говорить первым
	
	$\circ$ Перебивает, может быть грубым
	
	$\circ$ Часто выражает несогласие
\end{minipage}
\hfill
\begin{minipage}{0.45\textwidth}
	{\large Аффилятивный}
	
	\smallskip
	$\circ$ Нередко ведом в общении
	
	$\circ$ Хороший слушатель
	
	$\circ$ Ищет точки соприкосновения
\end{minipage}

\bigskip
\begin{minipage}{0.45\textwidth}
	{\large Сжатый}
	
	\smallskip
	$\circ$ Только необходимые детали
	
	$\circ$ Использует простые предложения
	
	$\circ$ Приводит мало примеров
\end{minipage}
\hfill
\begin{minipage}{0.45\textwidth}
	{\large Вычурный}
	
	\smallskip
	$\circ$ Говорит предложениями с большим количеством оборотов
	
	$\circ$ Богатая речь (большой лексикон)
	
	$\circ$ Ищет точки соприкосновения
\end{minipage}

\bigskip
$\rightarrow$ Как использовать стили?
\begin{itemize}
	\item Важно понимать, какой стиль у вас.
	\item Необходимо наблюдать за собеседником и определять какой стиль свойственен ему, чтобы вам под него подстроиться.
	\item Адаптируйте свой стиль, если является коммуникатором.
	\item Учитывайте особенности стиля коммуникации при интерпретации сообщения.
\end{itemize}

\newpage
\section*{Урок 4. Самопрезентация}
$\ast$ Нетворкинг -- социальная и профессиональная деятельность, направленная на то, чтобы с помощью друзей и знакомых эффективнее решать различные задачи. (создание вокруг себя круга различных людей, умение поддерживать взаимоотношение так, чтобы можно было к ним всегда обратиться)

\medskip
$\bullet$ Нетворкинг -- взаимовыгодный процесс.

\bigskip
Правила нетворкинга:
\begin{itemize}
	\item Будьте открыты и доброжелательными.
	\item Запаситесь визитными картами.
	\item Уберите гаджеты в карман.
	\item Активно знакомьтесь.
	\item Подготовьте коротную самопрезентацию. <<Презентация для лифта - Elevator speech>>
	\item Сохраняйте формат диалога.
	\item Поддерживайте знакомства, а также обменяйтесь контактами и назначьте встречу.
\end{itemize}

\smallskip 
$\bullet$ Составьте несколько коротких самопрезентаций для знакомства.

\bigskip
$\diamond$ Сетевой нетворкинг: 

\smallskip
$\ast$ Теория шести рукопожатий -- социологическая теория, согласно которой любые два человека на Земле разделены не более чем пятью уровнями общих знакомых.


\subsection*{Письменные коммуникации}
$\bullet$ То, как вы пишите, влияет на ваши отношения людьми.

\bigskip
$\rightarrow$ Что вызывает раздражение в переписке?
\begin{enumerate}
	\item Небрежность (опечатки, ошибки, неверный адресат).
	\item Панибратство - не спешить переходить на <<ты>>.
	\item Некорректный тон письма. (задавайте себе вопрос: <<Как бы я разговаривал с этим человек при реальном общении?>>).
	\item Клише и отстутствие <<заботы>> о читателе.
	\item Спешка и срочность.
	\item Неверный выбор платформы для коммуникации (уточняйте удобный для человека способ и время для общения, уважайте личные границы).
\end{enumerate}

$\bullet$ Человеку не стоит писать на мессенджеры, в случае деловой переписки, или если переходите, то это должно быть обговорено. Также не нужно тревожить его в неудобное время.

\newpage
\medskip
$\rightarrow$ Забота о читателе:
\begin{itemize}
	\item Сформулируйте цель вашего письма.
	\item Обозначьте тему максимально корректно.
	\item Структурируйте ваши письма.
	\item Подготовьте дополнительные материалы для лучшего понимания.
	\item Не перегружайте адресата.
	\item Указывайте свои контактные данные, чтобы с вами можно было связаться.
\end{itemize}

\medskip 
$\rightarrow$ Что важно еще помнить?
\begin{enumerate}
	\item Выбирайте подходящий адрес электронной почты.
	\item Используйте смайлики с аккуратностью.
	\item Избегайте большого количества одинаковых знаков и написание капслоком.
	\item Обращайтесь к читателю по имени.
	\item Благодарите.
	\item Не пишите на эмоциях.
\end{enumerate}

\bigskip 
Два простых правила: 
\begin{itemize}
	\item[\checkmark] Не пишите то, что нельзя переслать. 
	\item[\checkmark] Не пишите то, что было бы стыдно увидеть на первой полосе Wall Street Journal.
\end{itemize}
\end{document}