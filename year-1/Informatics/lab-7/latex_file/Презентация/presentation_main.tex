\documentclass[aspectratio=169, 11pt]{beamer}
\usepackage[utf8]{inputenc}
\usepackage[russian]{babel}
\usepackage[T2A]{fontenc}
\usetheme{Pittsburgh}
\usecolortheme{dove}
\usepackage{ragged2e}
\usepackage{color}
\usepackage[absolute, overlay]{textpos}
\usepackage{graphicx}  % Для вставки рисунков
\graphicspath{{images/}}  % папки с картинками
\usepackage{setspace}
\setstretch{0,9}

\title{Определение термина Информатика}
\subtitle{Лекция №1}
\author{Кулаков Н.\,В. P3130}
\date{2020 год}
\institute{Факультет ПИиКТ}

% установка единиц измерения по горизонтали и вертикали
\setlength{\TPHorizModule}{1mm}
\setlength{\TPVertModule}{1mm}
\begin{document}
		
	\begin{frame}
		\maketitle
	\end{frame}


	\begin{frame}
		
		\begin{textblock}{50}(32,13)
			\includegraphics[scale=0.4]{Screenshot_1.png}
		\end{textblock}
		
		\begin{textblock}{50}(4,4)
			\includegraphics[scale=0.25]{Screenshot_2.png}
		\end{textblock}
		

		\frametitle{\textbf{\textcolor[rgb]{0.4,0.4,0.4}{\textcolor[rgb]{0,0,1}{О}пределение термина <<Информатика>>}}}
		
		\textcolor[rgb]{0,0.64,0.09}{Информатика} - дисциплина, изучающая свойства и структуру информации, закономерности ее создания, преобразования, накопления, передачи и использования.
		
		\textcolor[rgb]{0,0.64,0.09}{Англ}: informatics = information technology $+$ computer science $ + $ information theory
\vspace{3mm}
	
		\begin{center}
			\textbf{Важные даты}
		\end{center}
	\begin{itemize}
		\item1956 - появление термина <<Информатика>> (нем. Informatik, Штейнбух)
		
		\item1968 - первое упоминание в СССР (информология, Харкевич)
		
		\item197X - информатика стала отдельной наукой
		
		\item4 декабря - день российской информатики
	\end{itemize}
	\end{frame}
	
	
	\begin{frame}
		\frametitle{\textbf{\textcolor[rgb]{0.4,0.4,0.4}{\textcolor[rgb]{0,0,1}{Т}ерминолония: информация и данные}}}
		
\begin{textblock}{50}(32,13)
	\includegraphics[scale=0.4]{Screenshot_1.png}
\end{textblock}
		
		\begin{textblock}{50}(4,4)
			\includegraphics[scale=0.25]{Screenshot_2.png}
		\end{textblock}
	
		Международный стандарт ISO/IEC 2382:2015
		
		<<Information technology - Vocabulary>> (вольный пересказ):
		\vspace{2mm}
		
		~~~~~~\textcolor[rgb]{0,0.64,0.09}{Информация} - знания относительно фактов, событий, вещей, идей и понятий.
		
		~~~~~~\textcolor[rgb]{0,0.64,0.09}{Данные} - форма представления информации в виде, пригодном для передачи или обработки.
		\vspace{4mm}
		
		\begin{itemize}
		\itemЧто есть предмет информатики: информация или данные?
		
		\itemКак измерить информацию? Как измерить данные?
		\end{itemize}
		~~~~~~Пример: <<Байкал - самое глубокое озерол Земли>>.
	\end{frame}
	
	
	\begin{frame}
		\frametitle{\textbf{\textcolor[rgb]{0.4,0.4,0.4}{\textcolor[rgb]{0,0,1}{И}змерение количества информации}}}
		
\begin{textblock}{50}(32,13)
	\includegraphics[scale=0.4]{Screenshot_1.png}
\end{textblock}
		
		\begin{textblock}{50}(4,4)
			\includegraphics[scale=0.25]{Screenshot_2.png}
		\end{textblock}
	
		\textcolor[rgb]{0,0.64,0.09}{Количество информации $\equiv$ информационная энтропия} - это численная мера непредсказуемости информации. Количество информации в некотором объекте определяется непредсказуемостью состояния, в котором находится этот объект.
		\vspace{4mm}
		
		Пусть i (s) -  функция для измерения количеств информации в объекте s, состоящем из n независимых частей sk, где k  измеряется от 1 до n. \textcolor[rgb]{0,0.64,0.09}{Тогда свойства меры количества информации} \textbf{i(s)} таковы:
		\begin{itemize}
		\itemНеотрицательность: i(s) >= 0.
		
		\itemПринцип предопределенности: если об объекте уже все известно, то i(s) = 0.
		
		\itemАддитивность: i(s) = sum i(sk) по всем k.
		
		\itemМонотонность: i(s) монотонна при монотонности измерени вероятностей.
		\end{itemize}
		\end{frame}
	
	
	\begin{frame}
		\frametitle{\textbf{\textcolor[rgb]{0.4,0.4,0.4}{\textcolor[rgb]{0,0,1}{П}ример применения меры Хартли на практике}}}
		
\begin{textblock}{50}(12,13)
	\includegraphics[height=0.08 cm, width = 13.6 cm]{Screenshot_1.png}
\end{textblock}
	
		\textbf{Пример 1}. Ведущий загадывает число от 1 до 64. Какое количество вопросов типа <<да-нет>> понадобится, чтобы гарантировано угадать число?
		\begin{itemize}
		\itemПервый вопрос: <<Загаданное число меньше 32?>>. Ответ: <<Да>>.
		
		\itemВторой вопрос: <<Загаданное число меньше 16?>>. Ответ: <<Нет>>.
		
		\item$\dots$
		
		\itemШестой вопрос (в худшем случае) точно приведет к верному ответу.
		
		\itemЗначит, в соответствии с мерой Хартли в загадке ведущего содержится ровно log 2 64 = 6 бит непредсказуемости (т.е информации).
		\end{itemize}
		\vspace{4mm}
		
		\textbf{Пример 2}. Ведущий держит за спиной ферзя и собирается поставить его на произвольную клетку доски. Насколько непредсказуемо его решение?
		\begin{itemize}
		\itemВсего на доске 8x8 клеток, а цвет ферзя может быть белым или черным, т.е всего возможно 8x8x2 = 128 равновероятных состояний.
		
		\itemЗначит, количество информации по Хартли равно log 2 128 = 7 бит.
		\end{itemize}
	\end{frame}
	
	
	\begin{frame}
		\frametitle{\textbf{\textcolor[rgb]{0.4,0.4,0.4}{\textcolor[rgb]{0,0,1}{А}нализ свойств меры Хартли}}}
		
		\begin{textblock}{50}(32,13)
			\includegraphics[scale=0.4]{Screenshot_1.png}
		\end{textblock}
		
		\begin{textblock}{50}(4,4)
			\includegraphics[scale=0.25]{Screenshot_2.png}
		\end{textblock}
	
		Экспериментатор одновременно подбрасывает монету $ ( $M$ ) $ и кидает игральную кость$ ( $K$ ) $. Какое количество информации содержится в эксперименте $ ( $Э$ ) $?
		\vspace{4 mm}
		
		\textcolor[rgb]{0,0.64,0.09}{Аддитивность}:
		
		\qquad $i($Э$)  = i($M$)  + i($K$) \geq i($2 исхода$) + i($6 исходов$): \log_x 12 = \log_x 2 + \log_x 6$
		
		\vspace{2mm}
		\textcolor[rgb]{0,0.64,0.09}{Неотрицательность}:
		
		\qquad Функция $\log_x N$ неотрицательна при любом $x > 1$ и $N \geq 1.$
		\vspace{2 mm}
		
		\textcolor[rgb]{0,0.64,0.09}{Монотонность}:
		
		\qquad С увеличением $p($M$)$ или $p($K$)$ функция $i($Э$)$ монотонно возрастает.
		\vspace{2 mm}
		
		\textcolor[rgb]{0,0.64,0.09}{Принцип предопределенности}:
		
		\qquad При наличии всегда только одного исхода (монета и кость с магнитом) количество информации равно нулю: $\log_x 1 + \log_x 1 = 0.$
	\end{frame}

\end{document}
