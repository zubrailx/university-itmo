\begin{markdown}
# Этап 3

## Запросы

### Для администраторов системы
* Редактирование доступных единиц измерения
* Редактирование доступных ингредиентов

### Для поставщика
* Создание и удаление аккаунта поставщика
* Размещение цен на товары

### Для владельца кафе
* Добавление, изменение, удаление кафе 
* Добавление, удаление, изменение меню
* Добавление, удаление меню в кафе
* Добавление, изменение, удаление рецептов
* Добавление, удаление рецептов в меню
* Поиск товаров у поставщиков
* Покупка товара у поставщиков
* Получение информации о купленных товаров, т.е. кассового чека, и обновление склада при изменении статуса чека на получено
* Редактирование и удаление записей об инвенторе

### Для пользователя
* Просмотр меню кафе
* Просмотр рецептов блюд

## Создание объектов

Идет в правильном порядке запуска скриптов.

### Таблицы
\end{markdown}
\lstinputlisting[
  style = User-SQL,
  % caption = {create-functions.sql}
  ]{../../backend/sql/ddl/create-scheme.sql}

\begin{markdown}
### Функции
\end{markdown}
\lstinputlisting[
  style = User-SQL,
  % caption = {create-functions.sql}
  ]{../../backend/sql/ddl/create-functions.sql}

\begin{markdown}
### Процедуры
\end{markdown}
\lstinputlisting[
  style = User-SQL,
  % caption = {create-procedures}
  ]{../../backend/sql/ddl/create-procedures.sql}

\begin{markdown}
## Удаление объектов

### Таблицы
\end{markdown}
\lstinputlisting[
  style = User-SQL,
  % caption = {drop-procedures}
  ]{../../backend/sql/ddl/drop-scheme.sql}

\begin{markdown}
### Процедуры
\end{markdown}
\lstinputlisting[
  style = User-SQL,
  % caption = {drop-scheme}
  ]{../../backend/sql/ddl/drop-procedures.sql}

\begin{markdown}
### Функции
\end{markdown}
\lstinputlisting[
  style = User-SQL,
  % caption = {drop-functions}
  ]{../../backend/sql/ddl/drop-functions.sql}

\begin{markdown}
## Начальные значения

### Создание
\end{markdown}
\lstinputlisting[
  style = User-SQL,
  % caption = {drop-functions}
  ]{../../backend/sql/init.sql}

\begin{markdown}
### Удаление
\end{markdown}
\lstinputlisting[
  style = User-SQL,
  % caption = {drop-functions}
  ]{../../backend/sql/destroy.sql}

\begin{markdown}
## Примеры запросов выше

### Администратор
\end{markdown}
\lstinputlisting[
  style = User-SQL,
  % caption = {drop-functions}
  ]{../../backend/sql/req/admin.sql}

\begin{markdown}
### Поставщик
\end{markdown}
\lstinputlisting[
  style = User-SQL,
  % caption = {drop-functions}
  ]{../../backend/sql/req/supplier.sql}

\begin{markdown}
### Владельцы
\end{markdown}
\lstinputlisting[
  style = User-SQL,
  % caption = {drop-functions}
  ]{../../backend/sql/req/owner.sql}

\begin{markdown}
### Пользователи
\end{markdown}
\lstinputlisting[
  style = User-SQL,
  % caption = {drop-functions}
  ]{../../backend/sql/req/user.sql}

\begin{markdown}
## Индексы

### Описание
По умолчания для primary key PSQL использует btree. На основании полученная даталогической модели и написанным выше запросам были выбраны следующие индексы, исключая индексы по умолчанию:
\end{markdown}

\begin{itemize}
  \item \Verb"prices(nominal_price)" - btree. Будет производиться сортировка чисел.
  \item \Verb"stocks(expiry_date)" - btree. Осуществление сортировки и фильтрации по времени окончания срока хранения.
  \item \Verb"cafes(name)" - gin. Поиск кафе по паттерну.
  \item \Verb"menus(name)" - gin. Поиск позиции в меню по паттерну.
  \item \Verb"recipes(name)" - gin. Поиск рецепта по паттерну.
  \item \Verb"ingredients(name)" - gin. Поиск ингредиента по паттерну.
\end{itemize}

Однако поскольку вылетает ошибка при попытке создания gin и мы не имеем достаточными правами доступа к дб, чтобы ее исправить, то обойдемся hash. Описание ошибки:

\begin{lstlisting}[
  style = User-output
]
psql:create-indexes.sql:5: ERROR:  data type character varying has no default operator class for access method "gin"
ПОДСКАЗКА:  You must specify an operator class for the index or define a default operator class for the data type.
\end{lstlisting}

\begin{markdown}
### Реализация
\end{markdown}

\lstinputlisting[
  style = User-SQL,
  % caption = {drop-functions}
  ]{../../backend/sql/ddl/create-indexes.sql}
